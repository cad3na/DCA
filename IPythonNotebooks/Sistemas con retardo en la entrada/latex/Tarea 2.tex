
% Default to the notebook output style

    


% Inherit from the specified cell style.




    
\documentclass{article}

    
    
    \usepackage{graphicx} % Used to insert images
    \usepackage{adjustbox} % Used to constrain images to a maximum size 
    \usepackage{color} % Allow colors to be defined
    \usepackage{enumerate} % Needed for markdown enumerations to work
    \usepackage{geometry} % Used to adjust the document margins
    \usepackage{amsmath} % Equations
    \usepackage{amssymb} % Equations
    \usepackage[mathletters]{ucs} % Extended unicode (utf-8) support
    \usepackage[utf8x]{inputenc} % Allow utf-8 characters in the tex document
    \usepackage{fancyvrb} % verbatim replacement that allows latex
    \usepackage{grffile} % extends the file name processing of package graphics 
                         % to support a larger range 
    % The hyperref package gives us a pdf with properly built
    % internal navigation ('pdf bookmarks' for the table of contents,
    % internal cross-reference links, web links for URLs, etc.)
    \usepackage{hyperref}
    \usepackage{longtable} % longtable support required by pandoc >1.10
    \usepackage{booktabs}  % table support for pandoc > 1.12.2
    \usepackage{mathpazo}
    \usepackage[spanish]{babel}
    

    
    
    \definecolor{orange}{cmyk}{0,0.4,0.8,0.2}
    \definecolor{darkorange}{rgb}{.71,0.21,0.01}
    \definecolor{darkgreen}{rgb}{.12,.54,.11}
    \definecolor{myteal}{rgb}{.26, .44, .56}
    \definecolor{gray}{gray}{0.45}
    \definecolor{lightgray}{gray}{.95}
    \definecolor{mediumgray}{gray}{.8}
    \definecolor{inputbackground}{rgb}{.95, .95, .85}
    \definecolor{outputbackground}{rgb}{.95, .95, .95}
    \definecolor{traceback}{rgb}{1, .95, .95}
    % ansi colors
    \definecolor{red}{rgb}{.6,0,0}
    \definecolor{green}{rgb}{0,.65,0}
    \definecolor{brown}{rgb}{0.6,0.6,0}
    \definecolor{blue}{rgb}{0,.145,.698}
    \definecolor{purple}{rgb}{.698,.145,.698}
    \definecolor{cyan}{rgb}{0,.698,.698}
    \definecolor{lightgray}{gray}{0.5}
    
    % bright ansi colors
    \definecolor{darkgray}{gray}{0.25}
    \definecolor{lightred}{rgb}{1.0,0.39,0.28}
    \definecolor{lightgreen}{rgb}{0.48,0.99,0.0}
    \definecolor{lightblue}{rgb}{0.53,0.81,0.92}
    \definecolor{lightpurple}{rgb}{0.87,0.63,0.87}
    \definecolor{lightcyan}{rgb}{0.5,1.0,0.83}
    
    % commands and environments needed by pandoc snippets
    % extracted from the output of `pandoc -s`
    \DefineVerbatimEnvironment{Highlighting}{Verbatim}{commandchars=\\\{\}}
    % Add ',fontsize=\small' for more characters per line
    \newenvironment{Shaded}{}{}
    \newcommand{\KeywordTok}[1]{\textcolor[rgb]{0.00,0.44,0.13}{\textbf{{#1}}}}
    \newcommand{\DataTypeTok}[1]{\textcolor[rgb]{0.56,0.13,0.00}{{#1}}}
    \newcommand{\DecValTok}[1]{\textcolor[rgb]{0.25,0.63,0.44}{{#1}}}
    \newcommand{\BaseNTok}[1]{\textcolor[rgb]{0.25,0.63,0.44}{{#1}}}
    \newcommand{\FloatTok}[1]{\textcolor[rgb]{0.25,0.63,0.44}{{#1}}}
    \newcommand{\CharTok}[1]{\textcolor[rgb]{0.25,0.44,0.63}{{#1}}}
    \newcommand{\StringTok}[1]{\textcolor[rgb]{0.25,0.44,0.63}{{#1}}}
    \newcommand{\CommentTok}[1]{\textcolor[rgb]{0.38,0.63,0.69}{\textit{{#1}}}}
    \newcommand{\OtherTok}[1]{\textcolor[rgb]{0.00,0.44,0.13}{{#1}}}
    \newcommand{\AlertTok}[1]{\textcolor[rgb]{1.00,0.00,0.00}{\textbf{{#1}}}}
    \newcommand{\FunctionTok}[1]{\textcolor[rgb]{0.02,0.16,0.49}{{#1}}}
    \newcommand{\RegionMarkerTok}[1]{{#1}}
    \newcommand{\ErrorTok}[1]{\textcolor[rgb]{1.00,0.00,0.00}{\textbf{{#1}}}}
    \newcommand{\NormalTok}[1]{{#1}}
    
    % Define a nice break command that doesn't care if a line doesn't already
    % exist.
    \def\br{\hspace*{\fill} \\* }
    % Math Jax compatability definitions
    \def\gt{>}
    \def\lt{<}
    % Document parameters
    \title{Tarea 2 - Sistemas con retardos en la entrada}
    
    
    

    % Pygments definitions
    
\makeatletter
\def\PY@reset{\let\PY@it=\relax \let\PY@bf=\relax%
    \let\PY@ul=\relax \let\PY@tc=\relax%
    \let\PY@bc=\relax \let\PY@ff=\relax}
\def\PY@tok#1{\csname PY@tok@#1\endcsname}
\def\PY@toks#1+{\ifx\relax#1\empty\else%
    \PY@tok{#1}\expandafter\PY@toks\fi}
\def\PY@do#1{\PY@bc{\PY@tc{\PY@ul{%
    \PY@it{\PY@bf{\PY@ff{#1}}}}}}}
\def\PY#1#2{\PY@reset\PY@toks#1+\relax+\PY@do{#2}}

\expandafter\def\csname PY@tok@gd\endcsname{\def\PY@tc##1{\textcolor[rgb]{0.63,0.00,0.00}{##1}}}
\expandafter\def\csname PY@tok@gu\endcsname{\let\PY@bf=\textbf\def\PY@tc##1{\textcolor[rgb]{0.50,0.00,0.50}{##1}}}
\expandafter\def\csname PY@tok@gt\endcsname{\def\PY@tc##1{\textcolor[rgb]{0.00,0.27,0.87}{##1}}}
\expandafter\def\csname PY@tok@gs\endcsname{\let\PY@bf=\textbf}
\expandafter\def\csname PY@tok@gr\endcsname{\def\PY@tc##1{\textcolor[rgb]{1.00,0.00,0.00}{##1}}}
\expandafter\def\csname PY@tok@cm\endcsname{\let\PY@it=\textit\def\PY@tc##1{\textcolor[rgb]{0.25,0.50,0.50}{##1}}}
\expandafter\def\csname PY@tok@vg\endcsname{\def\PY@tc##1{\textcolor[rgb]{0.10,0.09,0.49}{##1}}}
\expandafter\def\csname PY@tok@m\endcsname{\def\PY@tc##1{\textcolor[rgb]{0.40,0.40,0.40}{##1}}}
\expandafter\def\csname PY@tok@mh\endcsname{\def\PY@tc##1{\textcolor[rgb]{0.40,0.40,0.40}{##1}}}
\expandafter\def\csname PY@tok@go\endcsname{\def\PY@tc##1{\textcolor[rgb]{0.53,0.53,0.53}{##1}}}
\expandafter\def\csname PY@tok@ge\endcsname{\let\PY@it=\textit}
\expandafter\def\csname PY@tok@vc\endcsname{\def\PY@tc##1{\textcolor[rgb]{0.10,0.09,0.49}{##1}}}
\expandafter\def\csname PY@tok@il\endcsname{\def\PY@tc##1{\textcolor[rgb]{0.40,0.40,0.40}{##1}}}
\expandafter\def\csname PY@tok@cs\endcsname{\let\PY@it=\textit\def\PY@tc##1{\textcolor[rgb]{0.25,0.50,0.50}{##1}}}
\expandafter\def\csname PY@tok@cp\endcsname{\def\PY@tc##1{\textcolor[rgb]{0.74,0.48,0.00}{##1}}}
\expandafter\def\csname PY@tok@gi\endcsname{\def\PY@tc##1{\textcolor[rgb]{0.00,0.63,0.00}{##1}}}
\expandafter\def\csname PY@tok@gh\endcsname{\let\PY@bf=\textbf\def\PY@tc##1{\textcolor[rgb]{0.00,0.00,0.50}{##1}}}
\expandafter\def\csname PY@tok@ni\endcsname{\let\PY@bf=\textbf\def\PY@tc##1{\textcolor[rgb]{0.60,0.60,0.60}{##1}}}
\expandafter\def\csname PY@tok@nl\endcsname{\def\PY@tc##1{\textcolor[rgb]{0.63,0.63,0.00}{##1}}}
\expandafter\def\csname PY@tok@nn\endcsname{\let\PY@bf=\textbf\def\PY@tc##1{\textcolor[rgb]{0.00,0.00,1.00}{##1}}}
\expandafter\def\csname PY@tok@no\endcsname{\def\PY@tc##1{\textcolor[rgb]{0.53,0.00,0.00}{##1}}}
\expandafter\def\csname PY@tok@na\endcsname{\def\PY@tc##1{\textcolor[rgb]{0.49,0.56,0.16}{##1}}}
\expandafter\def\csname PY@tok@nb\endcsname{\def\PY@tc##1{\textcolor[rgb]{0.00,0.50,0.00}{##1}}}
\expandafter\def\csname PY@tok@nc\endcsname{\let\PY@bf=\textbf\def\PY@tc##1{\textcolor[rgb]{0.00,0.00,1.00}{##1}}}
\expandafter\def\csname PY@tok@nd\endcsname{\def\PY@tc##1{\textcolor[rgb]{0.67,0.13,1.00}{##1}}}
\expandafter\def\csname PY@tok@ne\endcsname{\let\PY@bf=\textbf\def\PY@tc##1{\textcolor[rgb]{0.82,0.25,0.23}{##1}}}
\expandafter\def\csname PY@tok@nf\endcsname{\def\PY@tc##1{\textcolor[rgb]{0.00,0.00,1.00}{##1}}}
\expandafter\def\csname PY@tok@si\endcsname{\let\PY@bf=\textbf\def\PY@tc##1{\textcolor[rgb]{0.73,0.40,0.53}{##1}}}
\expandafter\def\csname PY@tok@s2\endcsname{\def\PY@tc##1{\textcolor[rgb]{0.73,0.13,0.13}{##1}}}
\expandafter\def\csname PY@tok@vi\endcsname{\def\PY@tc##1{\textcolor[rgb]{0.10,0.09,0.49}{##1}}}
\expandafter\def\csname PY@tok@nt\endcsname{\let\PY@bf=\textbf\def\PY@tc##1{\textcolor[rgb]{0.00,0.50,0.00}{##1}}}
\expandafter\def\csname PY@tok@nv\endcsname{\def\PY@tc##1{\textcolor[rgb]{0.10,0.09,0.49}{##1}}}
\expandafter\def\csname PY@tok@s1\endcsname{\def\PY@tc##1{\textcolor[rgb]{0.73,0.13,0.13}{##1}}}
\expandafter\def\csname PY@tok@kd\endcsname{\let\PY@bf=\textbf\def\PY@tc##1{\textcolor[rgb]{0.00,0.50,0.00}{##1}}}
\expandafter\def\csname PY@tok@sh\endcsname{\def\PY@tc##1{\textcolor[rgb]{0.73,0.13,0.13}{##1}}}
\expandafter\def\csname PY@tok@sc\endcsname{\def\PY@tc##1{\textcolor[rgb]{0.73,0.13,0.13}{##1}}}
\expandafter\def\csname PY@tok@sx\endcsname{\def\PY@tc##1{\textcolor[rgb]{0.00,0.50,0.00}{##1}}}
\expandafter\def\csname PY@tok@bp\endcsname{\def\PY@tc##1{\textcolor[rgb]{0.00,0.50,0.00}{##1}}}
\expandafter\def\csname PY@tok@c1\endcsname{\let\PY@it=\textit\def\PY@tc##1{\textcolor[rgb]{0.25,0.50,0.50}{##1}}}
\expandafter\def\csname PY@tok@kc\endcsname{\let\PY@bf=\textbf\def\PY@tc##1{\textcolor[rgb]{0.00,0.50,0.00}{##1}}}
\expandafter\def\csname PY@tok@c\endcsname{\let\PY@it=\textit\def\PY@tc##1{\textcolor[rgb]{0.25,0.50,0.50}{##1}}}
\expandafter\def\csname PY@tok@mf\endcsname{\def\PY@tc##1{\textcolor[rgb]{0.40,0.40,0.40}{##1}}}
\expandafter\def\csname PY@tok@err\endcsname{\def\PY@bc##1{\setlength{\fboxsep}{0pt}\fcolorbox[rgb]{1.00,0.00,0.00}{1,1,1}{\strut ##1}}}
\expandafter\def\csname PY@tok@mb\endcsname{\def\PY@tc##1{\textcolor[rgb]{0.40,0.40,0.40}{##1}}}
\expandafter\def\csname PY@tok@ss\endcsname{\def\PY@tc##1{\textcolor[rgb]{0.10,0.09,0.49}{##1}}}
\expandafter\def\csname PY@tok@sr\endcsname{\def\PY@tc##1{\textcolor[rgb]{0.73,0.40,0.53}{##1}}}
\expandafter\def\csname PY@tok@mo\endcsname{\def\PY@tc##1{\textcolor[rgb]{0.40,0.40,0.40}{##1}}}
\expandafter\def\csname PY@tok@kn\endcsname{\let\PY@bf=\textbf\def\PY@tc##1{\textcolor[rgb]{0.00,0.50,0.00}{##1}}}
\expandafter\def\csname PY@tok@mi\endcsname{\def\PY@tc##1{\textcolor[rgb]{0.40,0.40,0.40}{##1}}}
\expandafter\def\csname PY@tok@gp\endcsname{\let\PY@bf=\textbf\def\PY@tc##1{\textcolor[rgb]{0.00,0.00,0.50}{##1}}}
\expandafter\def\csname PY@tok@o\endcsname{\def\PY@tc##1{\textcolor[rgb]{0.40,0.40,0.40}{##1}}}
\expandafter\def\csname PY@tok@kr\endcsname{\let\PY@bf=\textbf\def\PY@tc##1{\textcolor[rgb]{0.00,0.50,0.00}{##1}}}
\expandafter\def\csname PY@tok@s\endcsname{\def\PY@tc##1{\textcolor[rgb]{0.73,0.13,0.13}{##1}}}
\expandafter\def\csname PY@tok@kp\endcsname{\def\PY@tc##1{\textcolor[rgb]{0.00,0.50,0.00}{##1}}}
\expandafter\def\csname PY@tok@w\endcsname{\def\PY@tc##1{\textcolor[rgb]{0.73,0.73,0.73}{##1}}}
\expandafter\def\csname PY@tok@kt\endcsname{\def\PY@tc##1{\textcolor[rgb]{0.69,0.00,0.25}{##1}}}
\expandafter\def\csname PY@tok@ow\endcsname{\let\PY@bf=\textbf\def\PY@tc##1{\textcolor[rgb]{0.67,0.13,1.00}{##1}}}
\expandafter\def\csname PY@tok@sb\endcsname{\def\PY@tc##1{\textcolor[rgb]{0.73,0.13,0.13}{##1}}}
\expandafter\def\csname PY@tok@k\endcsname{\let\PY@bf=\textbf\def\PY@tc##1{\textcolor[rgb]{0.00,0.50,0.00}{##1}}}
\expandafter\def\csname PY@tok@se\endcsname{\let\PY@bf=\textbf\def\PY@tc##1{\textcolor[rgb]{0.73,0.40,0.13}{##1}}}
\expandafter\def\csname PY@tok@sd\endcsname{\let\PY@it=\textit\def\PY@tc##1{\textcolor[rgb]{0.73,0.13,0.13}{##1}}}

\def\PYZbs{\char`\\}
\def\PYZus{\char`\_}
\def\PYZob{\char`\{}
\def\PYZcb{\char`\}}
\def\PYZca{\char`\^}
\def\PYZam{\char`\&}
\def\PYZlt{\char`\<}
\def\PYZgt{\char`\>}
\def\PYZsh{\char`\#}
\def\PYZpc{\char`\%}
\def\PYZdl{\char`\$}
\def\PYZhy{\char`\-}
\def\PYZsq{\char`\'}
\def\PYZdq{\char`\"}
\def\PYZti{\char`\~}
% for compatibility with earlier versions
\def\PYZat{@}
\def\PYZlb{[}
\def\PYZrb{]}
\makeatother


    % Exact colors from NB
    \definecolor{incolor}{rgb}{0.0, 0.0, 0.5}
    \definecolor{outcolor}{rgb}{0.545, 0.0, 0.0}



    
    % Prevent overflowing lines due to hard-to-break entities
    \sloppy 
    % Setup hyperref package
    \hypersetup{
      breaklinks=true,  % so long urls are correctly broken across lines
      colorlinks=true,
      urlcolor=blue,
      linkcolor=darkorange,
      citecolor=darkgreen,
      }
    % Slightly bigger margins than the latex defaults
    
    \geometry{verbose,tmargin=1in,bmargin=1in,lmargin=1in,rmargin=1in}
    
    \author{Roberto Cadena Vega}

    \begin{document}
    
    
    \maketitle
    
    

    
    \begin{Verbatim}[commandchars=\\\{\}]
{\color{incolor}In [{\color{incolor}1}]:} \PY{c}{\PYZsh{} Se importan librerias para graficar, y se define un estilo especifico}
        \PY{o}{\PYZpc{}}\PY{k}{matplotlib} \PY{n}{inline}
        \PY{k+kn}{from} \PY{n+nn}{matplotlib.pyplot} \PY{k+kn}{import} \PY{n}{plot}\PY{p}{,} \PY{n}{figure}\PY{p}{,} \PY{n}{style}
        \PY{n}{style}\PY{o}{.}\PY{n}{use}\PY{p}{(}\PY{l+s}{\PYZdq{}}\PY{l+s}{ggplot}\PY{l+s}{\PYZdq{}}\PY{p}{)}
\end{Verbatim}


    \section{Tarea 2 - Simulación de sistema simple con retardos}


    \subsubsection{Problema}\label{problema}

Dado el sistema:

\[
\dot{x} = - 3 x(t) + x(t - 1) + x(t - 2)
\]

Simular el sistema con las siguientes condiciones iniciales \(\varphi\):

\begin{enumerate}
\def\labelenumi{\arabic{enumi}.}
\itemsep1pt\parskip0pt\parsep0pt
\item
  \(\varphi(t) = 1 \quad t \in [-2, 0]\)
\item
  \(\varphi(t) = \sin(t) \quad t \in [-2, 0]\)
\item
  \(\varphi(t) = \begin{cases} 1 & t = 0 \\ 0 & t \ne 0 \end{cases} \quad t \in [-2, 0]\)
\end{enumerate}

    \subsubsection{Solución}\label{soluciuxf3n}

Para simular y gráficar este sistema, tenemos que definir una función la
cual sea capaz de comportarse como el sistema descrito, esto lo logramos
con el código de MATLAB descrito en el archivo \texttt{f.m}, el cual
podemos ver a continuación:

    \begin{Verbatim}[commandchars=\\\{\}]
{\color{incolor}In [{\color{incolor}56}]:} \PY{n}{cat} \PY{n}{MATLAB}\PY{o}{/}\PY{n}{f}\PY{o}{.}\PY{n}{m}
\end{Verbatim}

    \begin{Verbatim}[commandchars=\\\{\}]
function dydt = f(t, y, Z)
    yret1 = Z(:, 1);
    yret2 = Z(:, 2);
    
    dydt = -3*y + yret1(1) + yret2(1);
end
    \end{Verbatim}

    Esta función se puede analizar por lineas, la primer linea

\begin{Shaded}
\begin{Highlighting}[]
\NormalTok{function dydt = f(t, y, Z)}
\end{Highlighting}
\end{Shaded}

tan solo describe la función \texttt{f}, con variables de entrada,
\texttt{t}, \texttt{y}, \texttt{Z} y una salida \texttt{dydt}. En esta
función \texttt{t} es el tiempo para el que queremos calcular el valor
de nuestra función; en especifico para nuestro ejemplo, como nuestro
sistema es invariante en el tiempo, no necesitamos incluirla dentro de
los calculos, \texttt{y} es el valor de nuestra variable, el cual será
aproximado por la función \texttt{dde23} y \texttt{Z} es el valor de la
variable del sistema, con un retardo aplicado.

Las siguientes lineas tan solo guardan los valores de la variable del
sistema en variables con un nombre significativo:

\begin{Shaded}
\begin{Highlighting}[]
    \NormalTok{yret1 = Z(:, }\FloatTok{1}\NormalTok{);}
    \NormalTok{yret2 = Z(:, }\FloatTok{2}\NormalTok{);}
\end{Highlighting}
\end{Shaded}

Y la ultima linea calcula el valor de la salida del sistema, para
nosostros \(\dot{x}\),

\begin{Shaded}
\begin{Highlighting}[]
    \NormalTok{dydt = -}\FloatTok{3}\NormalTok{*y + yret1(}\FloatTok{1}\NormalTok{) + yret2(}\FloatTok{1}\NormalTok{);}
\end{Highlighting}
\end{Shaded}

Sin embargo, en este punto, nuestro modelo computacional se acerca mas
bien a:

\[
\dot{x} = -3x(t) + x(t - \tau_1) + x(t - \tau_2)
\]

y los valores de los retrasos \(\tau_1\) y \(\tau_2\) los ingresaremos
en la función \texttt{dde23}.

Para poder simular este sistema, ahora tenemos que introducir la función
que define el comportamiento de nuestro sistema, en la función
\texttt{dde23}, lo cual se verá así:

\begin{Shaded}
\begin{Highlighting}[]
\NormalTok{sol0 = dde23(@f, [}\FloatTok{1}\NormalTok{, }\FloatTok{2}\NormalTok{], @phi0, [}\FloatTok{0}\NormalTok{, }\FloatTok{5}\NormalTok{])}
\end{Highlighting}
\end{Shaded}

En donde \texttt{@f}es la sintaxis para pasar como argumento a la
función \texttt{f}, \texttt{{[}1,\ 2{]}} es el conjunto de los retardos
a utilizar, \texttt{@phi0} es la función que define a las condiciones
iniciales y \texttt{{[}0,\ 5{]}} es el intervalo de simulación.

Una vez que hemos simulado, la variable \texttt{sol0} es una estructura
que contiene a las variables asociadas a la simulación, de donde podemos
obtener \texttt{x} y \texttt{y}, las cuales corresponden a \(t\) y
\(x\).

Ahora importamos los resultados de la simulación en MATLAB:

    \begin{Verbatim}[commandchars=\\\{\}]
{\color{incolor}In [{\color{incolor}3}]:} \PY{c}{\PYZsh{} Se importa libreria de entrada y salida y se importan datos de simulacion en MATLAB}
        \PY{k+kn}{import} \PY{n+nn}{scipy.io}
        \PY{n}{sistemaretardos} \PY{o}{=} \PY{n}{scipy}\PY{o}{.}\PY{n}{io}\PY{o}{.}\PY{n}{loadmat}\PY{p}{(}\PY{l+s}{\PYZsq{}}\PY{l+s}{./MATLAB/sistemaretardos.mat}\PY{l+s}{\PYZsq{}}\PY{p}{)}
\end{Verbatim}

    \begin{Verbatim}[commandchars=\\\{\}]
{\color{incolor}In [{\color{incolor}5}]:} \PY{c}{\PYZsh{} Se importan datos de simulación a variables de Python para poder manipularlas}
        \PY{n}{x0} \PY{o}{=} \PY{n}{sistemaretardos}\PY{o}{.}\PY{n}{get}\PY{p}{(}\PY{l+s}{\PYZdq{}}\PY{l+s}{x0}\PY{l+s}{\PYZdq{}}\PY{p}{)}
        \PY{n}{x1} \PY{o}{=} \PY{n}{sistemaretardos}\PY{o}{.}\PY{n}{get}\PY{p}{(}\PY{l+s}{\PYZdq{}}\PY{l+s}{x1}\PY{l+s}{\PYZdq{}}\PY{p}{)}
        \PY{n}{x2} \PY{o}{=} \PY{n}{sistemaretardos}\PY{o}{.}\PY{n}{get}\PY{p}{(}\PY{l+s}{\PYZdq{}}\PY{l+s}{x2}\PY{l+s}{\PYZdq{}}\PY{p}{)}
        \PY{n}{y0} \PY{o}{=} \PY{n}{sistemaretardos}\PY{o}{.}\PY{n}{get}\PY{p}{(}\PY{l+s}{\PYZdq{}}\PY{l+s}{y0}\PY{l+s}{\PYZdq{}}\PY{p}{)}
        \PY{n}{y1} \PY{o}{=} \PY{n}{sistemaretardos}\PY{o}{.}\PY{n}{get}\PY{p}{(}\PY{l+s}{\PYZdq{}}\PY{l+s}{y1}\PY{l+s}{\PYZdq{}}\PY{p}{)}
        \PY{n}{y2} \PY{o}{=} \PY{n}{sistemaretardos}\PY{o}{.}\PY{n}{get}\PY{p}{(}\PY{l+s}{\PYZdq{}}\PY{l+s}{y2}\PY{l+s}{\PYZdq{}}\PY{p}{)}
\end{Verbatim}

    Pero aun no hemos hablado de las funciones \(\phi\), las cuales deben
ser definidas de la misma manera que la función que nos da el
comportamiento del sistema, por lo que para la primer condición inicial:

\[
\varphi(t) = 1 \quad t \in [-2, 0]
\]

nos quedará:

    \begin{Verbatim}[commandchars=\\\{\}]
{\color{incolor}In [{\color{incolor}49}]:} \PY{n}{cat} \PY{n}{MATLAB}\PY{o}{/}\PY{n}{phi0}\PY{o}{.}\PY{n}{m}
\end{Verbatim}

    \begin{Verbatim}[commandchars=\\\{\}]
function y = phi0(t)
    y = 1;
end
    \end{Verbatim}

    y por lo tanto, para esta condición inicial, el comportamiento del
sistema nos queda:

    \begin{Verbatim}[commandchars=\\\{\}]
{\color{incolor}In [{\color{incolor}61}]:} \PY{n}{f} \PY{o}{=} \PY{n}{figure}\PY{p}{(}\PY{n}{figsize}\PY{o}{=}\PY{p}{(}\PY{l+m+mi}{18}\PY{p}{,} \PY{l+m+mi}{7}\PY{p}{)}\PY{p}{)}
         \PY{n}{plot}\PY{p}{(}\PY{n}{x0}\PY{p}{[}\PY{l+m+mi}{0}\PY{p}{]}\PY{p}{,} \PY{n}{y0}\PY{p}{[}\PY{l+m+mi}{0}\PY{p}{]}\PY{p}{)}
         
         \PY{n}{ax} \PY{o}{=} \PY{n}{f}\PY{o}{.}\PY{n}{gca}\PY{p}{(}\PY{p}{)}
         \PY{n}{ax}\PY{o}{.}\PY{n}{set\PYZus{}xlabel}\PY{p}{(}\PY{l+s}{r\PYZsq{}}\PY{l+s}{\PYZdl{}t\PYZdl{}}\PY{l+s}{\PYZsq{}}\PY{p}{,} \PY{n}{fontsize}\PY{o}{=}\PY{l+m+mi}{20}\PY{p}{)}
         \PY{n}{ax}\PY{o}{.}\PY{n}{set\PYZus{}ylabel}\PY{p}{(}\PY{l+s}{r\PYZsq{}}\PY{l+s}{\PYZdl{}x\PYZdl{}}\PY{l+s}{\PYZsq{}}\PY{p}{,} \PY{n}{fontsize}\PY{o}{=}\PY{l+m+mi}{20}\PY{p}{)}\PY{p}{;}
\end{Verbatim}

    \begin{center}
    \adjustimage{max size={0.9\linewidth}{0.9\paperheight}}{Tarea 2_files/Tarea 2_11_0.png}
    \end{center}
    { \hspace*{\fill} \\}
    
    De la misma manera, para la segunda condición inicial:

\[
\varphi(t) = \sin(t) \quad t \in [-2, 0]
\]

tenemos que la función \texttt{phi1} nos quedará:

    \begin{Verbatim}[commandchars=\\\{\}]
{\color{incolor}In [{\color{incolor}50}]:} \PY{n}{cat} \PY{n}{MATLAB}\PY{o}{/}\PY{n}{phi1}\PY{o}{.}\PY{n}{m}
\end{Verbatim}

    \begin{Verbatim}[commandchars=\\\{\}]
function y = phi1(t)
    y = sin(t);
end
    \end{Verbatim}

    \begin{Verbatim}[commandchars=\\\{\}]
{\color{incolor}In [{\color{incolor}62}]:} \PY{n}{f} \PY{o}{=} \PY{n}{figure}\PY{p}{(}\PY{n}{figsize}\PY{o}{=}\PY{p}{(}\PY{l+m+mi}{18}\PY{p}{,} \PY{l+m+mi}{7}\PY{p}{)}\PY{p}{)}
         \PY{n}{plot}\PY{p}{(}\PY{n}{x1}\PY{p}{[}\PY{l+m+mi}{0}\PY{p}{]}\PY{p}{,} \PY{n}{y1}\PY{p}{[}\PY{l+m+mi}{0}\PY{p}{]}\PY{p}{)}
         
         \PY{n}{ax} \PY{o}{=} \PY{n}{f}\PY{o}{.}\PY{n}{gca}\PY{p}{(}\PY{p}{)}
         \PY{n}{ax}\PY{o}{.}\PY{n}{set\PYZus{}xlabel}\PY{p}{(}\PY{l+s}{r\PYZsq{}}\PY{l+s}{\PYZdl{}t\PYZdl{}}\PY{l+s}{\PYZsq{}}\PY{p}{,} \PY{n}{fontsize}\PY{o}{=}\PY{l+m+mi}{20}\PY{p}{)}
         \PY{n}{ax}\PY{o}{.}\PY{n}{set\PYZus{}ylabel}\PY{p}{(}\PY{l+s}{r\PYZsq{}}\PY{l+s}{\PYZdl{}x\PYZdl{}}\PY{l+s}{\PYZsq{}}\PY{p}{,} \PY{n}{fontsize}\PY{o}{=}\PY{l+m+mi}{20}\PY{p}{)}\PY{p}{;}
\end{Verbatim}

    \begin{center}
    \adjustimage{max size={0.9\linewidth}{0.9\paperheight}}{Tarea 2_files/Tarea 2_14_0.png}
    \end{center}
    { \hspace*{\fill} \\}
    
    Y para la tercer condición inicial:

\[
\varphi(t) = \begin{cases} 1 & t = 0 \\ 0 & t \ne 0 \end{cases} \quad t \in [-2, 0]
\]

tenemos:

    \begin{Verbatim}[commandchars=\\\{\}]
{\color{incolor}In [{\color{incolor}55}]:} \PY{n}{cat} \PY{n}{MATLAB}\PY{o}{/}\PY{n}{phi2}\PY{o}{.}\PY{n}{m}
\end{Verbatim}

    \begin{Verbatim}[commandchars=\\\{\}]
function y = phi2(t)
    if t == 0
        y = 1;
    else
        y = 0;
    end
end
    \end{Verbatim}

    \begin{Verbatim}[commandchars=\\\{\}]
{\color{incolor}In [{\color{incolor}63}]:} \PY{n}{f} \PY{o}{=} \PY{n}{figure}\PY{p}{(}\PY{n}{figsize}\PY{o}{=}\PY{p}{(}\PY{l+m+mi}{18}\PY{p}{,} \PY{l+m+mi}{7}\PY{p}{)}\PY{p}{)}
         \PY{n}{plot}\PY{p}{(}\PY{n}{x2}\PY{p}{[}\PY{l+m+mi}{0}\PY{p}{]}\PY{p}{,} \PY{n}{y2}\PY{p}{[}\PY{l+m+mi}{0}\PY{p}{]}\PY{p}{)}
         
         \PY{n}{ax} \PY{o}{=} \PY{n}{f}\PY{o}{.}\PY{n}{gca}\PY{p}{(}\PY{p}{)}
         \PY{n}{ax}\PY{o}{.}\PY{n}{set\PYZus{}xlabel}\PY{p}{(}\PY{l+s}{r\PYZsq{}}\PY{l+s}{\PYZdl{}t\PYZdl{}}\PY{l+s}{\PYZsq{}}\PY{p}{,} \PY{n}{fontsize}\PY{o}{=}\PY{l+m+mi}{20}\PY{p}{)}
         \PY{n}{ax}\PY{o}{.}\PY{n}{set\PYZus{}ylabel}\PY{p}{(}\PY{l+s}{r\PYZsq{}}\PY{l+s}{\PYZdl{}x\PYZdl{}}\PY{l+s}{\PYZsq{}}\PY{p}{,} \PY{n}{fontsize}\PY{o}{=}\PY{l+m+mi}{20}\PY{p}{)}\PY{p}{;}
\end{Verbatim}

    \begin{center}
    \adjustimage{max size={0.9\linewidth}{0.9\paperheight}}{Tarea 2_files/Tarea 2_17_0.png}
    \end{center}
    { \hspace*{\fill} \\}
    
    Puedes acceder a este notebook a traves de la página

http://nbviewer.ipython.org/github/robblack007/DCA/blob/master/ \\
IPythonNotebooks/Sistemas\%20con\%20retardo\%20en\%20la\%20entrada/Tarea\%202.ipynb

o escaneando el siguiente código:

\begin{figure*}[htbp]
\centering
\includegraphics[width=0.2\textwidth]{codigos/prueba.jpg}
\end{figure*}

    \begin{Verbatim}[commandchars=\\\{\}]
{\color{incolor}In [{\color{incolor}12}]:} \PY{c}{\PYZsh{} Codigo para generar codigo :)}
         \PY{k+kn}{from} \PY{n+nn}{qrcode} \PY{k+kn}{import} \PY{n}{make}
         \PY{n}{img} \PY{o}{=} \PY{n}{make}\PY{p}{(}\PY{l+s}{\PYZdq{}}\PY{l+s}{http://nbviewer.ipython.org/github/robblack007/DCA/blob/master/IPythonNotebooks/Sistemas}\PY{l+s+si}{\PYZpc{}20c}\PY{l+s}{on}\PY{l+s+si}{\PYZpc{}20r}\PY{l+s}{etardo}\PY{l+s+si}{\PYZpc{}20e}\PY{l+s}{n}\PY{l+s}{\PYZpc{}}\PY{l+s}{20la}\PY{l+s+si}{\PYZpc{}20e}\PY{l+s}{ntrada/Tarea}\PY{l+s}{\PYZpc{}}\PY{l+s}{202.ipynb}\PY{l+s}{\PYZdq{}}\PY{p}{)}
         \PY{n}{img}\PY{o}{.}\PY{n}{save}\PY{p}{(}\PY{l+s}{\PYZdq{}}\PY{l+s}{codigos/prueba.jpg}\PY{l+s}{\PYZdq{}}\PY{p}{)}
\end{Verbatim}


    % Add a bibliography block to the postdoc
    
    
    
    \end{document}
