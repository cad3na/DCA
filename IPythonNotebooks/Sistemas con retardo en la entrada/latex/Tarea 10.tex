
% Default to the notebook output style

    


% Inherit from the specified cell style.




    
\documentclass{article}

    
    
    \usepackage{graphicx} % Used to insert images
    \usepackage{adjustbox} % Used to constrain images to a maximum size 
    \usepackage{color} % Allow colors to be defined
    \usepackage{enumerate} % Needed for markdown enumerations to work
    \usepackage{geometry} % Used to adjust the document margins
    \usepackage{amsmath} % Equations
    \usepackage{amssymb} % Equations
    \usepackage{eurosym} % defines \euro
    \usepackage[mathletters]{ucs} % Extended unicode (utf-8) support
    \usepackage[utf8x]{inputenc} % Allow utf-8 characters in the tex document
    \usepackage{fancyvrb} % verbatim replacement that allows latex
    \usepackage{grffile} % extends the file name processing of package graphics 
                         % to support a larger range 
    % The hyperref package gives us a pdf with properly built
    % internal navigation ('pdf bookmarks' for the table of contents,
    % internal cross-reference links, web links for URLs, etc.)
    \usepackage{hyperref}
    \usepackage{longtable} % longtable support required by pandoc >1.10
    \usepackage{booktabs}  % table support for pandoc > 1.12.2
    \usepackage{mathpazo}
    \usepackage[spanish]{babel}

    
    
    \definecolor{orange}{cmyk}{0,0.4,0.8,0.2}
    \definecolor{darkorange}{rgb}{.71,0.21,0.01}
    \definecolor{darkgreen}{rgb}{.12,.54,.11}
    \definecolor{myteal}{rgb}{.26, .44, .56}
    \definecolor{gray}{gray}{0.45}
    \definecolor{lightgray}{gray}{.95}
    \definecolor{mediumgray}{gray}{.8}
    \definecolor{inputbackground}{rgb}{.95, .95, .85}
    \definecolor{outputbackground}{rgb}{.95, .95, .95}
    \definecolor{traceback}{rgb}{1, .95, .95}
    % ansi colors
    \definecolor{red}{rgb}{.6,0,0}
    \definecolor{green}{rgb}{0,.65,0}
    \definecolor{brown}{rgb}{0.6,0.6,0}
    \definecolor{blue}{rgb}{0,.145,.698}
    \definecolor{purple}{rgb}{.698,.145,.698}
    \definecolor{cyan}{rgb}{0,.698,.698}
    \definecolor{lightgray}{gray}{0.5}
    
    % bright ansi colors
    \definecolor{darkgray}{gray}{0.25}
    \definecolor{lightred}{rgb}{1.0,0.39,0.28}
    \definecolor{lightgreen}{rgb}{0.48,0.99,0.0}
    \definecolor{lightblue}{rgb}{0.53,0.81,0.92}
    \definecolor{lightpurple}{rgb}{0.87,0.63,0.87}
    \definecolor{lightcyan}{rgb}{0.5,1.0,0.83}
    
    % commands and environments needed by pandoc snippets
    % extracted from the output of `pandoc -s`
    \DefineVerbatimEnvironment{Highlighting}{Verbatim}{commandchars=\\\{\}}
    % Add ',fontsize=\small' for more characters per line
    \newenvironment{Shaded}{}{}
    \newcommand{\KeywordTok}[1]{\textcolor[rgb]{0.00,0.44,0.13}{\textbf{{#1}}}}
    \newcommand{\DataTypeTok}[1]{\textcolor[rgb]{0.56,0.13,0.00}{{#1}}}
    \newcommand{\DecValTok}[1]{\textcolor[rgb]{0.25,0.63,0.44}{{#1}}}
    \newcommand{\BaseNTok}[1]{\textcolor[rgb]{0.25,0.63,0.44}{{#1}}}
    \newcommand{\FloatTok}[1]{\textcolor[rgb]{0.25,0.63,0.44}{{#1}}}
    \newcommand{\CharTok}[1]{\textcolor[rgb]{0.25,0.44,0.63}{{#1}}}
    \newcommand{\StringTok}[1]{\textcolor[rgb]{0.25,0.44,0.63}{{#1}}}
    \newcommand{\CommentTok}[1]{\textcolor[rgb]{0.38,0.63,0.69}{\textit{{#1}}}}
    \newcommand{\OtherTok}[1]{\textcolor[rgb]{0.00,0.44,0.13}{{#1}}}
    \newcommand{\AlertTok}[1]{\textcolor[rgb]{1.00,0.00,0.00}{\textbf{{#1}}}}
    \newcommand{\FunctionTok}[1]{\textcolor[rgb]{0.02,0.16,0.49}{{#1}}}
    \newcommand{\RegionMarkerTok}[1]{{#1}}
    \newcommand{\ErrorTok}[1]{\textcolor[rgb]{1.00,0.00,0.00}{\textbf{{#1}}}}
    \newcommand{\NormalTok}[1]{{#1}}
    
    % Define a nice break command that doesn't care if a line doesn't already
    % exist.
    \def\br{\hspace*{\fill} \\* }
    % Math Jax compatability definitions
    \def\gt{>}
    \def\lt{<}
    % Document parameters
    \title{Tarea 10 - Atrasada}
    
    
    

    % Pygments definitions
    
\makeatletter
\def\PY@reset{\let\PY@it=\relax \let\PY@bf=\relax%
    \let\PY@ul=\relax \let\PY@tc=\relax%
    \let\PY@bc=\relax \let\PY@ff=\relax}
\def\PY@tok#1{\csname PY@tok@#1\endcsname}
\def\PY@toks#1+{\ifx\relax#1\empty\else%
    \PY@tok{#1}\expandafter\PY@toks\fi}
\def\PY@do#1{\PY@bc{\PY@tc{\PY@ul{%
    \PY@it{\PY@bf{\PY@ff{#1}}}}}}}
\def\PY#1#2{\PY@reset\PY@toks#1+\relax+\PY@do{#2}}

\expandafter\def\csname PY@tok@cm\endcsname{\let\PY@it=\textit\def\PY@tc##1{\textcolor[rgb]{0.25,0.50,0.50}{##1}}}
\expandafter\def\csname PY@tok@vg\endcsname{\def\PY@tc##1{\textcolor[rgb]{0.10,0.09,0.49}{##1}}}
\expandafter\def\csname PY@tok@kd\endcsname{\let\PY@bf=\textbf\def\PY@tc##1{\textcolor[rgb]{0.00,0.50,0.00}{##1}}}
\expandafter\def\csname PY@tok@ow\endcsname{\let\PY@bf=\textbf\def\PY@tc##1{\textcolor[rgb]{0.67,0.13,1.00}{##1}}}
\expandafter\def\csname PY@tok@s1\endcsname{\def\PY@tc##1{\textcolor[rgb]{0.73,0.13,0.13}{##1}}}
\expandafter\def\csname PY@tok@m\endcsname{\def\PY@tc##1{\textcolor[rgb]{0.40,0.40,0.40}{##1}}}
\expandafter\def\csname PY@tok@k\endcsname{\let\PY@bf=\textbf\def\PY@tc##1{\textcolor[rgb]{0.00,0.50,0.00}{##1}}}
\expandafter\def\csname PY@tok@gi\endcsname{\def\PY@tc##1{\textcolor[rgb]{0.00,0.63,0.00}{##1}}}
\expandafter\def\csname PY@tok@kn\endcsname{\let\PY@bf=\textbf\def\PY@tc##1{\textcolor[rgb]{0.00,0.50,0.00}{##1}}}
\expandafter\def\csname PY@tok@sd\endcsname{\let\PY@it=\textit\def\PY@tc##1{\textcolor[rgb]{0.73,0.13,0.13}{##1}}}
\expandafter\def\csname PY@tok@gr\endcsname{\def\PY@tc##1{\textcolor[rgb]{1.00,0.00,0.00}{##1}}}
\expandafter\def\csname PY@tok@sx\endcsname{\def\PY@tc##1{\textcolor[rgb]{0.00,0.50,0.00}{##1}}}
\expandafter\def\csname PY@tok@gh\endcsname{\let\PY@bf=\textbf\def\PY@tc##1{\textcolor[rgb]{0.00,0.00,0.50}{##1}}}
\expandafter\def\csname PY@tok@no\endcsname{\def\PY@tc##1{\textcolor[rgb]{0.53,0.00,0.00}{##1}}}
\expandafter\def\csname PY@tok@ge\endcsname{\let\PY@it=\textit}
\expandafter\def\csname PY@tok@cs\endcsname{\let\PY@it=\textit\def\PY@tc##1{\textcolor[rgb]{0.25,0.50,0.50}{##1}}}
\expandafter\def\csname PY@tok@nf\endcsname{\def\PY@tc##1{\textcolor[rgb]{0.00,0.00,1.00}{##1}}}
\expandafter\def\csname PY@tok@o\endcsname{\def\PY@tc##1{\textcolor[rgb]{0.40,0.40,0.40}{##1}}}
\expandafter\def\csname PY@tok@nl\endcsname{\def\PY@tc##1{\textcolor[rgb]{0.63,0.63,0.00}{##1}}}
\expandafter\def\csname PY@tok@gp\endcsname{\let\PY@bf=\textbf\def\PY@tc##1{\textcolor[rgb]{0.00,0.00,0.50}{##1}}}
\expandafter\def\csname PY@tok@sb\endcsname{\def\PY@tc##1{\textcolor[rgb]{0.73,0.13,0.13}{##1}}}
\expandafter\def\csname PY@tok@c\endcsname{\let\PY@it=\textit\def\PY@tc##1{\textcolor[rgb]{0.25,0.50,0.50}{##1}}}
\expandafter\def\csname PY@tok@ni\endcsname{\let\PY@bf=\textbf\def\PY@tc##1{\textcolor[rgb]{0.60,0.60,0.60}{##1}}}
\expandafter\def\csname PY@tok@sr\endcsname{\def\PY@tc##1{\textcolor[rgb]{0.73,0.40,0.53}{##1}}}
\expandafter\def\csname PY@tok@kt\endcsname{\def\PY@tc##1{\textcolor[rgb]{0.69,0.00,0.25}{##1}}}
\expandafter\def\csname PY@tok@gt\endcsname{\def\PY@tc##1{\textcolor[rgb]{0.00,0.27,0.87}{##1}}}
\expandafter\def\csname PY@tok@il\endcsname{\def\PY@tc##1{\textcolor[rgb]{0.40,0.40,0.40}{##1}}}
\expandafter\def\csname PY@tok@sh\endcsname{\def\PY@tc##1{\textcolor[rgb]{0.73,0.13,0.13}{##1}}}
\expandafter\def\csname PY@tok@bp\endcsname{\def\PY@tc##1{\textcolor[rgb]{0.00,0.50,0.00}{##1}}}
\expandafter\def\csname PY@tok@c1\endcsname{\let\PY@it=\textit\def\PY@tc##1{\textcolor[rgb]{0.25,0.50,0.50}{##1}}}
\expandafter\def\csname PY@tok@gu\endcsname{\let\PY@bf=\textbf\def\PY@tc##1{\textcolor[rgb]{0.50,0.00,0.50}{##1}}}
\expandafter\def\csname PY@tok@nd\endcsname{\def\PY@tc##1{\textcolor[rgb]{0.67,0.13,1.00}{##1}}}
\expandafter\def\csname PY@tok@s2\endcsname{\def\PY@tc##1{\textcolor[rgb]{0.73,0.13,0.13}{##1}}}
\expandafter\def\csname PY@tok@gd\endcsname{\def\PY@tc##1{\textcolor[rgb]{0.63,0.00,0.00}{##1}}}
\expandafter\def\csname PY@tok@kp\endcsname{\def\PY@tc##1{\textcolor[rgb]{0.00,0.50,0.00}{##1}}}
\expandafter\def\csname PY@tok@na\endcsname{\def\PY@tc##1{\textcolor[rgb]{0.49,0.56,0.16}{##1}}}
\expandafter\def\csname PY@tok@si\endcsname{\let\PY@bf=\textbf\def\PY@tc##1{\textcolor[rgb]{0.73,0.40,0.53}{##1}}}
\expandafter\def\csname PY@tok@ss\endcsname{\def\PY@tc##1{\textcolor[rgb]{0.10,0.09,0.49}{##1}}}
\expandafter\def\csname PY@tok@sc\endcsname{\def\PY@tc##1{\textcolor[rgb]{0.73,0.13,0.13}{##1}}}
\expandafter\def\csname PY@tok@mb\endcsname{\def\PY@tc##1{\textcolor[rgb]{0.40,0.40,0.40}{##1}}}
\expandafter\def\csname PY@tok@gs\endcsname{\let\PY@bf=\textbf}
\expandafter\def\csname PY@tok@nn\endcsname{\let\PY@bf=\textbf\def\PY@tc##1{\textcolor[rgb]{0.00,0.00,1.00}{##1}}}
\expandafter\def\csname PY@tok@mh\endcsname{\def\PY@tc##1{\textcolor[rgb]{0.40,0.40,0.40}{##1}}}
\expandafter\def\csname PY@tok@nt\endcsname{\let\PY@bf=\textbf\def\PY@tc##1{\textcolor[rgb]{0.00,0.50,0.00}{##1}}}
\expandafter\def\csname PY@tok@vc\endcsname{\def\PY@tc##1{\textcolor[rgb]{0.10,0.09,0.49}{##1}}}
\expandafter\def\csname PY@tok@go\endcsname{\def\PY@tc##1{\textcolor[rgb]{0.53,0.53,0.53}{##1}}}
\expandafter\def\csname PY@tok@ne\endcsname{\let\PY@bf=\textbf\def\PY@tc##1{\textcolor[rgb]{0.82,0.25,0.23}{##1}}}
\expandafter\def\csname PY@tok@mi\endcsname{\def\PY@tc##1{\textcolor[rgb]{0.40,0.40,0.40}{##1}}}
\expandafter\def\csname PY@tok@nv\endcsname{\def\PY@tc##1{\textcolor[rgb]{0.10,0.09,0.49}{##1}}}
\expandafter\def\csname PY@tok@nb\endcsname{\def\PY@tc##1{\textcolor[rgb]{0.00,0.50,0.00}{##1}}}
\expandafter\def\csname PY@tok@kr\endcsname{\let\PY@bf=\textbf\def\PY@tc##1{\textcolor[rgb]{0.00,0.50,0.00}{##1}}}
\expandafter\def\csname PY@tok@s\endcsname{\def\PY@tc##1{\textcolor[rgb]{0.73,0.13,0.13}{##1}}}
\expandafter\def\csname PY@tok@cp\endcsname{\def\PY@tc##1{\textcolor[rgb]{0.74,0.48,0.00}{##1}}}
\expandafter\def\csname PY@tok@mf\endcsname{\def\PY@tc##1{\textcolor[rgb]{0.40,0.40,0.40}{##1}}}
\expandafter\def\csname PY@tok@w\endcsname{\def\PY@tc##1{\textcolor[rgb]{0.73,0.73,0.73}{##1}}}
\expandafter\def\csname PY@tok@nc\endcsname{\let\PY@bf=\textbf\def\PY@tc##1{\textcolor[rgb]{0.00,0.00,1.00}{##1}}}
\expandafter\def\csname PY@tok@kc\endcsname{\let\PY@bf=\textbf\def\PY@tc##1{\textcolor[rgb]{0.00,0.50,0.00}{##1}}}
\expandafter\def\csname PY@tok@err\endcsname{\def\PY@bc##1{\setlength{\fboxsep}{0pt}\fcolorbox[rgb]{1.00,0.00,0.00}{1,1,1}{\strut ##1}}}
\expandafter\def\csname PY@tok@vi\endcsname{\def\PY@tc##1{\textcolor[rgb]{0.10,0.09,0.49}{##1}}}
\expandafter\def\csname PY@tok@se\endcsname{\let\PY@bf=\textbf\def\PY@tc##1{\textcolor[rgb]{0.73,0.40,0.13}{##1}}}
\expandafter\def\csname PY@tok@mo\endcsname{\def\PY@tc##1{\textcolor[rgb]{0.40,0.40,0.40}{##1}}}

\def\PYZbs{\char`\\}
\def\PYZus{\char`\_}
\def\PYZob{\char`\{}
\def\PYZcb{\char`\}}
\def\PYZca{\char`\^}
\def\PYZam{\char`\&}
\def\PYZlt{\char`\<}
\def\PYZgt{\char`\>}
\def\PYZsh{\char`\#}
\def\PYZpc{\char`\%}
\def\PYZdl{\char`\$}
\def\PYZhy{\char`\-}
\def\PYZsq{\char`\'}
\def\PYZdq{\char`\"}
\def\PYZti{\char`\~}
% for compatibility with earlier versions
\def\PYZat{@}
\def\PYZlb{[}
\def\PYZrb{]}
\makeatother


    % Exact colors from NB
    \definecolor{incolor}{rgb}{0.0, 0.0, 0.5}
    \definecolor{outcolor}{rgb}{0.545, 0.0, 0.0}



    
    % Prevent overflowing lines due to hard-to-break entities
    \sloppy 
    % Setup hyperref package
    \hypersetup{
      breaklinks=true,  % so long urls are correctly broken across lines
      colorlinks=true,
      urlcolor=blue,
      linkcolor=darkorange,
      citecolor=darkgreen,
      }
    % Slightly bigger margins than the latex defaults
    
    \geometry{verbose,tmargin=1in,bmargin=1in,lmargin=1in,rmargin=1in}
    
    \author{Roberto Cadena Vega}

    \begin{document}
    
    
    \maketitle

    
    
    \section*{Análisis de estabilidad para sistema bajo realimentación
integral}\label{anuxe1lisis-de-estabilidad-para-sistema-bajo-realimentaciuxf3n-integral}

    \subsection*{Ejemplo 1}\label{ejemplo-1}

    Dado el sistema:

\[
\dot{x}(t) = A x(t) + B u(t - h)
\]

con la ley de control:

\[
u(t) = k \left[ x(t) + \int_{-h}^0 e^{-A(\theta + h)} B u(t + \theta) d\theta \right]
\]

por lo que el sistema en lazo cerrado es:

\[
\dot{x}(t) = \left( A + e^{-Ah} B K \right) x(t)
\]

el sistema para el que haremos este desarrollo es:

\[
\dot{x}(t) =
\begin{pmatrix}
0 & 0 \\
1 & 1
\end{pmatrix} x(t) +
\begin{pmatrix}
1 \\
0
\end{pmatrix} u(t - h)
\]

con \(h = 1\).

Calculando \(e^{-A (\theta + h)}\) tenemos:

    \begin{Verbatim}[commandchars=\\\{\}]
{\color{incolor}In [{\color{incolor}2}]:} \PY{k+kn}{from} \PY{n+nn}{IPython}\PY{n+nn}{.}\PY{n+nn}{display} \PY{k}{import} \PY{n}{display}
        
        \PY{k+kn}{from} \PY{n+nn}{sympy} \PY{k}{import} \PY{n}{var}\PY{p}{,} \PY{n}{sin}\PY{p}{,} \PY{n}{cos}\PY{p}{,} \PY{n}{Matrix}\PY{p}{,} \PY{n}{Integer}\PY{p}{,} \PY{n}{eye}\PY{p}{,} \PY{n}{Function}\PY{p}{,} \PY{n}{Rational}\PY{p}{,} \PY{n}{exp}\PY{p}{,} \PY{n}{Symbol}\PY{p}{,} \PY{n}{I}
        \PY{k+kn}{from} \PY{n+nn}{sympy}\PY{n+nn}{.}\PY{n+nn}{physics}\PY{n+nn}{.}\PY{n+nn}{mechanics} \PY{k}{import} \PY{n}{mlatex}\PY{p}{,} \PY{n}{mechanics\PYZus{}printing}
        \PY{k+kn}{from} \PY{n+nn}{sympy}\PY{n+nn}{.}\PY{n+nn}{integrals} \PY{k}{import} \PY{n}{laplace\PYZus{}transform}
        \PY{n}{mechanics\PYZus{}printing}\PY{p}{(}\PY{p}{)}
\end{Verbatim}

    \begin{Verbatim}[commandchars=\\\{\}]
{\color{incolor}In [{\color{incolor}3}]:} \PY{n}{var}\PY{p}{(}\PY{l+s}{\PYZdq{}}\PY{l+s}{t h θ s ω}\PY{l+s}{\PYZdq{}}\PY{p}{)}
\end{Verbatim}
\texttt{\color{outcolor}Out[{\color{outcolor}3}]:}
    
    
        \begin{equation*}\adjustbox{max width=\hsize}{$
        \left ( t, \quad h, \quad θ, \quad s, \quad ω\right )
        $}\end{equation*}

    

    \begin{Verbatim}[commandchars=\\\{\}]
{\color{incolor}In [{\color{incolor}4}]:} \PY{n}{A} \PY{o}{=} \PY{n}{Matrix}\PY{p}{(}\PY{p}{[}\PY{p}{[}\PY{l+m+mi}{0}\PY{p}{,} \PY{l+m+mi}{0}\PY{p}{]}\PY{p}{,} \PY{p}{[}\PY{l+m+mi}{1}\PY{p}{,} \PY{l+m+mi}{1}\PY{p}{]}\PY{p}{]}\PY{p}{)}
        \PY{n}{A}
\end{Verbatim}
\texttt{\color{outcolor}Out[{\color{outcolor}4}]:}
    
    
        \begin{equation*}\adjustbox{max width=\hsize}{$
        \left[\begin{matrix}0 & 0\\1 & 1\end{matrix}\right]
        $}\end{equation*}

    

    \begin{Verbatim}[commandchars=\\\{\}]
{\color{incolor}In [{\color{incolor}5}]:} \PY{n}{exp}\PY{p}{(}\PY{o}{\PYZhy{}}\PY{n}{A}\PY{o}{*}\PY{p}{(}\PY{n}{θ}\PY{o}{+} \PY{n}{h}\PY{p}{)}\PY{p}{)}
\end{Verbatim}
\texttt{\color{outcolor}Out[{\color{outcolor}5}]:}
    
    
        \begin{equation*}\adjustbox{max width=\hsize}{$
        \left[\begin{matrix}1 & 0\\e^{- h - θ} - 1 & e^{- h - θ}\end{matrix}\right]
        $}\end{equation*}

    

    Sustituyendo \(A\), \(B\) y \(e^{-A(t + \theta)}\) en \(u(t)\), tenemos:

\[
\begin{align}
u(t) &=
\begin{pmatrix}
k_1 & k_2
\end{pmatrix} x(t) +
\begin{pmatrix}
k_1 & k_2
\end{pmatrix}
\int_{-h}^0 e^{-A(\theta + h)} B u(t + \theta) d\theta \\
&=
\begin{pmatrix}
k_1 & k_2
\end{pmatrix} x(t) +
\begin{pmatrix}
k_1 & k_2
\end{pmatrix}
\int_{-h}^0
\begin{pmatrix}
1 & 0 \\
e^{-(\theta + h)} - 1 & e^{-(\theta + h)}
\end{pmatrix}
\begin{pmatrix}
1 \\
0
\end{pmatrix}
u(t + \theta) d\theta \\
&=
\begin{pmatrix}
k_1 & k_2
\end{pmatrix} x(t) +
\begin{pmatrix}
k_1 & k_2
\end{pmatrix}
\int_{-h}^0
\begin{pmatrix}
1 \\
e^{-(\theta + h)} - 1
\end{pmatrix}
u(t + \theta) d\theta
\end{align}
\]

Sustituyendo \(k_1 = 1 - 4 e^h\) y \(k_2 = -4e^h\) tenemos:

\[
u(t)=
\begin{pmatrix}
1 - 4 e^h & -4e^h
\end{pmatrix} x(t) +
\begin{pmatrix}
1 - 4 e^h & -4e^h
\end{pmatrix}
\int_{-h}^0
\begin{pmatrix}
1 \\
e^{-(\theta + h)} - 1
\end{pmatrix}
u(t + \theta) d\theta
\]

y podemos meter estas ganancias a la integral, para obtener:

\[
\begin{align}
u(t) &=
\begin{pmatrix}
1 - 4 e^h & -4e^h
\end{pmatrix} x(t) +
\int_{-h}^0
\left( 1 - 4e^{-\theta} \right)
u(t + \theta) d\theta \\
&=
\begin{pmatrix}
1 - 4 e^h & -4e^h
\end{pmatrix} x(t) +
\int_{-h}^0 u(t + \theta) d\theta -
\int_{-h}^0 4e^{-\theta} u(t + \theta) d\theta
\end{align}
\]

    \begin{Verbatim}[commandchars=\\\{\}]
{\color{incolor}In [{\color{incolor}6}]:} \PY{n}{B} \PY{o}{=} \PY{n}{Matrix}\PY{p}{(}\PY{p}{[}\PY{p}{[}\PY{l+m+mi}{1}\PY{p}{]}\PY{p}{,} \PY{p}{[}\PY{l+m+mi}{0}\PY{p}{]}\PY{p}{]}\PY{p}{)}
        \PY{n}{K} \PY{o}{=} \PY{n}{Matrix}\PY{p}{(}\PY{p}{[}\PY{p}{[}\PY{l+m+mi}{1} \PY{o}{\PYZhy{}} \PY{l+m+mi}{4}\PY{o}{*}\PY{n}{exp}\PY{p}{(}\PY{n}{h}\PY{p}{)}\PY{p}{,} \PY{o}{\PYZhy{}}\PY{l+m+mi}{4}\PY{o}{*}\PY{n}{exp}\PY{p}{(}\PY{n}{h}\PY{p}{)}\PY{p}{]}\PY{p}{]}\PY{p}{)}
\end{Verbatim}

    \begin{Verbatim}[commandchars=\\\{\}]
{\color{incolor}In [{\color{incolor}7}]:} \PY{p}{(}\PY{n}{K}\PY{o}{*}\PY{n}{exp}\PY{p}{(}\PY{o}{\PYZhy{}}\PY{n}{A}\PY{o}{*}\PY{p}{(}\PY{n}{θ} \PY{o}{+} \PY{n}{h}\PY{p}{)}\PY{p}{)}\PY{o}{*}\PY{n}{B}\PY{p}{)}\PY{p}{[}\PY{l+m+mi}{0}\PY{p}{]}\PY{o}{.}\PY{n}{simplify}\PY{p}{(}\PY{p}{)}
\end{Verbatim}
\texttt{\color{outcolor}Out[{\color{outcolor}7}]:}
    
    
        \begin{equation*}\adjustbox{max width=\hsize}{$
        1 - 4 e^{- θ}
        $}\end{equation*}

    

    \begin{Verbatim}[commandchars=\\\{\}]
{\color{incolor}In [{\color{incolor}8}]:} \PY{n}{x1} \PY{o}{=} \PY{n}{Function}\PY{p}{(}\PY{l+s}{\PYZdq{}}\PY{l+s}{x1}\PY{l+s}{\PYZdq{}}\PY{p}{)}\PY{p}{(}\PY{n}{t}\PY{p}{)}
        \PY{n}{x2} \PY{o}{=} \PY{n}{Function}\PY{p}{(}\PY{l+s}{\PYZdq{}}\PY{l+s}{x2}\PY{l+s}{\PYZdq{}}\PY{p}{)}\PY{p}{(}\PY{n}{t}\PY{p}{)}
        
        \PY{n}{X} \PY{o}{=} \PY{n}{Matrix}\PY{p}{(}\PY{p}{[}\PY{p}{[}\PY{n}{x1}\PY{p}{]}\PY{p}{,} \PY{p}{[}\PY{n}{x2}\PY{p}{]}\PY{p}{]}\PY{p}{)}
        
        \PY{n}{u} \PY{o}{=} \PY{n}{Function}\PY{p}{(}\PY{l+s}{\PYZdq{}}\PY{l+s}{u}\PY{l+s}{\PYZdq{}}\PY{p}{)}\PY{p}{(}\PY{n}{t} \PY{o}{+} \PY{n}{θ}\PY{p}{)}
\end{Verbatim}

    \begin{Verbatim}[commandchars=\\\{\}]
{\color{incolor}In [{\color{incolor}9}]:} \PY{n}{A}\PY{o}{*}\PY{n}{X}
\end{Verbatim}
\texttt{\color{outcolor}Out[{\color{outcolor}9}]:}
    
    
        \begin{equation*}\adjustbox{max width=\hsize}{$
        \left[\begin{matrix}0\\x_{1} + x_{2}\end{matrix}\right]
        $}\end{equation*}

    

    \begin{Verbatim}[commandchars=\\\{\}]
{\color{incolor}In [{\color{incolor}10}]:} \PY{p}{(}\PY{p}{(}\PY{n}{K}\PY{o}{*}\PY{n}{exp}\PY{p}{(}\PY{o}{\PYZhy{}}\PY{n}{A}\PY{o}{*}\PY{p}{(}\PY{n}{θ} \PY{o}{+} \PY{n}{h}\PY{p}{)}\PY{p}{)}\PY{o}{*}\PY{n}{B}\PY{p}{)}\PY{p}{[}\PY{l+m+mi}{0}\PY{p}{]}\PY{o}{.}\PY{n}{simplify}\PY{p}{(}\PY{p}{)}\PY{o}{*}\PY{n}{u}\PY{p}{)}\PY{o}{.}\PY{n}{integrate}\PY{p}{(}\PY{p}{(}\PY{n}{θ}\PY{p}{,} \PY{o}{\PYZhy{}}\PY{n}{h}\PY{p}{,} \PY{l+m+mi}{0}\PY{p}{)}\PY{p}{)}
\end{Verbatim}
\texttt{\color{outcolor}Out[{\color{outcolor}10}]:}
    
    
        \begin{equation*}\adjustbox{max width=\hsize}{$
        \int_{- h}^{0} \left(e^{θ} - 4\right) \operatorname{u}\left(t + θ\right) e^{- θ}\, dθ
        $}\end{equation*}

    

    \begin{Verbatim}[commandchars=\\\{\}]
{\color{incolor}In [{\color{incolor}11}]:} \PY{p}{(}\PY{n}{K}\PY{o}{*}\PY{n}{X}\PY{p}{)}\PY{p}{[}\PY{l+m+mi}{0}\PY{p}{]} \PY{o}{+} \PY{p}{(}\PY{p}{(}\PY{n}{K}\PY{o}{*}\PY{n}{exp}\PY{p}{(}\PY{o}{\PYZhy{}}\PY{n}{A}\PY{o}{*}\PY{p}{(}\PY{n}{θ} \PY{o}{+} \PY{n}{h}\PY{p}{)}\PY{p}{)}\PY{o}{*}\PY{n}{B}\PY{p}{)}\PY{p}{[}\PY{l+m+mi}{0}\PY{p}{]}\PY{o}{.}\PY{n}{simplify}\PY{p}{(}\PY{p}{)}\PY{o}{*}\PY{n}{u}\PY{p}{)}\PY{o}{.}\PY{n}{integrate}\PY{p}{(}\PY{p}{(}\PY{n}{θ}\PY{p}{,} \PY{o}{\PYZhy{}}\PY{n}{h}\PY{p}{,} \PY{l+m+mi}{0}\PY{p}{)}\PY{p}{)}
\end{Verbatim}
\texttt{\color{outcolor}Out[{\color{outcolor}11}]:}
    
    
        \begin{equation*}\adjustbox{max width=\hsize}{$
        \left(- 4 e^{h} + 1\right) x_{1} - 4 x_{2} e^{h} + \int_{- h}^{0} \left(e^{θ} - 4\right) \operatorname{u}\left(t + θ\right) e^{- θ}\, dθ
        $}\end{equation*}

    

    Si aplicamos la transformada de Laplace a esto, obtendremos:

\[
\begin{align}
u(t) - \int_{-h}^0 u(t + \theta) d\theta + \int_{-h}^0 4e^{-\theta} u(t + \theta) d\theta &=
(1 - 4 e^h) x_1(t) - 4e^h x_2(t) \\
\left[ 1 - \frac{1 - e^{-hs}}{s} + 4 \frac{1 - e^{-h(s-1)}}{s-1} \right] u(s) &=
(1 - 4 e^h) x_1(s) - 4e^h x_2(s)
\end{align}
\]

Por lo que el polinomio caracteristico del controlador del sistema es:

\[
1 - \frac{1 - e^{-hs}}{s} + 4\frac{1 - e^{-h(s-1)}}{s-1} = 0
\]

Si ahora introducimos los parametros \(\alpha_1 = k_1 - k_2 = 1\) y
\(\alpha_2 = e^{-h} k_2 = -4\), este polinomio caracteristico queda de
la forma:

\[
1 - \alpha_1 \frac{1 - e^{-hs}}{s} - \alpha_2 \frac{1 - e^{-h(s-1)}}{s-1} = 0
\]

Este polinomio caracteristico corresponde al controlador del sistema,
por otro lado, tambien tenemos que analizar la estabilidad del sistema
bajo esta realimentación, la ecuación para esto es:

\[
\dot{x}(t) = \left( A + e^{-Ah} B K \right) x(t)
\]

por lo que la función de transferencia del sistema realimentado será:

\[
\det{\left( sI - A - e^{-Ah} B K \right)}
\]

    \begin{Verbatim}[commandchars=\\\{\}]
{\color{incolor}In [{\color{incolor}12}]:} \PY{p}{(}\PY{n}{s}\PY{o}{*}\PY{n}{eye}\PY{p}{(}\PY{l+m+mi}{2}\PY{p}{)} \PY{o}{\PYZhy{}} \PY{n}{A} \PY{o}{\PYZhy{}} \PY{n}{exp}\PY{p}{(}\PY{o}{\PYZhy{}}\PY{n}{h}\PY{o}{*}\PY{n}{A}\PY{p}{)}\PY{o}{*}\PY{n}{B}\PY{o}{*}\PY{n}{K}\PY{p}{)}\PY{o}{.}\PY{n}{det}\PY{p}{(}\PY{p}{)}
\end{Verbatim}
\texttt{\color{outcolor}Out[{\color{outcolor}12}]:}
    
    
        \begin{equation*}\adjustbox{max width=\hsize}{$
        s^{2} + 2 s + 1
        $}\end{equation*}

    

    Por lo que observamos que esta realimentación coloca dos polos en
\(-1\), sin embargo queremos analizar la estabilidad bajo los parametros
que establecimos, por lo que notamos que este polinomio puede ser
escrito como:

\[
s^2 - \left( 1 + \alpha_1 + \alpha_2 \right)s + \alpha_1
\]

    \begin{Verbatim}[commandchars=\\\{\}]
{\color{incolor}In [{\color{incolor}13}]:} \PY{n}{α1}\PY{p}{,} \PY{n}{α2} \PY{o}{=} \PY{n}{K}\PY{p}{[}\PY{l+m+mi}{0}\PY{p}{]} \PY{o}{\PYZhy{}} \PY{n}{K}\PY{p}{[}\PY{l+m+mi}{1}\PY{p}{]}\PY{p}{,} \PY{n}{exp}\PY{p}{(}\PY{o}{\PYZhy{}}\PY{n}{h}\PY{p}{)}\PY{o}{*}\PY{n}{K}\PY{p}{[}\PY{l+m+mi}{1}\PY{p}{]}
         \PY{n}{α1}\PY{p}{,} \PY{n}{α2}
\end{Verbatim}
\texttt{\color{outcolor}Out[{\color{outcolor}13}]:}
    
    
        \begin{equation*}\adjustbox{max width=\hsize}{$
        \left ( 1, \quad -4\right )
        $}\end{equation*}

    

    \begin{Verbatim}[commandchars=\\\{\}]
{\color{incolor}In [{\color{incolor}14}]:} \PY{n}{s}\PY{o}{*}\PY{o}{*}\PY{l+m+mi}{2} \PY{o}{\PYZhy{}} \PY{p}{(}\PY{l+m+mi}{1} \PY{o}{+} \PY{n}{α1} \PY{o}{+} \PY{n}{α2}\PY{p}{)}\PY{o}{*}\PY{n}{s} \PY{o}{+} \PY{n}{α1}
\end{Verbatim}
\texttt{\color{outcolor}Out[{\color{outcolor}14}]:}
    
    
        \begin{equation*}\adjustbox{max width=\hsize}{$
        s^{2} + 2 s + 1
        $}\end{equation*}

    

    Este polinomio caracteristico esta libre de retardos, por lo que podemos
analizarlo con Routh-Hurwitz y obtener las siguientes condiciones:

\[
\begin{align}
\alpha_1 &> 0 \\
\alpha_1 &< -1 - \alpha_2
\end{align}
\]

Por otro lado, si hacemos un analisis de D-particiones, al sustituir
\(s = 0\) y \(s = j \omega\) obtenemos que:

\[
\begin{align}
\alpha_1 &= 0 \\
\alpha_1 &= -1 - \alpha_2
\end{align}
\]

    \begin{Verbatim}[commandchars=\\\{\}]
{\color{incolor}In [{\color{incolor}15}]:} \PY{n}{var}\PY{p}{(}\PY{l+s}{\PYZdq{}}\PY{l+s}{α\PYZus{}1 α\PYZus{}2 ω}\PY{l+s}{\PYZdq{}}\PY{p}{)}
\end{Verbatim}
\texttt{\color{outcolor}Out[{\color{outcolor}15}]:}
    
    
        \begin{equation*}\adjustbox{max width=\hsize}{$
        \left ( α_{1}, \quad α_{2}, \quad ω\right )
        $}\end{equation*}

    

    \begin{Verbatim}[commandchars=\\\{\}]
{\color{incolor}In [{\color{incolor}16}]:} \PY{p}{(}\PY{n}{s}\PY{o}{*}\PY{o}{*}\PY{l+m+mi}{2} \PY{o}{\PYZhy{}} \PY{p}{(}\PY{l+m+mi}{1} \PY{o}{+} \PY{n}{α\PYZus{}1} \PY{o}{+} \PY{n}{α\PYZus{}2}\PY{p}{)}\PY{o}{*}\PY{n}{s} \PY{o}{+} \PY{n}{α\PYZus{}1}\PY{p}{)}\PY{o}{.}\PY{n}{subs}\PY{p}{(}\PY{n}{s}\PY{p}{,} \PY{l+m+mi}{0}\PY{p}{)}
\end{Verbatim}
\texttt{\color{outcolor}Out[{\color{outcolor}16}]:}
    
    
        \begin{equation*}\adjustbox{max width=\hsize}{$
        α_{1}
        $}\end{equation*}

    

    \begin{Verbatim}[commandchars=\\\{\}]
{\color{incolor}In [{\color{incolor}17}]:} \PY{p}{(}\PY{n}{s}\PY{o}{*}\PY{o}{*}\PY{l+m+mi}{2} \PY{o}{\PYZhy{}} \PY{p}{(}\PY{l+m+mi}{1} \PY{o}{+} \PY{n}{α\PYZus{}1} \PY{o}{+} \PY{n}{α\PYZus{}2}\PY{p}{)}\PY{o}{*}\PY{n}{s} \PY{o}{+} \PY{n}{α\PYZus{}1}\PY{p}{)}\PY{o}{.}\PY{n}{subs}\PY{p}{(}\PY{n}{s}\PY{p}{,} \PY{l+m+mi}{1}\PY{n}{j}\PY{o}{*}\PY{n}{ω}\PY{p}{)}\PY{o}{.}\PY{n}{coeff}\PY{p}{(}\PY{o}{\PYZhy{}}\PY{l+m+mi}{1}\PY{n}{j}\PY{o}{*}\PY{n}{ω}\PY{p}{)}
\end{Verbatim}
\texttt{\color{outcolor}Out[{\color{outcolor}17}]:}
    
    
        \begin{equation*}\adjustbox{max width=\hsize}{$
        α_{1} + α_{2} + 1
        $}\end{equation*}

    

    Lo cual es consistente con los resultados de Routh-Hurwitz. Al graficar
estas curvas limite de las D-particiones, obtenemos:

    \begin{Verbatim}[commandchars=\\\{\}]
{\color{incolor}In [{\color{incolor}18}]:} \PY{k+kn}{from} \PY{n+nn}{numpy} \PY{k}{import} \PY{n}{linspace}\PY{p}{,} \PY{n}{zeros}
\end{Verbatim}

    \begin{Verbatim}[commandchars=\\\{\}]
{\color{incolor}In [{\color{incolor}19}]:} \PY{o}{\PYZpc{}}\PY{k}{matplotlib} inline
         \PY{k+kn}{from} \PY{n+nn}{matplotlib}\PY{n+nn}{.}\PY{n+nn}{pyplot} \PY{k}{import} \PY{n}{plot}\PY{p}{,} \PY{n}{style}\PY{p}{,} \PY{n}{figure}\PY{p}{,} \PY{n}{legend}\PY{p}{,} \PY{n}{fill}
         \PY{n}{style}\PY{o}{.}\PY{n}{use}\PY{p}{(}\PY{l+s}{\PYZdq{}}\PY{l+s}{ggplot}\PY{l+s}{\PYZdq{}}\PY{p}{)}
\end{Verbatim}

    \begin{Verbatim}[commandchars=\\\{\}]
{\color{incolor}In [{\color{incolor}20}]:} \PY{n}{x} \PY{o}{=} \PY{n}{linspace}\PY{p}{(}\PY{o}{\PYZhy{}}\PY{l+m+mi}{4}\PY{p}{,} \PY{o}{\PYZhy{}}\PY{l+m+mi}{1}\PY{p}{,} \PY{l+m+mi}{100}\PY{p}{)}
         \PY{n}{alpha1} \PY{o}{=} \PY{n}{linspace}\PY{p}{(}\PY{o}{\PYZhy{}}\PY{l+m+mi}{4}\PY{p}{,} \PY{l+m+mi}{4}\PY{p}{,} \PY{l+m+mi}{100}\PY{p}{)}
         \PY{n}{alpha2} \PY{o}{=} \PY{o}{\PYZhy{}}\PY{n}{alpha1} \PY{o}{\PYZhy{}} \PY{l+m+mi}{1}
\end{Verbatim}

    \begin{Verbatim}[commandchars=\\\{\}]
{\color{incolor}In [{\color{incolor}21}]:} \PY{n}{f} \PY{o}{=} \PY{n}{figure}\PY{p}{(}\PY{n}{figsize}\PY{o}{=}\PY{p}{(}\PY{l+m+mi}{8}\PY{p}{,} \PY{l+m+mi}{8}\PY{p}{)}\PY{p}{)}
         \PY{n}{plot}\PY{p}{(}\PY{n}{zeros}\PY{p}{(}\PY{n+nb}{len}\PY{p}{(}\PY{n}{alpha1}\PY{p}{)}\PY{p}{)}\PY{p}{,} \PY{n}{alpha1}\PY{p}{)}
         \PY{n}{plot}\PY{p}{(}\PY{n}{alpha1}\PY{p}{,} \PY{n}{alpha2}\PY{p}{)}
         
         \PY{n}{ax} \PY{o}{=} \PY{n}{f}\PY{o}{.}\PY{n}{gca}\PY{p}{(}\PY{p}{)}
         \PY{n}{ax}\PY{o}{.}\PY{n}{set\PYZus{}xlim}\PY{p}{(}\PY{o}{\PYZhy{}}\PY{l+m+mi}{3}\PY{p}{,} \PY{l+m+mi}{3}\PY{p}{)}
         \PY{n}{ax}\PY{o}{.}\PY{n}{set\PYZus{}ylim}\PY{p}{(}\PY{o}{\PYZhy{}}\PY{l+m+mi}{3}\PY{p}{,} \PY{l+m+mi}{3}\PY{p}{)}
         
         \PY{n}{ax}\PY{o}{.}\PY{n}{fill\PYZus{}betweenx}\PY{p}{(}\PY{n}{alpha2}\PY{p}{,} \PY{l+m+mi}{0}\PY{p}{,} \PY{n}{alpha1}\PY{p}{,} \PY{n}{where}\PY{o}{=}\PY{n}{alpha1}\PY{o}{\PYZgt{}}\PY{l+m+mi}{0}\PY{p}{,} \PY{n}{alpha}\PY{o}{=}\PY{l+m+mf}{0.3}\PY{p}{,} \PY{n}{facecolor}\PY{o}{=}\PY{l+s}{\PYZsq{}}\PY{l+s}{purple}\PY{l+s}{\PYZsq{}}\PY{p}{)}
         
         \PY{n}{ax}\PY{o}{.}\PY{n}{set\PYZus{}xlabel}\PY{p}{(}\PY{l+s}{r\PYZdq{}}\PY{l+s}{\PYZdl{}α\PYZus{}1\PYZdl{}}\PY{l+s}{\PYZdq{}}\PY{p}{,} \PY{n}{fontsize}\PY{o}{=}\PY{l+m+mi}{20}\PY{p}{)}
         \PY{n}{ax}\PY{o}{.}\PY{n}{set\PYZus{}ylabel}\PY{p}{(}\PY{l+s}{r\PYZdq{}}\PY{l+s}{\PYZdl{}α\PYZus{}2\PYZdl{}}\PY{l+s}{\PYZdq{}}\PY{p}{,} \PY{n}{fontsize}\PY{o}{=}\PY{l+m+mi}{20}\PY{p}{)}\PY{p}{;}
\end{Verbatim}

    \begin{center}
    \adjustimage{max size={0.5\linewidth}{0.9\paperheight}}{Tarea 10_files/Tarea 10_28_0.png}
    \end{center}
    { \hspace*{\fill} \\}
    
    \subsection*{Ejemplo 2}\label{ejemplo-2}

    Para el sistema:

\[
\dot{x}(t) =
\begin{pmatrix}
0 & 1 \\
0 & 0
\end{pmatrix}
x(t) +
\begin{pmatrix}
0 \\
1
\end{pmatrix}
u(t - h)
\]

con \(h = 1\).

    \begin{Verbatim}[commandchars=\\\{\}]
{\color{incolor}In [{\color{incolor}22}]:} \PY{n}{var}\PY{p}{(}\PY{l+s}{\PYZdq{}}\PY{l+s}{k1 k2}\PY{l+s}{\PYZdq{}}\PY{p}{)}
\end{Verbatim}
\texttt{\color{outcolor}Out[{\color{outcolor}22}]:}
    
    
        \begin{equation*}\adjustbox{max width=\hsize}{$
        \left ( k_{1}, \quad k_{2}\right )
        $}\end{equation*}

    

    \begin{Verbatim}[commandchars=\\\{\}]
{\color{incolor}In [{\color{incolor}23}]:} \PY{n}{A1} \PY{o}{=} \PY{n}{Matrix}\PY{p}{(}\PY{p}{[}\PY{p}{[}\PY{l+m+mi}{0}\PY{p}{,} \PY{l+m+mi}{1}\PY{p}{]}\PY{p}{,} \PY{p}{[}\PY{l+m+mi}{0}\PY{p}{,} \PY{l+m+mi}{0}\PY{p}{]}\PY{p}{]}\PY{p}{)}
         \PY{n}{B1} \PY{o}{=} \PY{n}{Matrix}\PY{p}{(}\PY{p}{[}\PY{p}{[}\PY{l+m+mi}{0}\PY{p}{]}\PY{p}{,} \PY{p}{[}\PY{l+m+mi}{1}\PY{p}{]}\PY{p}{]}\PY{p}{)}
         \PY{n}{K1} \PY{o}{=} \PY{n}{Matrix}\PY{p}{(}\PY{p}{[}\PY{p}{[}\PY{n}{k1}\PY{p}{,} \PY{n}{k2}\PY{p}{]}\PY{p}{]}\PY{p}{)}
\end{Verbatim}

    Tiene un polinomio caracteristico:

    \begin{Verbatim}[commandchars=\\\{\}]
{\color{incolor}In [{\color{incolor}24}]:} \PY{p}{(}\PY{p}{(}\PY{n}{s}\PY{o}{*}\PY{n}{eye}\PY{p}{(}\PY{l+m+mi}{2}\PY{p}{)} \PY{o}{\PYZhy{}} \PY{n}{A1} \PY{o}{\PYZhy{}} \PY{n}{exp}\PY{p}{(}\PY{o}{\PYZhy{}}\PY{n}{A1}\PY{o}{*}\PY{n}{h}\PY{p}{)}\PY{o}{*}\PY{n}{B1}\PY{o}{*}\PY{n}{K1}\PY{p}{)}\PY{o}{.}\PY{n}{det}\PY{p}{(}\PY{p}{)}\PY{p}{)}\PY{o}{.}\PY{n}{collect}\PY{p}{(}\PY{n}{s}\PY{p}{)}
\end{Verbatim}
\texttt{\color{outcolor}Out[{\color{outcolor}24}]:}
    
    
        \begin{equation*}\adjustbox{max width=\hsize}{$
        - k_{1} + s^{2} + s \left(h k_{1} - k_{2}\right)
        $}\end{equation*}

    

    o bien:

\[
s^2 +  \left( h k_1 - k_2 \right) s - k_1
\]

Al cual podemos aplicar el criterio de estabilidad de Routh-Hurwitz y
obtener:

\[
\begin{align}
k_1 &< 0 \\
k_2 &< h k_1
\end{align}
\]

Por lo que la gráfica de D-particiones se verá:

    \begin{Verbatim}[commandchars=\\\{\}]
{\color{incolor}In [{\color{incolor}25}]:} \PY{n}{h} \PY{o}{=} \PY{l+m+mi}{1}
         \PY{n}{x} \PY{o}{=} \PY{n}{linspace}\PY{p}{(}\PY{o}{\PYZhy{}}\PY{l+m+mi}{4}\PY{p}{,} \PY{o}{\PYZhy{}}\PY{l+m+mi}{1}\PY{p}{,} \PY{l+m+mi}{100}\PY{p}{)}
         \PY{n}{K\PYZus{}1} \PY{o}{=} \PY{n}{linspace}\PY{p}{(}\PY{o}{\PYZhy{}}\PY{l+m+mi}{4}\PY{p}{,} \PY{l+m+mi}{4}\PY{p}{,} \PY{l+m+mi}{100}\PY{p}{)}
         \PY{n}{K\PYZus{}2} \PY{o}{=} \PY{n}{h}\PY{o}{*}\PY{n}{K\PYZus{}1}
\end{Verbatim}

    \begin{Verbatim}[commandchars=\\\{\}]
{\color{incolor}In [{\color{incolor}26}]:} \PY{n}{f} \PY{o}{=} \PY{n}{figure}\PY{p}{(}\PY{n}{figsize}\PY{o}{=}\PY{p}{(}\PY{l+m+mi}{8}\PY{p}{,} \PY{l+m+mi}{8}\PY{p}{)}\PY{p}{)}
         \PY{n}{plot}\PY{p}{(}\PY{n}{zeros}\PY{p}{(}\PY{n+nb}{len}\PY{p}{(}\PY{n}{K\PYZus{}1}\PY{p}{)}\PY{p}{)}\PY{p}{,} \PY{n}{K\PYZus{}1}\PY{p}{)}
         \PY{n}{plot}\PY{p}{(}\PY{n}{K\PYZus{}1}\PY{p}{,} \PY{n}{K\PYZus{}2}\PY{p}{)}
         
         \PY{n}{ax} \PY{o}{=} \PY{n}{f}\PY{o}{.}\PY{n}{gca}\PY{p}{(}\PY{p}{)}
         \PY{n}{ax}\PY{o}{.}\PY{n}{set\PYZus{}xlim}\PY{p}{(}\PY{o}{\PYZhy{}}\PY{l+m+mi}{3}\PY{p}{,} \PY{l+m+mi}{3}\PY{p}{)}
         \PY{n}{ax}\PY{o}{.}\PY{n}{set\PYZus{}ylim}\PY{p}{(}\PY{o}{\PYZhy{}}\PY{l+m+mi}{3}\PY{p}{,} \PY{l+m+mi}{3}\PY{p}{)}
         
         \PY{n}{ax}\PY{o}{.}\PY{n}{fill\PYZus{}betweenx}\PY{p}{(}\PY{n}{K\PYZus{}2}\PY{p}{,} \PY{l+m+mi}{0}\PY{p}{,} \PY{n}{K\PYZus{}1}\PY{p}{,} \PY{n}{where}\PY{o}{=}\PY{n}{K\PYZus{}1}\PY{o}{\PYZlt{}}\PY{l+m+mi}{0}\PY{p}{,} \PY{n}{alpha}\PY{o}{=}\PY{l+m+mf}{0.3}\PY{p}{,} \PY{n}{facecolor}\PY{o}{=}\PY{l+s}{\PYZsq{}}\PY{l+s}{green}\PY{l+s}{\PYZsq{}}\PY{p}{)}
         
         \PY{n}{ax}\PY{o}{.}\PY{n}{set\PYZus{}xlabel}\PY{p}{(}\PY{l+s}{r\PYZdq{}}\PY{l+s}{\PYZdl{}k\PYZus{}1\PYZdl{}}\PY{l+s}{\PYZdq{}}\PY{p}{,} \PY{n}{fontsize}\PY{o}{=}\PY{l+m+mi}{20}\PY{p}{)}
         \PY{n}{ax}\PY{o}{.}\PY{n}{set\PYZus{}ylabel}\PY{p}{(}\PY{l+s}{r\PYZdq{}}\PY{l+s}{\PYZdl{}k\PYZus{}2\PYZdl{}}\PY{l+s}{\PYZdq{}}\PY{p}{,} \PY{n}{fontsize}\PY{o}{=}\PY{l+m+mi}{20}\PY{p}{)}\PY{p}{;}
\end{Verbatim}

    \begin{center}
    \adjustimage{max size={0.5\linewidth}{0.9\paperheight}}{Tarea 10_files/Tarea 10_37_0.png}
    \end{center}
    { \hspace*{\fill} \\}
    
    Por otro lado, para analizar el comportamiento del controlador,
sustituimos los datos en la ecuación del controlador:

\[
\begin{align}
u(t) &=
\begin{pmatrix}
k_1 & k_2
\end{pmatrix} x(t) +
\begin{pmatrix}
k_1 & k_2
\end{pmatrix}
\int_{-h}^0 e^{-A(\theta + h)} B u(t + \theta) d\theta \\
&=
\begin{pmatrix}
k_1 & k_2
\end{pmatrix}
\begin{pmatrix}
x_1(t) \\
x_2(t)
\end{pmatrix} +
\begin{pmatrix}
k_1 & k_2
\end{pmatrix}
\int_{-h}^0 e^{-A(\theta + h)} B u(t + \theta) d\theta \\
\end{align}
\]

    \begin{Verbatim}[commandchars=\\\{\}]
{\color{incolor}In [{\color{incolor}27}]:} \PY{n}{exp}\PY{p}{(}\PY{o}{\PYZhy{}}\PY{n}{A1}\PY{o}{*}\PY{p}{(}\PY{n}{θ} \PY{o}{+} \PY{n}{h}\PY{p}{)}\PY{p}{)}
\end{Verbatim}
\texttt{\color{outcolor}Out[{\color{outcolor}27}]:}
    
    
        \begin{equation*}\adjustbox{max width=\hsize}{$
        \left[\begin{matrix}1 & - θ - 1\\0 & 1\end{matrix}\right]
        $}\end{equation*}

    

    \[
\begin{align}
u(t) &=
\begin{pmatrix}
k_1 & k_2
\end{pmatrix}
\begin{pmatrix}
x_1(t) \\
x_2(t)
\end{pmatrix} +
\int_{-h}^0
\begin{pmatrix}
k_1 & k_2
\end{pmatrix}
\begin{pmatrix}
1 & - (\theta  +h) \\
0 & 1
\end{pmatrix}
\begin{pmatrix}
0 \\
1
\end{pmatrix} u(t + \theta) d\theta \\
&= k_1 x_1(t) + k_2 x_2(t) - \int_{-h}^0 k_1 \theta u(t + \theta) d\theta - \int_{-h}^0 k_1 h u(t + \theta) d\theta + \int_{-h}^0 k_2 u(t + \theta) d\theta \\
\end{align}
\]

y al aplicar la transformada de Laplace, tenemos:

\[
u(s) = k_1 x_1(s) + k_2 x_2(s) - k_1 \left( 1 - e^{-hs} \right) u(s) - k_1 h \frac{1 - e^{-hs}}{s} u(s) + k_2 \frac{1 - e^{-hs}}{s} u(s)
\]

por lo que al pasar a un solo lado todos los terminos de \(u(s)\):

\[
\left[ 1 + k_1 \left( 1 - e^{-hs} \right) + k_1 h \frac{1 - e^{-hs}}{s} - k_2 \frac{1 - e^{-hs}}{s} \right] u(s) = k_1 x_1(s) + k_2 x_2(s)
\]

obtenemos el polinomio caracteristico de la ecuación de control:

\[
1 + k_1 \left( 1 - e^{-hs} \right) + k_1 h \frac{1 - e^{-hs}}{s} - k_2 \frac{1 - e^{-hs}}{s}
\]

y al sustituir \(s = j \omega\), obtendremos dos ecuaciones,
correspondientes a la parte real e imaginaria:

\[
\begin{align}
k_1 \left[ h (1 - \cos{(\omega h)}) - \omega \sin{(\omega h)} \right] - k_2 \left[ 1 - \cos{(\omega h)} \right] &= 0 \\
\omega + k_1 \left[ \omega (1 - \cos{(\omega h)}) - h \sin{(\omega h)} \right] - k_2 \sin{(\omega h)} &= 0 \\
\end{align}
\]

por lo que podemos despejar \(k_2\) de ambas ecuaciones y obtener:

\[
k_2 = \frac{k_1 \left[ h (1 - \cos{(\omega h)}) - \omega \sin{(\omega h)} \right]}{1 - \cos{(\omega h)}} = \frac{k_1 \left[ \omega (1 - \cos{(\omega h)}) - h \sin{(\omega h)} \right] + \omega}{\sin{(\omega h)}}
\]

y haciendo un poco de algebra, podemos obtener:

\[
\begin{align}
\frac{k_1 \left[ h (1 - \cos{(\omega h)}) - \omega \sin{(\omega h)} \right] \sin{(\omega h)}}{1 - \cos{(\omega h)}} - k_1 \left[ \omega (1 - \cos{(\omega h)}) - h \sin{(\omega h)} \right] &= \omega \\
\frac{k_1 h (1 - \cos{(\omega h)}) \sin{(\omega h)}}{1 - \cos{(\omega h)}} - \frac{k_1 \omega \sin^2{(\omega h)}}{1 - \cos{(\omega h)}} - k_1 \omega (1 - \cos{(\omega h)}) - k_1 h \sin{(\omega h)} &= \omega
\end{align}
\]

\[
\begin{align}
- \frac{k_1 \omega \sin^2{(\omega h)}}{1 - \cos{(\omega h)}} - k_1 \omega (1 - \cos{(\omega h)}) &= \omega \\
- k_1 \omega \frac{ \sin^2{(\omega h)} + (1 - \cos{(\omega h)})^2}{1 - \cos{(\omega h)}} &= \omega \\
k_1 \frac{ \sin^2{(\omega h)} + (1 - \cos{(\omega h)})^2}{1 - \cos{(\omega h)}} &= -1 \\
k_1 \frac{ \sin^2{(\omega h)} + 1 + \cos^2{(\omega h)} - 2\cos{(\omega h)}}{1 - \cos{(\omega h)}} &= -1 \\
k_1 \frac{ 2 - 2\cos{(\omega h)}}{1 - \cos{(\omega h)}} &= -1 \\
k_1 &= - \frac{1}{2}
\end{align}
\]

por otro lado, podemos reducir aun mas la expresión para \(k_2\):

\[
k_2 = \frac{k_1 \left[ h (1 - \cos{(\omega h)}) - \omega \sin{(\omega h)} \right]}{1 - \cos{(\omega h)}} = k_1 \left[ h - \frac{\omega \sin{(\omega h)}}{1 - \cos{\omega h}} \right]
\]

Si sustituimos un punto a la derecha de esta curva,
\((k_1, k_2) = (0, 0)\), podemos ver que el polinomio caracteristico es
trivialmente estable por el criterio de Routh-Hurwitz:

\[
P(s) = 1
\]

por lo que la gráfica de D-particiones para el controlador queda:

    \begin{Verbatim}[commandchars=\\\{\}]
{\color{incolor}In [{\color{incolor}28}]:} \PY{k}{def} \PY{n+nf}{par1}\PY{p}{(}\PY{n}{ω}\PY{p}{,} \PY{n}{h}\PY{p}{)}\PY{p}{:}
             \PY{k+kn}{from} \PY{n+nn}{numpy} \PY{k}{import} \PY{n}{sin}\PY{p}{,} \PY{n}{cos}
             \PY{n}{num} \PY{o}{=} \PY{o}{\PYZhy{}}\PY{l+m+mf}{1.0}
             \PY{n}{den} \PY{o}{=} \PY{l+m+mf}{2.0}
             \PY{k}{return} \PY{n}{num}\PY{o}{/}\PY{n}{den}
             
         \PY{k}{def} \PY{n+nf}{par2}\PY{p}{(}\PY{n}{ω}\PY{p}{,} \PY{n}{h}\PY{p}{)}\PY{p}{:}
             \PY{k+kn}{from} \PY{n+nn}{numpy} \PY{k}{import} \PY{n}{sin}\PY{p}{,} \PY{n}{cos}
             \PY{n}{num} \PY{o}{=} \PY{n}{h} \PY{o}{\PYZhy{}} \PY{n}{ω}\PY{o}{*}\PY{n}{sin}\PY{p}{(}\PY{n}{ω}\PY{o}{*}\PY{n}{h}\PY{p}{)}\PY{o}{/}\PY{p}{(}\PY{l+m+mi}{1} \PY{o}{\PYZhy{}} \PY{n}{cos}\PY{p}{(}\PY{n}{ω}\PY{o}{*}\PY{n}{h}\PY{p}{)}\PY{p}{)}
             \PY{n}{den} \PY{o}{=} \PY{l+m+mf}{2.0}
             \PY{k}{return} \PY{n}{num}\PY{o}{/}\PY{n}{den}
\end{Verbatim}

    \begin{Verbatim}[commandchars=\\\{\}]
{\color{incolor}In [{\color{incolor}29}]:} \PY{k+kn}{from} \PY{n+nn}{numpy} \PY{k}{import} \PY{n}{pi}
\end{Verbatim}

    \begin{Verbatim}[commandchars=\\\{\}]
{\color{incolor}In [{\color{incolor}30}]:} \PY{n}{τ} \PY{o}{=} \PY{l+m+mi}{2}\PY{o}{*}\PY{n}{pi}
         \PY{n}{k\PYZus{}1} \PY{o}{=} \PY{p}{[}\PY{n}{par1}\PY{p}{(}\PY{n}{om}\PY{p}{,} \PY{l+m+mf}{1.0}\PY{p}{)} \PY{k}{for} \PY{n}{om} \PY{o+ow}{in} \PY{n}{linspace}\PY{p}{(}\PY{o}{\PYZhy{}}\PY{l+m+mi}{2}\PY{o}{*}\PY{n}{τ}\PY{p}{,} \PY{l+m+mi}{2}\PY{o}{*}\PY{n}{τ}\PY{p}{,} \PY{l+m+mi}{100}\PY{p}{)}\PY{p}{]}
         \PY{n}{k\PYZus{}2} \PY{o}{=} \PY{p}{[}\PY{n}{par2}\PY{p}{(}\PY{n}{om}\PY{p}{,} \PY{l+m+mf}{1.0}\PY{p}{)} \PY{k}{for} \PY{n}{om} \PY{o+ow}{in} \PY{n}{linspace}\PY{p}{(}\PY{o}{\PYZhy{}}\PY{l+m+mi}{2}\PY{o}{*}\PY{n}{τ}\PY{p}{,} \PY{l+m+mi}{2}\PY{o}{*}\PY{n}{τ}\PY{p}{,} \PY{l+m+mi}{100}\PY{p}{)}\PY{p}{]}
\end{Verbatim}

    \begin{Verbatim}[commandchars=\\\{\}]
/Users/roberto/miniconda3/lib/python3.4/site-packages/IPython/kernel/\_\_main\_\_.py:9: RuntimeWarning: divide by zero encountered in double\_scalars
    \end{Verbatim}

    \begin{Verbatim}[commandchars=\\\{\}]
{\color{incolor}In [{\color{incolor}31}]:} \PY{n}{f} \PY{o}{=} \PY{n}{figure}\PY{p}{(}\PY{n}{figsize}\PY{o}{=}\PY{p}{(}\PY{l+m+mi}{8}\PY{p}{,} \PY{l+m+mi}{8}\PY{p}{)}\PY{p}{)}
         \PY{n}{plot}\PY{p}{(}\PY{n}{k\PYZus{}1}\PY{p}{,} \PY{n}{k\PYZus{}2}\PY{p}{)}
         
         \PY{n}{ax} \PY{o}{=} \PY{n}{f}\PY{o}{.}\PY{n}{gca}\PY{p}{(}\PY{p}{)}
         \PY{n}{ax}\PY{o}{.}\PY{n}{set\PYZus{}xlim}\PY{p}{(}\PY{o}{\PYZhy{}}\PY{l+m+mi}{3}\PY{p}{,} \PY{l+m+mi}{3}\PY{p}{)}
         \PY{n}{ax}\PY{o}{.}\PY{n}{set\PYZus{}ylim}\PY{p}{(}\PY{o}{\PYZhy{}}\PY{l+m+mi}{3}\PY{p}{,} \PY{l+m+mi}{3}\PY{p}{)}
         
         \PY{n}{ax}\PY{o}{.}\PY{n}{axvspan}\PY{p}{(}\PY{o}{\PYZhy{}}\PY{l+m+mf}{0.5}\PY{p}{,} \PY{l+m+mi}{3}\PY{p}{,} \PY{n}{alpha}\PY{o}{=}\PY{l+m+mf}{0.3}\PY{p}{,} \PY{n}{color}\PY{o}{=}\PY{l+s}{\PYZsq{}}\PY{l+s}{orange}\PY{l+s}{\PYZsq{}}\PY{p}{)}
         
         \PY{n}{ax}\PY{o}{.}\PY{n}{set\PYZus{}xlabel}\PY{p}{(}\PY{l+s}{r\PYZdq{}}\PY{l+s}{\PYZdl{}k\PYZus{}1\PYZdl{}}\PY{l+s}{\PYZdq{}}\PY{p}{,} \PY{n}{fontsize}\PY{o}{=}\PY{l+m+mi}{20}\PY{p}{)}
         \PY{n}{ax}\PY{o}{.}\PY{n}{set\PYZus{}ylabel}\PY{p}{(}\PY{l+s}{r\PYZdq{}}\PY{l+s}{\PYZdl{}k\PYZus{}2\PYZdl{}}\PY{l+s}{\PYZdq{}}\PY{p}{,} \PY{n}{fontsize}\PY{o}{=}\PY{l+m+mi}{20}\PY{p}{)}\PY{p}{;}
\end{Verbatim}

    \begin{center}
    \adjustimage{max size={0.5\linewidth}{0.9\paperheight}}{Tarea 10_files/Tarea 10_44_0.png}
    \end{center}
    { \hspace*{\fill} \\}
    
    Y el sistema con este controlador será estable para los valores de
\(k_1\) y \(k_2\) escogidos tal que se encuentren en la intersección de
estas dos regiones:

    \begin{Verbatim}[commandchars=\\\{\}]
{\color{incolor}In [{\color{incolor}32}]:} \PY{n}{f} \PY{o}{=} \PY{n}{figure}\PY{p}{(}\PY{n}{figsize}\PY{o}{=}\PY{p}{(}\PY{l+m+mi}{8}\PY{p}{,} \PY{l+m+mi}{8}\PY{p}{)}\PY{p}{)}
         \PY{n}{plot}\PY{p}{(}\PY{n}{zeros}\PY{p}{(}\PY{n+nb}{len}\PY{p}{(}\PY{n}{K\PYZus{}1}\PY{p}{)}\PY{p}{)}\PY{p}{,} \PY{n}{K\PYZus{}1}\PY{p}{)}
         \PY{n}{plot}\PY{p}{(}\PY{n}{K\PYZus{}1}\PY{p}{,} \PY{n}{K\PYZus{}2}\PY{p}{)}
         \PY{n}{plot}\PY{p}{(}\PY{n}{k\PYZus{}1}\PY{p}{,} \PY{n}{k\PYZus{}2}\PY{p}{)}
         
         \PY{n}{ax} \PY{o}{=} \PY{n}{f}\PY{o}{.}\PY{n}{gca}\PY{p}{(}\PY{p}{)}
         \PY{n}{ax}\PY{o}{.}\PY{n}{set\PYZus{}xlim}\PY{p}{(}\PY{o}{\PYZhy{}}\PY{l+m+mi}{3}\PY{p}{,} \PY{l+m+mi}{3}\PY{p}{)}
         \PY{n}{ax}\PY{o}{.}\PY{n}{set\PYZus{}ylim}\PY{p}{(}\PY{o}{\PYZhy{}}\PY{l+m+mi}{3}\PY{p}{,} \PY{l+m+mi}{3}\PY{p}{)}
         
         \PY{n}{ax}\PY{o}{.}\PY{n}{fill\PYZus{}betweenx}\PY{p}{(}\PY{n}{K\PYZus{}2}\PY{p}{,} \PY{l+m+mi}{0}\PY{p}{,} \PY{n}{K\PYZus{}1}\PY{p}{,} \PY{n}{where}\PY{o}{=}\PY{n}{K\PYZus{}1}\PY{o}{\PYZlt{}}\PY{l+m+mi}{0}\PY{p}{,} \PY{n}{alpha}\PY{o}{=}\PY{l+m+mf}{0.4}\PY{p}{,} \PY{n}{facecolor}\PY{o}{=}\PY{l+s}{\PYZsq{}}\PY{l+s}{green}\PY{l+s}{\PYZsq{}}\PY{p}{)}
         \PY{n}{ax}\PY{o}{.}\PY{n}{axvspan}\PY{p}{(}\PY{o}{\PYZhy{}}\PY{l+m+mf}{0.5}\PY{p}{,} \PY{l+m+mi}{3}\PY{p}{,} \PY{n}{alpha}\PY{o}{=}\PY{l+m+mf}{0.3}\PY{p}{,} \PY{n}{color}\PY{o}{=}\PY{l+s}{\PYZsq{}}\PY{l+s}{orange}\PY{l+s}{\PYZsq{}}\PY{p}{)}
         
         \PY{n}{ax}\PY{o}{.}\PY{n}{set\PYZus{}xlabel}\PY{p}{(}\PY{l+s}{r\PYZdq{}}\PY{l+s}{\PYZdl{}k\PYZus{}1\PYZdl{}}\PY{l+s}{\PYZdq{}}\PY{p}{,} \PY{n}{fontsize}\PY{o}{=}\PY{l+m+mi}{20}\PY{p}{)}
         \PY{n}{ax}\PY{o}{.}\PY{n}{set\PYZus{}ylabel}\PY{p}{(}\PY{l+s}{r\PYZdq{}}\PY{l+s}{\PYZdl{}k\PYZus{}2\PYZdl{}}\PY{l+s}{\PYZdq{}}\PY{p}{,} \PY{n}{fontsize}\PY{o}{=}\PY{l+m+mi}{20}\PY{p}{)}\PY{p}{;}
\end{Verbatim}

    \begin{center}
    \adjustimage{max size={0.5\linewidth}{0.9\paperheight}}{Tarea 10_files/Tarea 10_46_0.png}
    \end{center}
    { \hspace*{\fill} \\}
    
    Puedes acceder a este notebook a traves de la página

http://bit.ly/1B705kd

o escaneando el siguiente código:

\begin{figure}[htbp]
\centering
\includegraphics[width=0.2\textwidth]{../codigos/codigo10.jpg}
\end{figure}


    % Add a bibliography block to the postdoc
    
    
    
    \end{document}
