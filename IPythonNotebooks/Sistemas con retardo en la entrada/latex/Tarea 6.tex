
% Default to the notebook output style

    


% Inherit from the specified cell style.




    
\documentclass{article}

    
    
    \usepackage{graphicx} % Used to insert images
    \usepackage{adjustbox} % Used to constrain images to a maximum size 
    \usepackage{color} % Allow colors to be defined
    \usepackage{enumerate} % Needed for markdown enumerations to work
    \usepackage{geometry} % Used to adjust the document margins
    \usepackage{amsmath} % Equations
    \usepackage{amssymb} % Equations
    \usepackage[mathletters]{ucs} % Extended unicode (utf-8) support
    \usepackage[utf8x]{inputenc} % Allow utf-8 characters in the tex document
    \usepackage{fancyvrb} % verbatim replacement that allows latex
    \usepackage{grffile} % extends the file name processing of package graphics 
                         % to support a larger range 
    % The hyperref package gives us a pdf with properly built
    % internal navigation ('pdf bookmarks' for the table of contents,
    % internal cross-reference links, web links for URLs, etc.)
    \usepackage{hyperref}
    \usepackage{longtable} % longtable support required by pandoc >1.10
    \usepackage{booktabs}  % table support for pandoc > 1.12.2
    \usepackage{mathpazo}
    \usepackage[spanish]{babel}

    
    
    \definecolor{orange}{cmyk}{0,0.4,0.8,0.2}
    \definecolor{darkorange}{rgb}{.71,0.21,0.01}
    \definecolor{darkgreen}{rgb}{.12,.54,.11}
    \definecolor{myteal}{rgb}{.26, .44, .56}
    \definecolor{gray}{gray}{0.45}
    \definecolor{lightgray}{gray}{.95}
    \definecolor{mediumgray}{gray}{.8}
    \definecolor{inputbackground}{rgb}{.95, .95, .85}
    \definecolor{outputbackground}{rgb}{.95, .95, .95}
    \definecolor{traceback}{rgb}{1, .95, .95}
    % ansi colors
    \definecolor{red}{rgb}{.6,0,0}
    \definecolor{green}{rgb}{0,.65,0}
    \definecolor{brown}{rgb}{0.6,0.6,0}
    \definecolor{blue}{rgb}{0,.145,.698}
    \definecolor{purple}{rgb}{.698,.145,.698}
    \definecolor{cyan}{rgb}{0,.698,.698}
    \definecolor{lightgray}{gray}{0.5}
    
    % bright ansi colors
    \definecolor{darkgray}{gray}{0.25}
    \definecolor{lightred}{rgb}{1.0,0.39,0.28}
    \definecolor{lightgreen}{rgb}{0.48,0.99,0.0}
    \definecolor{lightblue}{rgb}{0.53,0.81,0.92}
    \definecolor{lightpurple}{rgb}{0.87,0.63,0.87}
    \definecolor{lightcyan}{rgb}{0.5,1.0,0.83}
    
    % commands and environments needed by pandoc snippets
    % extracted from the output of `pandoc -s`
    \DefineVerbatimEnvironment{Highlighting}{Verbatim}{commandchars=\\\{\}}
    % Add ',fontsize=\small' for more characters per line
    \newenvironment{Shaded}{}{}
    \newcommand{\KeywordTok}[1]{\textcolor[rgb]{0.00,0.44,0.13}{\textbf{{#1}}}}
    \newcommand{\DataTypeTok}[1]{\textcolor[rgb]{0.56,0.13,0.00}{{#1}}}
    \newcommand{\DecValTok}[1]{\textcolor[rgb]{0.25,0.63,0.44}{{#1}}}
    \newcommand{\BaseNTok}[1]{\textcolor[rgb]{0.25,0.63,0.44}{{#1}}}
    \newcommand{\FloatTok}[1]{\textcolor[rgb]{0.25,0.63,0.44}{{#1}}}
    \newcommand{\CharTok}[1]{\textcolor[rgb]{0.25,0.44,0.63}{{#1}}}
    \newcommand{\StringTok}[1]{\textcolor[rgb]{0.25,0.44,0.63}{{#1}}}
    \newcommand{\CommentTok}[1]{\textcolor[rgb]{0.38,0.63,0.69}{\textit{{#1}}}}
    \newcommand{\OtherTok}[1]{\textcolor[rgb]{0.00,0.44,0.13}{{#1}}}
    \newcommand{\AlertTok}[1]{\textcolor[rgb]{1.00,0.00,0.00}{\textbf{{#1}}}}
    \newcommand{\FunctionTok}[1]{\textcolor[rgb]{0.02,0.16,0.49}{{#1}}}
    \newcommand{\RegionMarkerTok}[1]{{#1}}
    \newcommand{\ErrorTok}[1]{\textcolor[rgb]{1.00,0.00,0.00}{\textbf{{#1}}}}
    \newcommand{\NormalTok}[1]{{#1}}
    
    % Define a nice break command that doesn't care if a line doesn't already
    % exist.
    \def\br{\hspace*{\fill} \\* }
    % Math Jax compatability definitions
    \def\gt{>}
    \def\lt{<}
    % Document parameters
    \title{Tarea 6}
    
    
    

    % Pygments definitions
    
\makeatletter
\def\PY@reset{\let\PY@it=\relax \let\PY@bf=\relax%
    \let\PY@ul=\relax \let\PY@tc=\relax%
    \let\PY@bc=\relax \let\PY@ff=\relax}
\def\PY@tok#1{\csname PY@tok@#1\endcsname}
\def\PY@toks#1+{\ifx\relax#1\empty\else%
    \PY@tok{#1}\expandafter\PY@toks\fi}
\def\PY@do#1{\PY@bc{\PY@tc{\PY@ul{%
    \PY@it{\PY@bf{\PY@ff{#1}}}}}}}
\def\PY#1#2{\PY@reset\PY@toks#1+\relax+\PY@do{#2}}

\expandafter\def\csname PY@tok@gd\endcsname{\def\PY@tc##1{\textcolor[rgb]{0.63,0.00,0.00}{##1}}}
\expandafter\def\csname PY@tok@gu\endcsname{\let\PY@bf=\textbf\def\PY@tc##1{\textcolor[rgb]{0.50,0.00,0.50}{##1}}}
\expandafter\def\csname PY@tok@gt\endcsname{\def\PY@tc##1{\textcolor[rgb]{0.00,0.27,0.87}{##1}}}
\expandafter\def\csname PY@tok@gs\endcsname{\let\PY@bf=\textbf}
\expandafter\def\csname PY@tok@gr\endcsname{\def\PY@tc##1{\textcolor[rgb]{1.00,0.00,0.00}{##1}}}
\expandafter\def\csname PY@tok@cm\endcsname{\let\PY@it=\textit\def\PY@tc##1{\textcolor[rgb]{0.25,0.50,0.50}{##1}}}
\expandafter\def\csname PY@tok@vg\endcsname{\def\PY@tc##1{\textcolor[rgb]{0.10,0.09,0.49}{##1}}}
\expandafter\def\csname PY@tok@m\endcsname{\def\PY@tc##1{\textcolor[rgb]{0.40,0.40,0.40}{##1}}}
\expandafter\def\csname PY@tok@mh\endcsname{\def\PY@tc##1{\textcolor[rgb]{0.40,0.40,0.40}{##1}}}
\expandafter\def\csname PY@tok@go\endcsname{\def\PY@tc##1{\textcolor[rgb]{0.53,0.53,0.53}{##1}}}
\expandafter\def\csname PY@tok@ge\endcsname{\let\PY@it=\textit}
\expandafter\def\csname PY@tok@vc\endcsname{\def\PY@tc##1{\textcolor[rgb]{0.10,0.09,0.49}{##1}}}
\expandafter\def\csname PY@tok@il\endcsname{\def\PY@tc##1{\textcolor[rgb]{0.40,0.40,0.40}{##1}}}
\expandafter\def\csname PY@tok@cs\endcsname{\let\PY@it=\textit\def\PY@tc##1{\textcolor[rgb]{0.25,0.50,0.50}{##1}}}
\expandafter\def\csname PY@tok@cp\endcsname{\def\PY@tc##1{\textcolor[rgb]{0.74,0.48,0.00}{##1}}}
\expandafter\def\csname PY@tok@gi\endcsname{\def\PY@tc##1{\textcolor[rgb]{0.00,0.63,0.00}{##1}}}
\expandafter\def\csname PY@tok@gh\endcsname{\let\PY@bf=\textbf\def\PY@tc##1{\textcolor[rgb]{0.00,0.00,0.50}{##1}}}
\expandafter\def\csname PY@tok@ni\endcsname{\let\PY@bf=\textbf\def\PY@tc##1{\textcolor[rgb]{0.60,0.60,0.60}{##1}}}
\expandafter\def\csname PY@tok@nl\endcsname{\def\PY@tc##1{\textcolor[rgb]{0.63,0.63,0.00}{##1}}}
\expandafter\def\csname PY@tok@nn\endcsname{\let\PY@bf=\textbf\def\PY@tc##1{\textcolor[rgb]{0.00,0.00,1.00}{##1}}}
\expandafter\def\csname PY@tok@no\endcsname{\def\PY@tc##1{\textcolor[rgb]{0.53,0.00,0.00}{##1}}}
\expandafter\def\csname PY@tok@na\endcsname{\def\PY@tc##1{\textcolor[rgb]{0.49,0.56,0.16}{##1}}}
\expandafter\def\csname PY@tok@nb\endcsname{\def\PY@tc##1{\textcolor[rgb]{0.00,0.50,0.00}{##1}}}
\expandafter\def\csname PY@tok@nc\endcsname{\let\PY@bf=\textbf\def\PY@tc##1{\textcolor[rgb]{0.00,0.00,1.00}{##1}}}
\expandafter\def\csname PY@tok@nd\endcsname{\def\PY@tc##1{\textcolor[rgb]{0.67,0.13,1.00}{##1}}}
\expandafter\def\csname PY@tok@ne\endcsname{\let\PY@bf=\textbf\def\PY@tc##1{\textcolor[rgb]{0.82,0.25,0.23}{##1}}}
\expandafter\def\csname PY@tok@nf\endcsname{\def\PY@tc##1{\textcolor[rgb]{0.00,0.00,1.00}{##1}}}
\expandafter\def\csname PY@tok@si\endcsname{\let\PY@bf=\textbf\def\PY@tc##1{\textcolor[rgb]{0.73,0.40,0.53}{##1}}}
\expandafter\def\csname PY@tok@s2\endcsname{\def\PY@tc##1{\textcolor[rgb]{0.73,0.13,0.13}{##1}}}
\expandafter\def\csname PY@tok@vi\endcsname{\def\PY@tc##1{\textcolor[rgb]{0.10,0.09,0.49}{##1}}}
\expandafter\def\csname PY@tok@nt\endcsname{\let\PY@bf=\textbf\def\PY@tc##1{\textcolor[rgb]{0.00,0.50,0.00}{##1}}}
\expandafter\def\csname PY@tok@nv\endcsname{\def\PY@tc##1{\textcolor[rgb]{0.10,0.09,0.49}{##1}}}
\expandafter\def\csname PY@tok@s1\endcsname{\def\PY@tc##1{\textcolor[rgb]{0.73,0.13,0.13}{##1}}}
\expandafter\def\csname PY@tok@kd\endcsname{\let\PY@bf=\textbf\def\PY@tc##1{\textcolor[rgb]{0.00,0.50,0.00}{##1}}}
\expandafter\def\csname PY@tok@sh\endcsname{\def\PY@tc##1{\textcolor[rgb]{0.73,0.13,0.13}{##1}}}
\expandafter\def\csname PY@tok@sc\endcsname{\def\PY@tc##1{\textcolor[rgb]{0.73,0.13,0.13}{##1}}}
\expandafter\def\csname PY@tok@sx\endcsname{\def\PY@tc##1{\textcolor[rgb]{0.00,0.50,0.00}{##1}}}
\expandafter\def\csname PY@tok@bp\endcsname{\def\PY@tc##1{\textcolor[rgb]{0.00,0.50,0.00}{##1}}}
\expandafter\def\csname PY@tok@c1\endcsname{\let\PY@it=\textit\def\PY@tc##1{\textcolor[rgb]{0.25,0.50,0.50}{##1}}}
\expandafter\def\csname PY@tok@kc\endcsname{\let\PY@bf=\textbf\def\PY@tc##1{\textcolor[rgb]{0.00,0.50,0.00}{##1}}}
\expandafter\def\csname PY@tok@c\endcsname{\let\PY@it=\textit\def\PY@tc##1{\textcolor[rgb]{0.25,0.50,0.50}{##1}}}
\expandafter\def\csname PY@tok@mf\endcsname{\def\PY@tc##1{\textcolor[rgb]{0.40,0.40,0.40}{##1}}}
\expandafter\def\csname PY@tok@err\endcsname{\def\PY@bc##1{\setlength{\fboxsep}{0pt}\fcolorbox[rgb]{1.00,0.00,0.00}{1,1,1}{\strut ##1}}}
\expandafter\def\csname PY@tok@mb\endcsname{\def\PY@tc##1{\textcolor[rgb]{0.40,0.40,0.40}{##1}}}
\expandafter\def\csname PY@tok@ss\endcsname{\def\PY@tc##1{\textcolor[rgb]{0.10,0.09,0.49}{##1}}}
\expandafter\def\csname PY@tok@sr\endcsname{\def\PY@tc##1{\textcolor[rgb]{0.73,0.40,0.53}{##1}}}
\expandafter\def\csname PY@tok@mo\endcsname{\def\PY@tc##1{\textcolor[rgb]{0.40,0.40,0.40}{##1}}}
\expandafter\def\csname PY@tok@kn\endcsname{\let\PY@bf=\textbf\def\PY@tc##1{\textcolor[rgb]{0.00,0.50,0.00}{##1}}}
\expandafter\def\csname PY@tok@mi\endcsname{\def\PY@tc##1{\textcolor[rgb]{0.40,0.40,0.40}{##1}}}
\expandafter\def\csname PY@tok@gp\endcsname{\let\PY@bf=\textbf\def\PY@tc##1{\textcolor[rgb]{0.00,0.00,0.50}{##1}}}
\expandafter\def\csname PY@tok@o\endcsname{\def\PY@tc##1{\textcolor[rgb]{0.40,0.40,0.40}{##1}}}
\expandafter\def\csname PY@tok@kr\endcsname{\let\PY@bf=\textbf\def\PY@tc##1{\textcolor[rgb]{0.00,0.50,0.00}{##1}}}
\expandafter\def\csname PY@tok@s\endcsname{\def\PY@tc##1{\textcolor[rgb]{0.73,0.13,0.13}{##1}}}
\expandafter\def\csname PY@tok@kp\endcsname{\def\PY@tc##1{\textcolor[rgb]{0.00,0.50,0.00}{##1}}}
\expandafter\def\csname PY@tok@w\endcsname{\def\PY@tc##1{\textcolor[rgb]{0.73,0.73,0.73}{##1}}}
\expandafter\def\csname PY@tok@kt\endcsname{\def\PY@tc##1{\textcolor[rgb]{0.69,0.00,0.25}{##1}}}
\expandafter\def\csname PY@tok@ow\endcsname{\let\PY@bf=\textbf\def\PY@tc##1{\textcolor[rgb]{0.67,0.13,1.00}{##1}}}
\expandafter\def\csname PY@tok@sb\endcsname{\def\PY@tc##1{\textcolor[rgb]{0.73,0.13,0.13}{##1}}}
\expandafter\def\csname PY@tok@k\endcsname{\let\PY@bf=\textbf\def\PY@tc##1{\textcolor[rgb]{0.00,0.50,0.00}{##1}}}
\expandafter\def\csname PY@tok@se\endcsname{\let\PY@bf=\textbf\def\PY@tc##1{\textcolor[rgb]{0.73,0.40,0.13}{##1}}}
\expandafter\def\csname PY@tok@sd\endcsname{\let\PY@it=\textit\def\PY@tc##1{\textcolor[rgb]{0.73,0.13,0.13}{##1}}}

\def\PYZbs{\char`\\}
\def\PYZus{\char`\_}
\def\PYZob{\char`\{}
\def\PYZcb{\char`\}}
\def\PYZca{\char`\^}
\def\PYZam{\char`\&}
\def\PYZlt{\char`\<}
\def\PYZgt{\char`\>}
\def\PYZsh{\char`\#}
\def\PYZpc{\char`\%}
\def\PYZdl{\char`\$}
\def\PYZhy{\char`\-}
\def\PYZsq{\char`\'}
\def\PYZdq{\char`\"}
\def\PYZti{\char`\~}
% for compatibility with earlier versions
\def\PYZat{@}
\def\PYZlb{[}
\def\PYZrb{]}
\makeatother


    % Exact colors from NB
    \definecolor{incolor}{rgb}{0.0, 0.0, 0.5}
    \definecolor{outcolor}{rgb}{0.545, 0.0, 0.0}



    
    % Prevent overflowing lines due to hard-to-break entities
    \sloppy 
    % Setup hyperref package
    \hypersetup{
      breaklinks=true,  % so long urls are correctly broken across lines
      colorlinks=true,
      urlcolor=blue,
      linkcolor=darkorange,
      citecolor=darkgreen,
      }
    % Slightly bigger margins than the latex defaults
    
    \geometry{verbose,tmargin=1in,bmargin=1in,lmargin=1in,rmargin=1in}
    
    \author{Roberto Cadena Vega}

    \begin{document}
    
    
    \maketitle
    
    

    

    \section*{Simulación de un sistema con retardo por medio de la construcción de una
Matriz de Lyapunov}


    Dado el sistema:

\[
\dot{x}(t) = A_0 x(t) + A_1 x(t - \tau)
\]

podemos obtener la solución \(x(t)\), por medio de la construcción de la
matriz de Lyapunov y el calculo de la siguiente ecuación matricial:

\[
\begin{pmatrix}
\bar{Y}(\tau) \\
\bar{Z}(\tau)
\end{pmatrix} = e^{L \tau} \left( M + N e^{L \tau} \right)^{-1}
\begin{pmatrix}
\bar{0} \\
\bar{W}
\end{pmatrix}
\]

en donde \(\bar{W}\) es la vectorización de la matriz \(W\), y las
matrices \(L\), \(M\) y \(N\) son de la forma:

\[
L =
\begin{pmatrix}
A_0^T \otimes I & A_1^T \otimes I \\
-I \otimes A_1^T & -I \otimes A_0^T
\end{pmatrix}
\]

\[
M =
\begin{pmatrix}
I \otimes I & 0 \\
A_0^T \otimes I + I \otimes A_0^T & A_1^T \otimes I
\end{pmatrix}
\]

\[
N =
\begin{pmatrix}
0 & -I \otimes I \\
I \otimes A_1^T & 0
\end{pmatrix}
\]

en donde \(A \otimes B\) es el producto de Kronecker de \(A\) y \(B\).

Para esta simulación utilizaremos los siguientes valores:

\[
A_0 =
\begin{pmatrix}
0 & 1 \\
-1 & 0
\end{pmatrix} \quad
A_1 =
\begin{pmatrix}
-2 & 0 \\
0.3 & 0
\end{pmatrix} \quad W = I \quad \tau \in [0, 1]
\]

Declaramos estas matrices, asi como una matriz de ceros, la identidad y
las vectorizaciones de una matriz de ceros, asi como la de la matriz
\(W\).

    \begin{Verbatim}[commandchars=\\\{\}]
{\color{incolor}In [{\color{incolor}1}]:} \PY{k+kn}{from} \PY{n+nn}{numpy} \PY{k+kn}{import} \PY{n}{array}\PY{p}{,} \PY{n}{matrix}\PY{p}{,} \PY{n}{kron}\PY{p}{,} \PY{n}{eye}\PY{p}{,} \PY{n}{zeros}\PY{p}{,} \PY{n}{vstack}\PY{p}{,} \PY{n}{hstack}\PY{p}{,} \PY{n}{linspace}
        \PY{k+kn}{from} \PY{n+nn}{scipy.linalg} \PY{k+kn}{import} \PY{n}{expm}
\end{Verbatim}

    \begin{Verbatim}[commandchars=\\\{\}]
{\color{incolor}In [{\color{incolor}2}]:} \PY{n}{A0} \PY{o}{=} \PY{n}{matrix}\PY{p}{(}\PY{p}{[}\PY{p}{[}\PY{l+m+mi}{0}\PY{p}{,} \PY{l+m+mi}{1}\PY{p}{]}\PY{p}{,} \PY{p}{[}\PY{o}{\PYZhy{}}\PY{l+m+mi}{1}\PY{p}{,} \PY{l+m+mi}{0}\PY{p}{]}\PY{p}{]}\PY{p}{)}
        \PY{n}{A0}
\end{Verbatim}

            \begin{Verbatim}[commandchars=\\\{\}]
{\color{outcolor}Out[{\color{outcolor}2}]:} matrix([[ 0,  1],
                [-1,  0]])
\end{Verbatim}
        
    \begin{Verbatim}[commandchars=\\\{\}]
{\color{incolor}In [{\color{incolor}3}]:} \PY{n}{A1} \PY{o}{=} \PY{n}{matrix}\PY{p}{(}\PY{p}{[}\PY{p}{[}\PY{o}{\PYZhy{}}\PY{l+m+mi}{2}\PY{p}{,} \PY{l+m+mi}{0}\PY{p}{]}\PY{p}{,} \PY{p}{[}\PY{l+m+mf}{0.3}\PY{p}{,} \PY{l+m+mi}{0}\PY{p}{]}\PY{p}{]}\PY{p}{)}
        \PY{n}{A1}
\end{Verbatim}

            \begin{Verbatim}[commandchars=\\\{\}]
{\color{outcolor}Out[{\color{outcolor}3}]:} matrix([[-2. ,  0. ],
                [ 0.3,  0. ]])
\end{Verbatim}
        
    \begin{Verbatim}[commandchars=\\\{\}]
{\color{incolor}In [{\color{incolor}4}]:} \PY{n}{I} \PY{o}{=} \PY{n}{matrix}\PY{p}{(}\PY{n}{eye}\PY{p}{(}\PY{l+m+mi}{2}\PY{p}{)}\PY{p}{)}
        \PY{n}{I}
\end{Verbatim}

            \begin{Verbatim}[commandchars=\\\{\}]
{\color{outcolor}Out[{\color{outcolor}4}]:} matrix([[ 1.,  0.],
                [ 0.,  1.]])
\end{Verbatim}
        
    \begin{Verbatim}[commandchars=\\\{\}]
{\color{incolor}In [{\color{incolor}5}]:} \PY{n}{cero} \PY{o}{=} \PY{n}{zeros}\PY{p}{(}\PY{p}{(}\PY{l+m+mi}{2}\PY{o}{*}\PY{o}{*}\PY{l+m+mi}{2}\PY{p}{,} \PY{l+m+mi}{2}\PY{o}{*}\PY{o}{*}\PY{l+m+mi}{2}\PY{p}{)}\PY{p}{)}
        \PY{n}{cero}
\end{Verbatim}

            \begin{Verbatim}[commandchars=\\\{\}]
{\color{outcolor}Out[{\color{outcolor}5}]:} array([[ 0.,  0.,  0.,  0.],
               [ 0.,  0.,  0.,  0.],
               [ 0.,  0.,  0.,  0.],
               [ 0.,  0.,  0.,  0.]])
\end{Verbatim}
        
    \begin{Verbatim}[commandchars=\\\{\}]
{\color{incolor}In [{\color{incolor}6}]:} \PY{n}{v0} \PY{o}{=} \PY{n}{matrix}\PY{p}{(}\PY{n}{zeros}\PY{p}{(}\PY{l+m+mi}{4}\PY{p}{)}\PY{p}{)}
        \PY{n}{v0}
\end{Verbatim}

            \begin{Verbatim}[commandchars=\\\{\}]
{\color{outcolor}Out[{\color{outcolor}6}]:} matrix([[ 0.,  0.,  0.,  0.]])
\end{Verbatim}
        
    \begin{Verbatim}[commandchars=\\\{\}]
{\color{incolor}In [{\color{incolor}7}]:} \PY{n}{W} \PY{o}{=} \PY{n}{I}
        \PY{n}{W}\PY{o}{.}\PY{n}{flatten}\PY{p}{(}\PY{l+s}{\PYZdq{}}\PY{l+s}{F}\PY{l+s}{\PYZdq{}}\PY{p}{)}
\end{Verbatim}

            \begin{Verbatim}[commandchars=\\\{\}]
{\color{outcolor}Out[{\color{outcolor}7}]:} matrix([[ 1.,  0.,  0.,  1.]])
\end{Verbatim}
        
    Ahora calculamos la matriz de Lyapunov, asi como las matrices \(M\) y
\(N\)

    \begin{Verbatim}[commandchars=\\\{\}]
{\color{incolor}In [{\color{incolor}8}]:} \PY{n}{L} \PY{o}{=} \PY{n}{vstack}\PY{p}{(}\PY{p}{(}\PY{n}{hstack}\PY{p}{(}\PY{p}{(}\PY{n}{kron}\PY{p}{(}\PY{n}{A0}\PY{o}{.}\PY{n}{T}\PY{p}{,} \PY{n}{I}\PY{p}{)}\PY{p}{,} \PY{n}{kron}\PY{p}{(}\PY{n}{A1}\PY{o}{.}\PY{n}{T}\PY{p}{,} \PY{n}{I}\PY{p}{)}\PY{p}{)}\PY{p}{)}\PY{p}{,}
                \PY{n}{hstack}\PY{p}{(}\PY{p}{(}\PY{n}{kron}\PY{p}{(}\PY{o}{\PYZhy{}}\PY{n}{I}\PY{p}{,} \PY{n}{A1}\PY{o}{.}\PY{n}{T}\PY{p}{)}\PY{p}{,} \PY{n}{kron}\PY{p}{(}\PY{o}{\PYZhy{}}\PY{n}{I}\PY{p}{,} \PY{n}{A0}\PY{o}{.}\PY{n}{T}\PY{p}{)}\PY{p}{)}\PY{p}{)}\PY{p}{)}\PY{p}{)}
        \PY{n}{L}
\end{Verbatim}

            \begin{Verbatim}[commandchars=\\\{\}]
{\color{outcolor}Out[{\color{outcolor}8}]:} matrix([[ 0. ,  0. , -1. , -0. , -2. , -0. ,  0.3,  0. ],
                [ 0. ,  0. , -0. , -1. , -0. , -2. ,  0. ,  0.3],
                [ 1. ,  0. ,  0. ,  0. ,  0. ,  0. ,  0. ,  0. ],
                [ 0. ,  1. ,  0. ,  0. ,  0. ,  0. ,  0. ,  0. ],
                [ 2. , -0.3,  0. , -0. , -0. ,  1. , -0. ,  0. ],
                [-0. , -0. , -0. , -0. , -1. , -0. , -0. , -0. ],
                [ 0. , -0. ,  2. , -0.3, -0. ,  0. , -0. ,  1. ],
                [-0. , -0. , -0. , -0. , -0. , -0. , -1. , -0. ]])
\end{Verbatim}
        
    \begin{Verbatim}[commandchars=\\\{\}]
{\color{incolor}In [{\color{incolor}9}]:} \PY{n}{M} \PY{o}{=} \PY{n}{vstack}\PY{p}{(}\PY{p}{(}\PY{n}{hstack}\PY{p}{(}\PY{p}{(}\PY{n}{kron}\PY{p}{(}\PY{n}{I}\PY{p}{,} \PY{n}{I}\PY{p}{)}\PY{p}{,} \PY{n}{cero}\PY{p}{)}\PY{p}{)}\PY{p}{,}
                    \PY{n}{hstack}\PY{p}{(}\PY{p}{(}\PY{n}{kron}\PY{p}{(}\PY{n}{A0}\PY{o}{.}\PY{n}{T}\PY{p}{,} \PY{n}{I}\PY{p}{)} \PY{o}{+} \PY{n}{kron}\PY{p}{(}\PY{n}{I}\PY{p}{,} \PY{n}{A0}\PY{o}{.}\PY{n}{T}\PY{p}{)}\PY{p}{,} \PY{n}{kron}\PY{p}{(}\PY{n}{A1}\PY{o}{.}\PY{n}{T}\PY{p}{,} \PY{n}{I}\PY{p}{)}\PY{p}{)}\PY{p}{)}\PY{p}{)}\PY{p}{)}
        \PY{n}{M}
\end{Verbatim}

            \begin{Verbatim}[commandchars=\\\{\}]
{\color{outcolor}Out[{\color{outcolor}9}]:} matrix([[ 1. ,  0. ,  0. ,  0. ,  0. ,  0. ,  0. ,  0. ],
                [ 0. ,  1. ,  0. ,  0. ,  0. ,  0. ,  0. ,  0. ],
                [ 0. ,  0. ,  1. ,  0. ,  0. ,  0. ,  0. ,  0. ],
                [ 0. ,  0. ,  0. ,  1. ,  0. ,  0. ,  0. ,  0. ],
                [ 0. , -1. , -1. , -0. , -2. , -0. ,  0.3,  0. ],
                [ 1. ,  0. ,  0. , -1. , -0. , -2. ,  0. ,  0.3],
                [ 1. ,  0. ,  0. , -1. ,  0. ,  0. ,  0. ,  0. ],
                [ 0. ,  1. ,  1. ,  0. ,  0. ,  0. ,  0. ,  0. ]])
\end{Verbatim}
        
    \begin{Verbatim}[commandchars=\\\{\}]
{\color{incolor}In [{\color{incolor}10}]:} \PY{n}{N} \PY{o}{=} \PY{n}{vstack}\PY{p}{(}\PY{p}{(}\PY{n}{hstack}\PY{p}{(}\PY{p}{(}\PY{n}{cero}\PY{p}{,} \PY{n}{kron}\PY{p}{(}\PY{o}{\PYZhy{}}\PY{n}{I}\PY{p}{,} \PY{n}{I}\PY{p}{)}\PY{p}{)}\PY{p}{)}\PY{p}{,}
                     \PY{n}{hstack}\PY{p}{(}\PY{p}{(}\PY{n}{kron}\PY{p}{(}\PY{n}{I}\PY{p}{,} \PY{n}{A1}\PY{o}{.}\PY{n}{T}\PY{p}{)}\PY{p}{,} \PY{n}{cero}\PY{p}{)}\PY{p}{)}\PY{p}{)}\PY{p}{)}
         \PY{n}{N}
\end{Verbatim}

            \begin{Verbatim}[commandchars=\\\{\}]
{\color{outcolor}Out[{\color{outcolor}10}]:} matrix([[ 0. ,  0. ,  0. ,  0. , -1. , -0. , -0. , -0. ],
                 [ 0. ,  0. ,  0. ,  0. , -0. , -1. , -0. , -0. ],
                 [ 0. ,  0. ,  0. ,  0. , -0. , -0. , -1. , -0. ],
                 [ 0. ,  0. ,  0. ,  0. , -0. , -0. , -0. , -1. ],
                 [-2. ,  0.3, -0. ,  0. ,  0. ,  0. ,  0. ,  0. ],
                 [ 0. ,  0. ,  0. ,  0. ,  0. ,  0. ,  0. ,  0. ],
                 [-0. ,  0. , -2. ,  0.3,  0. ,  0. ,  0. ,  0. ],
                 [ 0. ,  0. ,  0. ,  0. ,  0. ,  0. ,  0. ,  0. ]])
\end{Verbatim}
        
    \begin{Verbatim}[commandchars=\\\{\}]
{\color{incolor}In [{\color{incolor}11}]:} \PY{n}{v} \PY{o}{=} \PY{n}{vstack}\PY{p}{(}\PY{p}{(}\PY{n}{v0}\PY{o}{.}\PY{n}{T}\PY{p}{,} \PY{n}{W}\PY{o}{.}\PY{n}{flatten}\PY{p}{(}\PY{l+s}{\PYZdq{}}\PY{l+s}{F}\PY{l+s}{\PYZdq{}}\PY{p}{)}\PY{o}{.}\PY{n}{T}\PY{p}{)}\PY{p}{)}
         \PY{n}{v}
\end{Verbatim}

            \begin{Verbatim}[commandchars=\\\{\}]
{\color{outcolor}Out[{\color{outcolor}11}]:} matrix([[ 0.],
                 [ 0.],
                 [ 0.],
                 [ 0.],
                 [ 1.],
                 [ 0.],
                 [ 0.],
                 [ 1.]])
\end{Verbatim}
        
    Ahora declaramos la función \(f(\tau)\) la cual es de la forma:

\[
f(\tau) := e^{L \tau} \left( M + N e^{L \tau} \right)^{-1}
\begin{pmatrix}
\bar{0} \\
\bar{W}
\end{pmatrix}
\]

con la que podremos calcular los valores de:

\[
\begin{pmatrix}
\bar{Y}(\tau) \\
\bar{Z}(\tau)
\end{pmatrix}
\]

para cada \(\tau\) que le demos como argumento:

    \begin{Verbatim}[commandchars=\\\{\}]
{\color{incolor}In [{\color{incolor}12}]:} \PY{n}{f} \PY{o}{=} \PY{k}{lambda} \PY{n}{tau}\PY{p}{:} \PY{p}{(}\PY{n}{expm}\PY{p}{(}\PY{n}{L}\PY{o}{*}\PY{n}{tau}\PY{p}{)}\PY{o}{*}\PY{p}{(}\PY{n}{M} \PY{o}{+} \PY{n}{N}\PY{o}{*}\PY{n}{expm}\PY{p}{(}\PY{n}{L}\PY{o}{*}\PY{n}{tau}\PY{p}{)}\PY{p}{)}\PY{o}{.}\PY{n}{I}\PY{o}{*}\PY{n}{v}\PY{p}{)}\PY{o}{.}\PY{n}{flatten}\PY{p}{(}\PY{p}{)}\PY{o}{.}\PY{n}{tolist}\PY{p}{(}\PY{p}{)}\PY{p}{[}\PY{l+m+mi}{0}\PY{p}{]}
\end{Verbatim}

    utilizamos \(h = \frac{1}{2}\) para probar nuestra función:

    \begin{Verbatim}[commandchars=\\\{\}]
{\color{incolor}In [{\color{incolor}13}]:} \PY{c}{\PYZsh{} Fijamos la precision de los datos mostrados a 3 cifras significativas}
         \PY{c}{\PYZsh{} para facilitar la lectura}
         
         \PY{o}{\PYZpc{}}\PY{k}{precision} \PY{l+m+mi}{3}
\end{Verbatim}

            \begin{Verbatim}[commandchars=\\\{\}]
{\color{outcolor}Out[{\color{outcolor}13}]:} u'\%.3f'
\end{Verbatim}
        
    \begin{Verbatim}[commandchars=\\\{\}]
{\color{incolor}In [{\color{incolor}14}]:} \PY{n}{h} \PY{o}{=} \PY{l+m+mf}{0.5}
         \PY{n}{v1} \PY{o}{=} \PY{n}{f}\PY{p}{(}\PY{n}{h}\PY{p}{)}
         \PY{n}{v1}
\end{Verbatim}

            \begin{Verbatim}[commandchars=\\\{\}]
{\color{outcolor}Out[{\color{outcolor}14}]:} [-0.373, 0.844, 0.034, -1.434, -1.303, 0.500, 0.500, -1.800]
\end{Verbatim}
        
    y para un incremento en el retardo de \(\frac{1}{10}\), tenemos:

    \begin{Verbatim}[commandchars=\\\{\}]
{\color{incolor}In [{\color{incolor}15}]:} \PY{n}{v2} \PY{o}{=} \PY{n}{f}\PY{p}{(}\PY{n}{h} \PY{o}{+} \PY{l+m+mf}{0.1}\PY{p}{)}
         \PY{n}{v2}
\end{Verbatim}

            \begin{Verbatim}[commandchars=\\\{\}]
{\color{outcolor}Out[{\color{outcolor}15}]:} [-0.280, 1.465, -0.639, -1.317, -2.888, 0.500, 0.500, -2.005]
\end{Verbatim}
        
    por lo que solo queda declarar el dominio de definición de los retardos:

\[
T = [0, 1]
\]

y obtener el conjunto de soluciones para cada retardo en el dominio
definido:

\[
V_{sol}= \left\{ f(t) \mid t \in T \right\} 
\]

    \begin{Verbatim}[commandchars=\\\{\}]
{\color{incolor}In [{\color{incolor}16}]:} \PY{n}{T} \PY{o}{=} \PY{n}{linspace}\PY{p}{(}\PY{l+m+mi}{0}\PY{p}{,} \PY{l+m+mi}{1}\PY{p}{,} \PY{l+m+mi}{1000}\PY{p}{)}
         \PY{n}{vsol} \PY{o}{=} \PY{p}{[}\PY{n}{f}\PY{p}{(}\PY{n}{t}\PY{p}{)} \PY{k}{for} \PY{n}{t} \PY{o+ow}{in} \PY{n}{T}\PY{p}{]}
\end{Verbatim}

    \begin{Verbatim}[commandchars=\\\{\}]
{\color{incolor}In [{\color{incolor}17}]:} \PY{n}{sol} \PY{o}{=} \PY{n+nb}{zip}\PY{p}{(}\PY{o}{*}\PY{n}{vsol}\PY{p}{)}
\end{Verbatim}

    Ahora solo queda graficar la soluciónes contra los retardos del sistema.

    \begin{Verbatim}[commandchars=\\\{\}]
{\color{incolor}In [{\color{incolor}18}]:} \PY{o}{\PYZpc{}}\PY{k}{matplotlib} \PY{n}{inline}
         \PY{k+kn}{from} \PY{n+nn}{matplotlib.pyplot} \PY{k+kn}{import} \PY{n}{plot}\PY{p}{,} \PY{n}{figure}\PY{p}{,} \PY{n}{style}\PY{p}{,} \PY{n}{legend}
         \PY{n}{style}\PY{o}{.}\PY{n}{use}\PY{p}{(}\PY{l+s}{\PYZdq{}}\PY{l+s}{ggplot}\PY{l+s}{\PYZdq{}}\PY{p}{)}
\end{Verbatim}

    \begin{Verbatim}[commandchars=\\\{\}]
{\color{incolor}In [{\color{incolor}19}]:} \PY{n}{fig} \PY{o}{=} \PY{n}{figure}\PY{p}{(}\PY{n}{figsize}\PY{o}{=}\PY{p}{(}\PY{l+m+mi}{12}\PY{p}{,} \PY{l+m+mi}{6}\PY{p}{)}\PY{p}{)}
         
         \PY{n}{p1}\PY{p}{,} \PY{o}{=} \PY{n}{plot}\PY{p}{(}\PY{n}{T}\PY{p}{,} \PY{n}{sol}\PY{p}{[}\PY{l+m+mi}{0}\PY{p}{]}\PY{p}{)}
         \PY{n}{p2}\PY{p}{,} \PY{o}{=} \PY{n}{plot}\PY{p}{(}\PY{n}{T}\PY{p}{,} \PY{n}{sol}\PY{p}{[}\PY{l+m+mi}{1}\PY{p}{]}\PY{p}{)}
         \PY{n}{p3}\PY{p}{,} \PY{o}{=} \PY{n}{plot}\PY{p}{(}\PY{n}{T}\PY{p}{,} \PY{n}{sol}\PY{p}{[}\PY{l+m+mi}{2}\PY{p}{]}\PY{p}{)}
         \PY{n}{p4}\PY{p}{,} \PY{o}{=} \PY{n}{plot}\PY{p}{(}\PY{n}{T}\PY{p}{,} \PY{n}{sol}\PY{p}{[}\PY{l+m+mi}{3}\PY{p}{]}\PY{p}{)}
         
         \PY{n}{ax} \PY{o}{=} \PY{n}{fig}\PY{o}{.}\PY{n}{gca}\PY{p}{(}\PY{p}{)}
         \PY{n}{ax}\PY{o}{.}\PY{n}{set\PYZus{}ylabel}\PY{p}{(}\PY{l+s}{r\PYZdq{}}\PY{l+s}{\PYZdl{}Y(}\PY{l+s}{\PYZbs{}}\PY{l+s}{tau)\PYZdl{}}\PY{l+s}{\PYZdq{}}\PY{p}{,} \PY{n}{fontsize}\PY{o}{=}\PY{l+m+mi}{20}\PY{p}{)}
         \PY{n}{ax}\PY{o}{.}\PY{n}{set\PYZus{}xlabel}\PY{p}{(}\PY{l+s}{r\PYZdq{}}\PY{l+s}{\PYZdl{}}\PY{l+s}{\PYZbs{}}\PY{l+s}{tau\PYZdl{}}\PY{l+s}{\PYZdq{}}\PY{p}{,} \PY{n}{fontsize}\PY{o}{=}\PY{l+m+mi}{20}\PY{p}{)}
         
         \PY{n}{legend}\PY{p}{(}\PY{p}{[}\PY{n}{p1}\PY{p}{,} \PY{n}{p2}\PY{p}{,} \PY{n}{p3}\PY{p}{,} \PY{n}{p4}\PY{p}{]}\PY{p}{,}
                \PY{p}{[}\PY{l+s}{r\PYZdq{}}\PY{l+s}{\PYZdl{}y\PYZus{}\PYZob{}11\PYZcb{}(}\PY{l+s}{\PYZbs{}}\PY{l+s}{tau)\PYZdl{}}\PY{l+s}{\PYZdq{}}\PY{p}{,}
                 \PY{l+s}{r\PYZdq{}}\PY{l+s}{\PYZdl{}y\PYZus{}\PYZob{}21\PYZcb{}(}\PY{l+s}{\PYZbs{}}\PY{l+s}{tau)\PYZdl{}}\PY{l+s}{\PYZdq{}}\PY{p}{,}
                 \PY{l+s}{r\PYZdq{}}\PY{l+s}{\PYZdl{}y\PYZus{}\PYZob{}12\PYZcb{}(}\PY{l+s}{\PYZbs{}}\PY{l+s}{tau)\PYZdl{}}\PY{l+s}{\PYZdq{}}\PY{p}{,}
                 \PY{l+s}{r\PYZdq{}}\PY{l+s}{\PYZdl{}y\PYZus{}\PYZob{}22\PYZcb{}(}\PY{l+s}{\PYZbs{}}\PY{l+s}{tau)\PYZdl{}}\PY{l+s}{\PYZdq{}}\PY{p}{]}\PY{p}{)}\PY{p}{;}
\end{Verbatim}

    \begin{center}
    \adjustimage{max size={0.9\linewidth}{0.9\paperheight}}{Tarea 6_files/Tarea 6_27_0.png}
    \end{center}
    { \hspace*{\fill} \\}
    
    Puedes acceder a este notebook a traves de la página

http://bit.ly/1CccnJ9

o escaneando el siguiente código:

\begin{figure*}[htbp]
\centering
\includegraphics[width=0.2\textwidth]{codigos/codigo6.jpg}
\end{figure*}
        

    % Add a bibliography block to the postdoc
    
    
    
    \end{document}
