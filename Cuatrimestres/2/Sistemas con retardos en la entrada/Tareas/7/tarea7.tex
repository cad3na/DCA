\documentclass{article}
\usepackage{graphicx} % Used to insert images
\usepackage{enumerate} % Needed for markdown enumerations to work
\usepackage{geometry} % Used to adjust the document margins
\usepackage{amsmath} % Equations
\usepackage{amssymb} % Equations
\usepackage[mathletters]{ucs} % Extended unicode (utf-8) support
\usepackage[utf8x]{inputenc} % Allow utf-8 characters in the tex document
\usepackage{fancyvrb} % verbatim replacement that allows latex
\usepackage{grffile} % extends the file name processing of package graphics 
                     % to support a larger range 
% The hyperref package gives us a pdf with properly built
% internal navigation ('pdf bookmarks' for the table of contents,
% internal cross-reference links, web links for URLs, etc.)
\usepackage{hyperref}
\usepackage{longtable} % longtable support required by pandoc >1.10
\usepackage{booktabs}  % table support for pandoc > 1.12.2
\usepackage{mathpazo}
\usepackage[spanish]{babel}

\geometry{verbose,tmargin=1in,bmargin=1in,lmargin=1in,rmargin=1in}

\author{Roberto Cadena Vega}
\title{Tarea 7 - Sistemas con retardos en la entrada}

\begin{document}
    \maketitle

    \section*{Tarea 7 - Desarrollo de resultados clásicos en sistemas con retardo en la entrada.}

        \subsection*{Función de transferencia de predictor de Smith\cite{Abe2003}}

        Dado un sistema con retardo $G(s) e^{-sh}$, en donde $G(s)$ es la parte del sistema sin el retardo $h$, tenemos el siguiente esquema de control:

        \begin{figure}[h]
            \centering
            \resizebox{\textwidth}{!}{
                \tikzstyle{block} = [draw, rectangle, minimum height=3em, minimum width=5em]

                \begin{tikzpicture}[auto, node distance=3.25cm, >=latex']
                    \node [input, name=pert] {};
                    \node [input, name=entrada] {};
                    \node [sum, right of=entrada] (suma1) {$+$};
                    \node [sum, right of=suma1] (suma2) {$+$};
                    \node [block, right of=suma2] (control) {$C(s)$};
                    \node [sum, right of=control, pin={[init]above:$\delta$}] (suma3) {$+$};
                    \node [block, right of=suma3] (sistema) {$G(s) e^{-sh}$};
                    \node [output, right of=sistema] (salida) {};
                    \node [block, below of=sistema] (retro1) {$-1$};
                    \node [block, above of=control] (retro2) {$\hat{G}(s) \left(1 - e^{-s\hat{h}} \right)$};

                    \draw [->] (entrada) -- node[name=x] {$r(s)$} (suma1);
                    \draw [->] (suma1) -- node[name=e1] {$e(s)$} (suma2);
                    \draw [->] (suma2) -- node[name=e2] {} (control);
                    \draw [->] (control) -- node[name=u1] {} (suma3);
                    \draw [->] (suma3) -- node[name=u2] {$u(s)$} (sistema);
                    \draw [->] (sistema) -- node[name=y] {$y(s)$} (salida);
                    \draw [->] (y) |- (retro1);
                    \draw [->] (retro1) -| (suma1);
                    \draw [->] (u1) |- (retro2);
                    \draw [->] (retro2) -| (suma2);
                \end{tikzpicture}}
        \end{figure}

        en donde $\hat{G}(s)$ es una aproximación de $G(s)$, y $\hat{h}$ es una aproximación del retardo $h$.

        Podemos obtener la función de transferencia de este sistema bajo el predictor de Smith:

        \begin{align*}
            \frac{y(s)}{r(s)} &= \frac{\frac{C(s)}{1 + C(s)\hat{G}(s)\left( 1 - e^{-s\hat{h}} \right)}G(s)e^{-sh}}{1 + \frac{C(s)}{1 + C(s)\hat{G}(s)\left( 1 - e^{-s\hat{h}} \right)}G(s)e^{-sh}} \\
            &= \frac{\frac{C(s)G(s)e^{-sh}}{1 + C(s)\hat{G}(s)\left( 1 - e^{-s\hat{h}} \right)}}{1 + \frac{C(s)G(s)e^{-sh}}{1 + C(s)\hat{G}(s)\left( 1 - e^{-s\hat{h}} \right)}} \\
            &= \frac{C(s)G(s)e^{-sh}}{1 + C(s)\hat{G}(s)\left( 1 - e^{-s\hat{h}} \right) + C(s)G(s)e^{-sh}} \\
            &= \frac{C(s)G(s)e^{-sh}}{1 + C(s) \left[ \hat{G}(s) - \hat{G}(s)e^{-s\hat{h}} + G(s)e^{-sh} \right]}
        \end{align*}

        Si ahora asumimos que la aproximación $\hat{G}(s)$ es precisa:

        \begin{equation}
            \frac{y(s)}{r(s)} = \frac{C(s)G(s)e^{-sh}}{1 + C(s) G(s) \left( 1 + e^{-sh} - e^{-s\hat{h}} \right)}
        \end{equation}

        Cabe notar que si tambien asumimos una aproximación precisa de $\hat{h}$ tendremos la ecuación que ya teniamos:

        \begin{equation}
            \frac{C(s)G(s)e^{-sh}}{1 + C(s)G(s)}
        \end{equation}

    \bibliographystyle{ieeetr}
    \bibliography{bibliografia}

\end{document}