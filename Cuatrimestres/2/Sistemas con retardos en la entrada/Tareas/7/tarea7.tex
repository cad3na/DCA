\documentclass{article}
\usepackage{graphicx} % Used to insert images
\usepackage{enumerate} % Needed for markdown enumerations to work
\usepackage{geometry} % Used to adjust the document margins
\usepackage{amsmath} % Equations
\usepackage{amssymb} % Equations
\usepackage[mathletters]{ucs} % Extended unicode (utf-8) support
\usepackage[utf8x]{inputenc} % Allow utf-8 characters in the tex document
\usepackage{fancyvrb} % verbatim replacement that allows latex
\usepackage{grffile} % extends the file name processing of package graphics 
                     % to support a larger range 
% The hyperref package gives us a pdf with properly built
% internal navigation ('pdf bookmarks' for the table of contents,
% internal cross-reference links, web links for URLs, etc.)
\usepackage{hyperref}
\usepackage{longtable} % longtable support required by pandoc >1.10
\usepackage{booktabs}  % table support for pandoc > 1.12.2
\usepackage{mathpazo}
\usepackage[spanish]{babel}

\geometry{verbose,tmargin=1in,bmargin=1in,lmargin=1in,rmargin=1in}

\author{Roberto Cadena Vega}
\title{Tarea 5 - Sistemas con retardos en la entrada}

\begin{document}
    \maketitle

    \section*{Tarea 5 - Integrales con criterios cuadraticos}

        \subsection*{Integral $\int_{-h}^0 x^T(t + \theta) e^{2 \beta \theta} x(t + \theta) d \theta$}

            Dada la integral

            \begin{equation}
                I_1 = \int_{-h}^0 x^T(t + \theta) e^{2 \beta \theta} x(t + \theta) d \theta
            \end{equation}

            obtener un cambio de variable para simplificar la obtención de la derivada y el calculo de la cota superior de esta integral.

            Primero realizamos el cambio de variable $s = t + \theta$, el cual nos implica  $ds = d \theta$ y notamos que $s$ variará desde $t - h$ hasta $t$, por lo que nuestra integral puede ser escrita de la siguiente manera:

            \begin{equation*}
                I_1 = \int_{t - h}^t x^T(s) e^{2 \beta (s - t)} x(s) ds
            \end{equation*}

            Ahora, para obtener la derivada de esta integral, utilizamos la regla de Leibnitz:

            \begin{equation*}
                \frac{d I_1}{dt} = x^T(t) e^{2 \beta (t - t)} x(t) \frac{dt}{dt} - x^T(t - h) e^{2 \beta (t - h - t)} x(t - h) \frac{d(t - h)}{dt}
            \end{equation*}

            Si ahora eliminamos los terminos irrelevantes, tendremos que:

            \begin{equation*}
                \frac{d I_1}{dt} = || x(t) ||^2 - x^T(t - h) e^{2 \beta (h)} x(t - h)
            \end{equation*}

            De aqui podemos notar que si $\beta > 0$, tenemos que el termino $e^{-2 \beta h}$ estara acotado por:

            \begin{equation*}
                1 > e^{-2  \beta h} > 0
            \end{equation*}

            por lo que en general, podemos decir que esta expresión tiene ua cota superior:

            \begin{equation}
                || x(t) ||^2 - || x^T(t - h) ||^2 e^{2 \beta (h)} < || x(t) ||^2
            \end{equation}

\end{document}