\documentclass{article}
\usepackage{graphicx} % Used to insert images
\usepackage{enumerate} % Needed for markdown enumerations to work
\usepackage{geometry} % Used to adjust the document margins
\usepackage{amsmath} % Equations
\usepackage{amssymb} % Equations
\usepackage[mathletters]{ucs} % Extended unicode (utf-8) support
\usepackage[utf8x]{inputenc} % Allow utf-8 characters in the tex document
\usepackage{fancyvrb} % verbatim replacement that allows latex
\usepackage{grffile} % extends the file name processing of package graphics 
                     % to support a larger range 
% The hyperref package gives us a pdf with properly built
% internal navigation ('pdf bookmarks' for the table of contents,
% internal cross-reference links, web links for URLs, etc.)
\usepackage{hyperref}
\usepackage{longtable} % longtable support required by pandoc >1.10
\usepackage{booktabs}  % table support for pandoc > 1.12.2
\usepackage{mathpazo}
\usepackage[spanish]{babel}

\geometry{verbose,tmargin=1in,bmargin=1in,lmargin=1in,rmargin=1in}

\author{Roberto Cadena Vega}
\title{Tarea 1 - Sistemas con retardos en la entrada}

\begin{document}
    \maketitle

    \section*{Tarea 1 - Artículos con relación a sistemas con retardos}

        \subsection*{Exponential $L_2$-stability for a class of linear systems governed by continuous-time difference equations\cite{Damak2014}}

            \paragraph{Abstract}

                We present necessary conditions for the exponential stability of linear systems with multiple delays. They are expressed in terms of the delay Lyapunov matrix of the Lyapunov–Krasovskii functionals of complete type approach. New properties of independent interest, that establish connections of the system fundamental matrix with its Lyapunov matrix, are crucial elements of our proof. We illustrate our work with a number of examples.

        \subsection*{Necessary stability conditions for linear delay systems\cite{Egorov2014}}

            \paragraph{Abstract}

                Linear systems governed by continuous-time difference equations cover a wide class of linear systems. From the Lyapunov–Krasovskii approach, we investigate $L_2$-stability for such a class of systems. Sufficient conditions, and in some particular cases, necessary and sufficient conditions for exponential $L_2$-stability are established, for multivariable systems with commensurate or rationally independent delays. An analysis of discontinuities evolution appearing in the system response is proposed. Finally, a robustness issue is discussed for time-varying delays.

        \subsection*{Passivity-preserving model reduction with finite frequency $H_{\infty}$ approximation performance\cite{Li2014}}

            \paragraph{Abstract}

                This paper is concerned with model reduction for passive systems. For a given linear time-invariant system that is stable and positive real (PR), our goal is to find a PR reduced-order model to approximate it, and our attention is focused on reducing the error with respect to a finite frequency H∞ performance, which is the most remarkable difference between the proposed approach and the existing ones. First, by applying multiplier expansion, new conditions in terms of linear matrix inequalities are derived for characterizing the positive realness of the reduced-order model and the finite frequency H∞ performance of the error system. A necessary and sufficient condition is then established for parameterizing a PR reduced-order model with finite frequency $H_{\infty}$ approximation performance, based on which, an iterative algorithm is constructed for numerically exploring such a reduced-order model. Particularly, a partial multiplier expansion treatment is introduced, which greatly reduces the decision variables but does not cause conservatism to the derived conditions. The proposed method is also extended to robust passivity-reserving model reduction with polytopic uncertainty. Finally, we provide two numerical examples about RLC circuits to show the effectiveness and advantages of the proposed model reduction method.

    \bibliographystyle{ieeetr}
    \bibliography{bibliografia}
\end{document}