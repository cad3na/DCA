\documentclass{article}
\usepackage{graphicx} % Used to insert images
\usepackage{enumerate} % Needed for markdown enumerations to work
\usepackage{geometry} % Used to adjust the document margins
\usepackage{amsmath} % Equations
\usepackage{amssymb} % Equations
\usepackage[mathletters]{ucs} % Extended unicode (utf-8) support
\usepackage[utf8x]{inputenc} % Allow utf-8 characters in the tex document
\usepackage{fancyvrb} % verbatim replacement that allows latex
\usepackage{grffile} % extends the file name processing of package graphics 
                     % to support a larger range 
% The hyperref package gives us a pdf with properly built
% internal navigation ('pdf bookmarks' for the table of contents,
% internal cross-reference links, web links for URLs, etc.)
\usepackage{hyperref}
\usepackage{longtable} % longtable support required by pandoc >1.10
\usepackage{booktabs}  % table support for pandoc > 1.12.2
\usepackage{mathpazo}
\usepackage[spanish]{babel}

\geometry{verbose,tmargin=1in,bmargin=1in,lmargin=1in,rmargin=1in}

\author{Roberto Cadena Vega}
\title{Tarea 10 - Sistemas con retardos en la entrada}

\begin{document}
    \maketitle

    \section*{Tarea 10 - Desarrollos para asignación de espectro finito.}

        \subsection*{Revisión de equivalencia de condición de estabilidad para predictor por asignación de espectro.}

        Dadas las expresiones para $x(t)$ y $u(t)$, demostrar que la estabilidad de una implica la otra.

        \begin{equation} \label{eq:eq1}
            x(t) = Bk \int_{-\tau}^0 e^{-A \delta} x(t + \delta) d\delta
        \end{equation}

        \begin{equation} \label{eq:eq2}
            u(t) = k \int_{-\tau}^0 e^{-A \theta} B u(t + \theta) d\theta
        \end{equation}

        Empezaremos con la ecuación~\ref{eq:eq1}. Para obtener la transformada de Laplace, primero haremos el caso mas general.

        \begin{equation*}
            \mathcal{L} \left\{ \int_{-\tau}^0 G(\theta) B u(t + \theta) d\theta \right\} = \int_0^{\infty} e^{-st} \int_{-\tau}^0 G(\theta) B u(t + \theta) d\theta dt
        \end{equation*}

        Si hacemos el cambio de variable $\theta = -\delta$, el diferencial será $d\theta = -d\delta$ con los limites de integración $\tau \to 0$, por lo que la integral queda:

        \begin{align*}
            \mathcal{L} \left\{ \int_{-\tau}^0 G(\theta) B u(t + \theta) d\theta \right\} &= -\int_0^{\infty} e^{-st} \int_{\tau}^0 G(-\delta) B u(t - \delta) d\delta dt \\
            &= \int_0^{\infty} e^{-st} \int_0^{\tau} G(-\delta) B u(t - \delta) d\delta dt \\
            &= \int_0^{\infty} \int_0^{\tau} e^{-st} G(-\delta) B u(t - \delta) d\delta dt \\
            &= \int_0^{\tau} \int_0^{\infty} e^{-st} G(-\delta) B u(t - \delta) dt d\delta \\
            &= \int_0^{\tau} G(-\delta) \int_0^{\infty} e^{-st} B u(t - \delta) dt d\delta \\
            &= \int_0^{\tau} G(-\delta) e^{-s\delta} B u(s) d\delta \\
            &= \int_0^{\tau} G(-\delta) e^{-s\delta} d\delta B u(s)
        \end{align*}

        Si ahora regresamos a nuestra variable original, tendremos:

        \begin{equation}
            \mathcal{L} \left\{ \int_{-\tau}^0 G(\theta) B u(t + \theta) d\theta \right\} = \int_{-\tau}^0 G(\theta) e^{s\theta} d\theta B u(s)
        \end{equation}

        Regresando a la ecuación~\ref{eq:eq1} y aplicando esta identidad, tenemos que:

        \begin{align*}
            x(t) &= Bk \int_{-\tau}^0 e^{-A \delta} x(t + \delta) d\delta \\
            x(s) &= Bk \int_{-\tau}^0 e^{-A \delta} e^{s \delta} d\delta x(s) \\
            &= Bk \int_{-\tau}^0 e^{(sI-A) \delta} d\delta x(s) \\
            &= Bk \left.(sI - A)^{-1} e^{(sI-A) \delta} \right|_{-\tau}^0 x(s) \\
            x(s) &= Bk (sI - A)^{-1} \left[ I - e^{-(sI-A) \tau} \right] x(s)
        \end{align*}

        Por lo que al despejar todo a un lado de la ecuación, tenemos que el polinomio caracteristico es:

        \begin{equation}
            \det{\left\{ I - Bk (sI - A)^{-1} \left[ I - e^{-(sI-A) \tau} \right] \right\}}
        \end{equation}

        Por otro lado, si efectuamos el mismo procedimiento a la ecuación~\ref{eq:eq2}, obtendremos:

        \begin{align*}
            u(t) &= k \int_{-\tau}^0 e^{-A \theta} B u(t + \theta) d\theta \\
            u(s) &= k \int_{-\tau}^0 e^{-A \theta} e^{s \theta} d\theta B u(s) \\
            &= k \int_{-\tau}^0 e^{(sI-A) \theta} d\theta B u(s) \\
            &= k \left.(sI-A)^{-1} e^{(sI-A) \theta} \right|_{-\tau}^0 B u(s) \\
            u(s) &= k (sI-A)^{-1} \left[ I - e^{-(sI-A) \tau} \right] B u(s)
        \end{align*}

        Por lo que al despejar todo a un lado de la ecuación, tenemos que el polinomio caracteristico es:

        \begin{equation}
            \det{\left\{ I -  k (sI-A)^{-1} \left[ I - e^{-(sI-A) \tau} \right] B \right\}}
        \end{equation}

        Y si aplicamos la identidad $\det{(I - MN)} = \det{(I - NM)}$\cite{kailath1980linear}, tendremos que este ultimo polinomio caracteristico es equivalente al primero:

        \begin{equation}
            \det{\left\{ I -  Bk (sI-A)^{-1} \left[ I - e^{-(sI-A) \tau} \right] \right\}}
        \end{equation}

        por lo que la estabilidad de una expresión, implica la de la otra.

        \subsection*{Balanceo de terminos en funcionales de tipo completo.}

        Dada la funcional asociada al problema de robustez ante la incertidumbre del retardo, tenemos que su derivada cumple:

        \begin{align}
            \frac{d v(z_t)}{dt} \leq& - \alpha_1 \left|\left| z(t) \right|\right|^2 - \alpha_2 \left|\left| z(t - h) \right|\right|^2 - \alpha_3 \int_{-h}^0 \left|\left| z(t + \theta) \right|\right|^2 d\theta \nonumber \\
            &+ \overbrace{\alpha_4 \int_{-h - \eta}^{-h} \left|\left| z(t + \theta) \right|\right|^2 + \alpha_5 \int_{-2h - 2\eta}^{-2h - \eta} \left|\left| z(t + \theta) \right|\right|^2}^{\text{terminos positivos a compensar}}
        \end{align}

        por lo que investigamos la manera de introducir terminos similares para compensarlos y obtener $\frac{d v(x_t)}{dt} \leq 0$. Propongamos una funcional de la forma:

        \begin{equation}
            \bar{v}(z_t) = \int_{\delta-\eta}^{\delta} \int_{t + \theta}^{t - h} z^T(s) R z(s) ds d\theta
        \end{equation}

        La cual tendrá una derivada de la forma:

        \begin{align*}
            \frac{d\bar{v}(z_t)}{dt} &= \int_{\delta-\eta}^{\delta} \frac{d}{dt} \int_{t + \theta}^{t - h} z^T(s) R z(s) ds d\theta \\
            &= \int_{\delta-\eta}^{\delta} z^T(t-h) R z(t-h) d\theta - \int_{\delta-\eta}^{\delta} z^T(t+\theta) R z(t+\theta) d\theta \\
            &= z^T(t-h) R z(t-h) \left[\delta - (\delta-\eta)\right] - \int_{\delta-\eta}^{\delta} z^T(t+\theta) R z(t+\theta) d\theta \\
            &= \eta z^T(t-h) R z(t-h) - \int_{\delta-\eta}^{\delta} z^T(t+\theta) R z(t+\theta) d\theta \\
            &\leq \eta \lambda_{max}(R) \left|\left| z(t - h) \right|\right|^2 - \lambda_{max}(R) \int_{\delta-\eta}^{\delta} \left|\left| z(t + \theta) \right|\right|^2 
        \end{align*}

        por lo que:

        \begin{equation*}
            \varepsilon \frac{d\bar{v}(z_t)}{dt} \leq \varepsilon \eta \lambda_{max}(R) \left|\left| z(t - h) \right|\right|^2 - \varepsilon \lambda_{max}(R) \int_{\delta-\eta}^{\delta} \left|\left| z(t + \theta) \right|\right|^2 
        \end{equation*}

        Si proponemos dos funcionales:

        \begin{align*}
            \bar{v}_4(z_t) &= \int_{-h-\eta}^{-h} \int_{t + \theta}^{t - h} z^T(s) R z(s) ds d\theta \\
            \bar{v}_5(z_t) &= \int_{-2h-2\eta}^{-2h-\eta} \int_{t + \theta}^{t - h} z^T(s) R z(s) ds d\theta
        \end{align*}
        
        tendremos que sus derivadas estarán acotadas por:

        \begin{align*}
            \varepsilon_4 \frac{d\bar{v}(z_t)}{dt} &\leq \varepsilon_4 \eta \lambda_{max}(R) \left|\left| z(t - h) \right|\right|^2 - \varepsilon_4 \lambda_{max}(R) \int_{-h-\eta}^{-h} \left|\left| z(t + \theta) \right|\right|^2 \\
            \varepsilon_5 \frac{d\bar{v}(z_t)}{dt} &\leq \varepsilon_5 \eta \lambda_{max}(R) \left|\left| z(t - h) \right|\right|^2 - \varepsilon_5 \lambda_{max}(R) \int_{-2h-2\eta}^{-2h-\eta} \left|\left| z(t + \theta) \right|\right|^2 \\
        \end{align*}

        y al agregar estas funcionales a la original, tendremos:

        \begin{align*}
            \frac{d v(z_t)}{dt} \leq& - \alpha_1 \left|\left| z(t) \right|\right|^2 - \left[ \alpha_2 - \varepsilon_4 \eta \lambda_{max}(R) - \varepsilon_5 \eta \lambda_{max}(R) \right] \left|\left| z(t - h) \right|\right|^2 \\
            &- \alpha_3 \int_{-h}^0 \left|\left| z(t + \theta) \right|\right|^2 d\theta
            -\left[ \varepsilon_4 \lambda_{max}(R) - \alpha_4 \right] \int_{-h - \eta}^{-h} \left|\left| z(t + \theta) \right|\right|^2 \\
            &-\left[ \varepsilon_5 \lambda_{max}(R) - \alpha_5 \right] \int_{-2h - 2\eta}^{-2h - \eta} \left|\left| z(t + \theta) \right|\right|^2
        \end{align*}

        Por lo que conlcuimos que para que esta funcional cumpla con la condición de negatividad de la derivada de Lyapunov-Krasovskii y estabilice al sistema, se deben cumplir:

        \begin{align}
            \alpha_1 &\geq 0 \\
            \alpha_2 - \varepsilon_4 \eta \lambda_{max}(R) - \varepsilon_5 \eta \lambda_{max}(R) &\geq 0 \\
            \alpha_3 &\geq 0 \\
            \varepsilon_4 \lambda_{max}(R) - \alpha_4 &\geq 0 \\
            \varepsilon_5 \lambda_{max}(R) - \alpha_5 &\geq 0
        \end{align}

    \bibliographystyle{ieeetr}
    \bibliography{bibliografia}

\end{document}