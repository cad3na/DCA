\documentclass{article}
\usepackage{graphicx} % Used to insert images
\usepackage{enumerate} % Needed for markdown enumerations to work
\usepackage{geometry} % Used to adjust the document margins
\usepackage{amsmath} % Equations
\usepackage{amssymb} % Equations
\usepackage[mathletters]{ucs} % Extended unicode (utf-8) support
\usepackage[utf8x]{inputenc} % Allow utf-8 characters in the tex document
\usepackage{fancyvrb} % verbatim replacement that allows latex
\usepackage{grffile} % extends the file name processing of package graphics 
                     % to support a larger range 
% The hyperref package gives us a pdf with properly built
% internal navigation ('pdf bookmarks' for the table of contents,
% internal cross-reference links, web links for URLs, etc.)
\usepackage{hyperref}
\usepackage{longtable} % longtable support required by pandoc >1.10
\usepackage{booktabs}  % table support for pandoc > 1.12.2
\usepackage{mathpazo}
\usepackage[spanish]{babel}

\geometry{verbose,tmargin=1in,bmargin=1in,lmargin=1in,rmargin=1in}

\author{Roberto Cadena Vega}
\title{Tarea 9 - Sistemas con retardos en la entrada}

\begin{document}
    \maketitle

    \section*{Tarea 9 - Desarrollos para asignación de espectro finito.}

        \subsection*{Equivalencia de condición de estabilidad para predictor por asignación de espectro\cite{kailath1980linear}.}

        \begin{equation}
            x(t) = B k \int_{- \tau}^0 e^{-A \delta} x(t + \delta) d \delta
        \end{equation}

        Si obtenemos la transformada de Laplace de esta integral de convolución, tendremos:

        \begin{equation*}
            x(s) = B k e^{-As} x(s)
        \end{equation*}

        Por otro lado, queremos demostrar que es equivalente a:

        \begin{equation}
            x(t) = k \int_{- \tau}^0 e^{-A \theta} B x(\theta) d \theta
        \end{equation}

        la cual, al cambiar las variable $\theta = \delta - \tau$, sabemos que cuando $\theta$ variaba de $-\tau \to 0$, $\delta$ variará de $0 \to \tau$:

        \begin{equation*}
            x(t) = k \int_{0}^{\tau} e^{-A (\delta - \tau)} B x(\delta - \theta) d \theta = k e^{A \tau} \int_{0}^{\tau} e^{-A \delta} B x(\delta - \theta) d \theta
        \end{equation*}

        por lo que su transformada de Laplace es:

        \begin{equation*}
            x(s) = k e^{A \tau} e^{-A s} B x(s)
        \end{equation*}

        \subsection*{Inversa de matriz de transformación para sistema acoplado.}

        \begin{equation}
            T(s) =
            \begin{pmatrix}
                I & 0 \\
                e^{s \tau}k & I
            \end{pmatrix}
        \end{equation}

        Si utilizamos la formula para inversión de matrices por bloques\cite{kailath1980linear},

        \begin{equation}
            \begin{pmatrix}
                A & 0 \\
                C & B
            \end{pmatrix}^{-1} =
            \begin{pmatrix}
                A^{-1} & 0 \\
                -B^{-1} C A^{-1} & B^{-1}
            \end{pmatrix}
        \end{equation}

        tendremos que la inversa de $T(s)$ es:

        \begin{equation*}
            T^{-1}(s) =
            \begin{pmatrix}
                I & 0 \\
                I e^{s \tau}k I & I
            \end{pmatrix} =
            \begin{pmatrix}
                I & 0 \\
                -e^{s \tau}k & I
            \end{pmatrix}
        \end{equation*}     

    \bibliographystyle{ieeetr}
    \bibliography{bibliografia}

\end{document}