\documentclass{article}

\usepackage{graphicx} % Used to insert images
\usepackage{enumerate} % Needed for markdown enumerations to work
\usepackage{geometry} % Used to adjust the document margins
\usepackage{amsmath} % Equations
\usepackage{amsthm}
\usepackage{mathrsfs}
\usepackage{amssymb} % Equations
\usepackage[mathletters]{ucs} % Extended unicode (utf-8) support
\usepackage[utf8x]{inputenc} % Allow utf-8 characters in the tex document
\usepackage{fancyvrb} % verbatim replacement that allows latex
\usepackage{grffile} % extends the file name processing of package graphics 
                     % to support a larger range 
% The hyperref package gives us a pdf with properly built
% internal navigation ('pdf bookmarks' for the table of contents,
% internal cross-reference links, web links for URLs, etc.)
\usepackage{hyperref}
\usepackage{longtable} % longtable support required by pandoc >1.10
\usepackage{booktabs}  % table support for pandoc > 1.12.2
\usepackage{mathpazo}
\usepackage[spanish, es-noquoting]{babel}
\usepackage{tikz}           % Paquete para crear diagramas

\geometry{verbose,tmargin=1in,bmargin=1in,lmargin=1in,rmargin=1in}

% Subpaquete para diagramas de bloques
\usetikzlibrary{shapes, arrows, decorations.markings, babel}
% Estilos comunes de bloques para diagramas
\tikzstyle{input} = [coordinate]
\tikzstyle{output} = [coordinate]
\tikzstyle{init} = [pin edge={to-, thin, black}]
\tikzstyle{sum} = [draw, circle]
\tikzstyle{empty} = [coordinate]
\tikzstyle{vecArrow} = [thick, decoration={markings, mark=at position
    1 with {\arrow[semithick]{open triangle 60}}},
    double distance=1.4pt, shorten >= 5.5pt,
    preaction = {decorate},
    postaction = {draw,line width=1.4pt, white, shorten >= 4.5pt}]
\tikzstyle{innerWhite} = [semithick, white, line width=1.4pt, shorten >= 4.5pt]
% Definiciones de teoremas
\theoremstyle{plain}
\newtheorem{lema}{Lema}
\newtheorem{teorema}{Teorema}
\newtheorem{corolario}{Corolario}
\theoremstyle{remark}
\newtheorem{nota}{Nota}
\newtheorem{resultado}{Resultado}
\theoremstyle{definition}
\newtheorem{definicion}{Definición}
\newtheorem{problema}{Problema}