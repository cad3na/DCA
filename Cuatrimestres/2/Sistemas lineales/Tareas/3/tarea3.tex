\documentclass{article}
\usepackage{graphicx} % Used to insert images
\usepackage{enumerate} % Needed for markdown enumerations to work
\usepackage{geometry} % Used to adjust the document margins
\usepackage{amsmath} % Equations
\usepackage{amssymb} % Equations
\usepackage[mathletters]{ucs} % Extended unicode (utf-8) support
\usepackage[utf8x]{inputenc} % Allow utf-8 characters in the tex document
\usepackage{fancyvrb} % verbatim replacement that allows latex
\usepackage{grffile} % extends the file name processing of package graphics 
                     % to support a larger range 
% The hyperref package gives us a pdf with properly built
% internal navigation ('pdf bookmarks' for the table of contents,
% internal cross-reference links, web links for URLs, etc.)
\usepackage{hyperref}
\usepackage{longtable} % longtable support required by pandoc >1.10
\usepackage{booktabs}  % table support for pandoc > 1.12.2
\usepackage{mathpazo}
\usepackage[spanish]{babel}

\geometry{verbose,tmargin=1in,bmargin=1in,lmargin=1in,rmargin=1in}

\author{Roberto Cadena Vega}
\title{Tarea 3}

\begin{document}
    \maketitle

    \section{Resúmen de articulo Controllability, observability, pole allocation, and state reconstruction\cite{Willems1971}}

        En este articulo se presentan formalmente los conceptos de controlabilidad y observabilidad, ayudandose de los conceptos de alcanzabilidad y reconstruccionabilidad. Luego se considera el problema de localización de polos en lazo cerrado, y se demuestra que esto es posible usando realimentación de estado, si y solo si el sistema es controlable en primer lugar. Luego se muestra que es posible hacer esto usando un reconstructor del estado (midiendo las entradas y salidas del sistema), si el sistema es observable. Despues se consideran las propiedades de estabilización y reconstruccionabilidad de los sistemas lineales variantes en el tiempo. Se consideran aspectos cualitativos de las funciones de transferencia contra las representaciones en espacio de estados y se concluye que la minimalidad del espacio de estado es equivalente a la controlabilidad y observabilidad. Finalmente se demuestra la equivalencia de la estabilidad interna en el sentido de Lyapunov y la estabilidad entrada salida en sistemas uniformemente controlables y observables.

        \subsection{Sistemas dinámicos}

        	\begin{definicion}
        		Un sistema dinámico es una quintupla $\{ \mathcal{U}, \mathcal{Y}, X, \phi, r \}$ que satisface los siguientes axiomas para todo $u_1, u_2 \in \mathcal{U}$, $x_0 \in X$, $t_0, t_1, t_2 \in \mathbb{R}$ con $t_0 \leq t_1 \leq t_2$:

        		\begin{description}
        			\item[Causalidad] $\phi(t_1, t_0, x_0 u_1) = \phi(t_1, t_0 x_0 u_2)$ siempre y cuando $u_1(t) = u_2(t)$ para $t_0 \leq t \leq t_1$.
        			\item[Consistencia]
        			\item[Propiedad de semigrupo]
        			\item[Suavidad]
        		\end{description}
        	\end{definicion}

    \bibliographystyle{ieeetr}
    \bibliography{bibliografia}

\end{document}