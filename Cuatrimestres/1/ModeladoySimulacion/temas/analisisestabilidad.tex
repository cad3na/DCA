\documentclass{article}

\usepackage[spanish,es-noquoting]{babel}
\usepackage[utf8]{inputenc}
\usepackage{amsmath}
\usepackage{amssymb}
\usepackage{graphicx}
\usepackage{bbm}
\usepackage{enumerate}
\usepackage{tikz}
\usepackage{mathpazo}
\usepackage{multicol}
\setlength{\columnsep}{0.8cm}
\usepackage{microtype}
\usepackage[skins]{tcolorbox}

\usepackage[a3paper, top=1cm, bottom=1.5cm, left=0.8cm, right=0.8cm]{geometry}

\usetikzlibrary{shapes, arrows, decorations.markings}

\newenvironment{amatrix}[1]
	{\left(\begin{array}{@{}*{#1}{c}|c@{}}}
	{\end{array}\right)}

\DeclareMathOperator{\atandos}{atan2}

\pagenumbering{gobble}

\definecolor{rojo}{rgb}{0.85, 0.15,	0.31}
\definecolor{verde}{rgb}{0.57, 0.70, 0.30}
\definecolor{azul}{rgb}{0.21, 0.49, 0.71}
\definecolor{amarillo}{rgb}{0.85, 0.72, 0.40}

\newcommand*{\mathcolor}{}
\def\mathcolor#1#{\mathcoloraux{#1}}

\newcommand*{\mathcoloraux}[3]{\protect \leavevmode \begingroup \color#1{#2}#3 \endgroup}

\newcommand{\ejex}{\mathcolor{rojo}{x}}
\newcommand{\ejey}{\mathcolor{verde}{y}}
\newcommand{\ejez}{\mathcolor{azul}{z}}
\newcommand{\ejek}{\mathcolor{amarillo}{k}}


\title{Análisis de Estabilidad}
\author{Roberto Cadena Vega}

\begin{document}
    \maketitle

    \section{Mapeos no lineales (representando sistemas dinámicos autónomos).}

        Sea un sistema dinámico no lineal, representado por la siguiente ecuación diferencial ordinaria:

        \begin{equation}
            \frac{dx}{dt} = f(x)
        \end{equation}

        donde $x \in \mathbbm{R}^n$, $f(x) : \mathbbm{R}^n \to \mathbbm{R}^n$ y $f \in C^1$, es decir, $f$ es continuamente diferenciable.

        Dado $p \in \mathbbm{R}^n$ un punto en donde se quiere analizar la estabilidad del sistema, tenemos que:

        \begin{equation}
            f(x) = f(p) + J_f(p)(x-p) + \mathcal{O}(||x-p||)
        \end{equation}

        donde $J_f(p)$ es el Jacobiano de $f$ en $p$, es decir la pendiente generalizada de la función, $(x-p)$ es la distancia entre el punto de análisis y el punto de equilibrio del sistema, y $\mathcal{O}(||x-p||)$ son terminos de orden mayor de esta distancia.

    \section{Estabilidad}

        Para un sistema dinámico que tenga puntos de equilibro de la forma

        \begin{equation}
            f(\bar{x}) = 0 \implies \frac{dx}{dt} = 0 \quad \text{en } \bar{x}
        \end{equation}

        diremos que $\bar{x}$ es un punto hiperbólico si:

        \begin{equation}
            \det{(\lambda I - J_f)} \ne bi \quad b \in \mathbbm{R}
        \end{equation}

        es decir, que este determinante no tenga raices puramente imaginarias.

        Por otro lado, una vez determinado que $\bar{x}$ es hiperbolico, podemos definir los siguientes comportamientos del sistema:

        $\bar{x}$ es estable, si las partes reales de todas las raices de $\det{(\lambda I - J_f(\bar{x}))}$ son negativas.

        $\bar{x}$ es inestable, si al menos una raiz de $\det{(\lambda I - J_f(\bar{x}))}$ tiene parte real positiva.


\end{document}
