\documentclass{article}

\usepackage[spanish,es-noquoting]{babel}
\usepackage[utf8]{inputenc}
\usepackage{amsmath}
\usepackage{amssymb}
\usepackage{graphicx}
\usepackage{bbm}
\usepackage{enumerate}
\usepackage{tikz}
\usepackage{mathpazo}
\usepackage{multicol}
\setlength{\columnsep}{0.8cm}
\usepackage{microtype}
\usepackage[skins]{tcolorbox}

\usepackage[a3paper, top=1cm, bottom=1.5cm, left=0.8cm, right=0.8cm]{geometry}

\usetikzlibrary{shapes, arrows, decorations.markings}

\newenvironment{amatrix}[1]
	{\left(\begin{array}{@{}*{#1}{c}|c@{}}}
	{\end{array}\right)}

\DeclareMathOperator{\atandos}{atan2}

\pagenumbering{gobble}

\title{Análisis de estabilidad}
\author{Roberto Cadena Vega}

\begin{document}
    \maketitle

    \section{Mapeos no lineales (representando sistemas dinámicos autónomos).}

        Sea un sistema dinámico no lineal, representado por la siguiente ecuación diferencial ordinaria:

        \begin{equation}
            \frac{dx}{dt} = f(x)
        \end{equation}

        donde $x \in \mathbbm{R}^n$, $f(x) : \mathbbm{R}^n \to \mathbbm{R}^n$ y $f \in C^1$, es decir, $f$ es continuamente diferenciable.

        Dado $p \in \mathbbm{R}^n$ un punto en donde se quiere analizar la estabilidad del sistema, tenemos que:

        \begin{equation}
            f(x) = f(p) + J_f(p)(x-p) + \mathcal{O}(||x-p||)
        \end{equation}

        donde $J_f(p)$ es el Jacobiano de $f$ en $p$, es decir la pendiente generalizada de la función, $(x-p)$ es la distancia entre el punto de análisis y el punto de equilibrio del sistema, y $\mathcal{O}(||x-p||)$ son terminos de orden mayor de esta distancia.

    \section{Teorema de Hartman - Grobman}

        Para un sistema dinámico que tenga puntos de equilibro de la forma

        \begin{equation}
            f(\bar{x}) = 0 \implies \frac{dx}{dt} = 0 \quad \text{en } \bar{x}
        \end{equation}

        diremos que $\bar{x}$ es un punto hiperbólico\footnote{Basicamente lo que dice el teorema de Hartman - Grobman, es que dado un sistema no lineal, existe una vecindad de puntos, $N$, alrededor de los puntos de equilibrio del sistema, $x_i$, para los cuales existe un homomorfismo entre el mismo sistema, $\dot{x} = f(x)$, y un mapeo lineal, $\dot{x} = Ax$, es decir una linealización, si la linealización tiene un polinomio caracteristico con raices con parte real no nula.} si:

        \begin{equation}
            \det{(\lambda I - J_f)} \ne bi \quad b \in \mathbbm{R}
        \end{equation}

        es decir, que este determinante no tenga raices puramente imaginarias.

        Por otro lado, una vez determinado que $\bar{x}$ es hiperbólico, podemos definir los siguientes comportamientos del sistema:

        $\bar{x}$ es estable, si las partes reales de todas las raices de $\det{(\lambda I - J_f(\bar{x}))}$ son negativas.

        $\bar{x}$ es inestable, si al menos una raiz de $\det{(\lambda I - J_f(\bar{x}))}$ tiene parte real positiva.

    \section{Ejemplo 1}

        Sea el sistema presa - depredador:

        \begin{eqnarray*}
            \dot{x}_1 & = & a x_1 - b x_1 x_2 \\
            \dot{x}_2 & = & -c x_2 + d x_1 x_2
        \end{eqnarray*}

        es decir:

        \begin{equation*}
            \dot{x} = f(x)
        \end{equation*}

        donde:

        \begin{equation}
            x =
            \begin{pmatrix}
                x_1 \\
                x_2
            \end{pmatrix}
        \end{equation}

        \begin{equation} \label{eq:predep}
            f(x) =
            \begin{pmatrix}
                a x_1 - b x_1 x_2 \\
                -c x_2 + d x_1 x_2
            \end{pmatrix} : \mathbbm{R}^2 \to \mathbbm{R}^2 
        \end{equation}

        \subsection{Puntos de equilibrio}

            Sea el primer punto de equilibrio:

            \begin{equation*}
                x_1 =
                \begin{pmatrix}
                    0 \\
                    0
                \end{pmatrix}
            \end{equation*}

            ya que:

            \begin{equation*}
                f(x_1) =
                \begin{pmatrix}
                    a \cdot 0 - b \cdot 0 \cdot 0 \\
                    -c \cdot 0 + d \cdot 0 \cdot 0
                \end{pmatrix} =
                \begin{pmatrix}
                    0 \\
                    0
                \end{pmatrix}
            \end{equation*}

            y el segundo punto de equilibrio:

            \begin{equation*}
                x_2 =
                \begin{pmatrix}
                    \frac{c}{d} \\
                    \frac{a}{b}
                \end{pmatrix}
            \end{equation*}

            ya que:

            \begin{equation*}
                f(x_2) =
                \begin{pmatrix}
                    a \cdot \frac{c}{d} - b \cdot \frac{c}{d} \cdot \frac{a}{b} \\
                    -c \cdot \frac{a}{b} + d \cdot \frac{c}{d} \cdot \frac{a}{b}
                \end{pmatrix} =
                \begin{pmatrix}
                    0 \\
                    0
                \end{pmatrix}
            \end{equation*}

        \subsection{Jacobiano}

            Si bien el sistema $\dot{x} = f(x)$, con los parametros de la ecuación ~\ref{eq:predep} es una función en $\mathbbm{R}^2$, tambien lo podemos ver como un par de funciones en $\mathbbm{R}$:

            \begin{equation*}
                f(x) =
                \begin{pmatrix}
                    a x_1 - b x_1 x_2 \\
                    -c x_2 + d x_1 x_2
                \end{pmatrix} =
                \begin{pmatrix}
                    f_1(x) \\
                    f_2(x)
                \end{pmatrix}
            \end{equation*}

            Entonces su Jacobiano estará dado por la formula:

            \begin{equation}
                J_f(x) =
                \begin{pmatrix}
                    \frac{\partial f_1}{\partial x_1} & \frac{\partial f_1}{\partial x_2} \\
                    \frac{\partial f_2}{\partial x_1} & \frac{\partial f_2}{\partial x_2}
                \end{pmatrix}
            \end{equation}

            En especifico, para nuestro sistema:

            \begin{equation*}
                J_f(x) =
                \begin{pmatrix}
                    a - b x_2 & - b x_1 \\
                    d x_2 & - c + d x_1
                \end{pmatrix}
            \end{equation*}

        \subsection{Hiperbolicidad}

            Para descubrir si el punto de equilibrio planteado es hiperbólico o no, tenemod que sacar las raices del polinomio que resulta de $\det{(\lambda I - J_f)}$, empezaremos con el segundo punto de equilibrio.

            \begin{equation*}
                \lambda I - J_f(x_2) =
                \begin{pmatrix}
                    \lambda & 0 \\
                    0 & \lambda
                \end{pmatrix} -
                \begin{pmatrix}
                    0 & -\frac{bc}{d} \\
                    \frac{da}{b} & 0
                \end{pmatrix} =
                \begin{pmatrix}
                    \lambda & \frac{bc}{d} \\
                    -\frac{da}{b} & \lambda
                \end{pmatrix}
            \end{equation*}

            por lo que:

            \begin{equation*}
                \det{(\lambda I - J_f)} = \lambda^2 + ac
            \end{equation*}

            de aqui podemos ver que sus raices son:

            \begin{equation*}
                \lambda_{1,2} = \pm \sqrt{-ac} = \pm ac i 
            \end{equation*}

            y como sus raices son puramente imaginarias, podemos decir que $x_2$ no es hiperbólico.

            Por otro lado, el primer punto de equilibrio:

            \begin{equation*}
                \lambda I - J_f(x_1) =
                \begin{pmatrix}
                    \lambda & 0 \\
                    0 & \lambda
                \end{pmatrix} -
                \begin{pmatrix}
                    a & 0 \\
                    0 & -c
                \end{pmatrix} =
                \begin{pmatrix}
                    \lambda - a & 0 \\
                    0 & \lambda + c
                \end{pmatrix}
            \end{equation*}

            por lo que:

            \begin{equation*}
                \det{(\lambda I - J_f)} = \left( \lambda - a \right) \left( \lambda + c \right)
            \end{equation*}

            Sus raices son $\lambda_1 = a$ y $\lambda_2 = -c$, por lo que si podemos aplicar el teorema de Hartman - Grobman y decir que es inestable debido a $\lambda_1 = a$.

    \section{Ejemplo 2}

        Sea el sistema presa - depredador con limitante de recursos:

        \begin{eqnarray*}
            \dot{x}_1 = f_1(x) & = & a x_1 - b x_1 x_2 - e x_1^2 \\
            \dot{x}_2 = f_2(x) & = & -c x_2 + d x_1 x_2
        \end{eqnarray*}

        \subsection{Puntos de Equilibrio}

        \begin{equation*}
            x_1 =
            \begin{pmatrix}
                0 \\
                0
            \end{pmatrix} \quad
            x_2 =
            \begin{pmatrix}
                \frac{c}{d} \\
                \frac{ad - ec}{db}
            \end{pmatrix}
        \end{equation*}

        \subsection{Jacobiano}

            \begin{equation*}
                J_f(x) =
                \begin{pmatrix}
                    a - b x_2 - 2 e x_1 & - b x_1 \\
                    d x_2 & - c + d x_1
                \end{pmatrix}
            \end{equation*}

        \subsection{Hiperbolicidad}

            Para el primer punto de equilibrio:

            \begin{equation*}
                J_f(x_1) =
                \begin{pmatrix}
                    a & 0 \\
                    0 & -c
                \end{pmatrix}
            \end{equation*}

            \begin{equation*}
                \det{(\lambda I - J_f(x_1))} = \left( \lambda - a \right) \left( \lambda + c \right)
            \end{equation*}

            por lo que el primer punto de equilibrio $x_1$ es inestable.

            Por otro lado, el segundo punto de equilibrio:

            \begin{equation*}
                J_f(x_2) =
                \begin{pmatrix}
                    -\frac{ec}{d} & - \frac{bc}{d} \\
                    \frac{ad - ec}{b} & 0
                \end{pmatrix}
            \end{equation*}

            \begin{equation*}
                \det{(\lambda I - J_f(x_2))} = \lambda \left( \lambda + \frac{ec}{d} \right) + \frac{bc}{d} \frac{ad - ec}{b} = \lambda^2 + \frac{ec}{d} \lambda + ac - \frac{ec^2}{d}
            \end{equation*}

            Y si analizamos esta ecuación y el significado detras de los parametros, nos daremos cuenta que estas raices solo pueden ser de parte real negativa, por lo que $x_2$ es estable.

\end{document}
