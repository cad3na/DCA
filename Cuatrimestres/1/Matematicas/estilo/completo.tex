% Compila en uno de dos formatos diferentes dependiendo del valor de ebook, el
% cual puede ser pasado por medio de una cadena de caracteres al mandar compilar
% el archivo, un ejemplo es este:
%
% pdflatex -interaction=batchmode "\newif\ifebook\ebookfalse\input{archivo.tex}"
%
% pdflatex -interaction=batchmode "\newif\ifebook\ebooktrue\input{archivo.tex}"
%
% Estas dos entradas produciran archivos diferentes con un formato especifico
\ifebook
    \documentclass[a5paper, 15pt]{book} % Papel: A5, Letra: 15pt
    \usepackage[margin=2mm]{geometry} % Margenes estrechos
    \usepackage{mathpazo} % Letra: Palatino
    \usepackage{eulervm} % Letra para matematicas: Euler
    \usepackage{wrapfig}
    \newenvironment{marginfigure} % Definicion complementaria
        {\begin{wrapfigure}{r}{0.2\textwidth}}
        {\end{wrapfigure}}
\else
    \documentclass{tufte-book} % Papel: Letter, Estilo: Tufte
    \AtBeginDocument{\fontsize{11}{15}\selectfont}
    \geometry{
    left=15mm, % left margin
    textwidth=120mm, % main text block
    marginparsep=6mm, % gutter between main text block and margin notes
    marginparwidth=60mm % width of margin notes
    }
\fi
% Paquetes a utilizar
\usepackage[spanish, es-noquoting]{babel} % Español
\usepackage[usenames,dvipsnames]{xcolor} % Nombres de colores
\usepackage[utf8]{inputenc} % Texto de entrada en UTF-8
\usepackage{amsmath}        % Paquete con definiciones matematicas
\usepackage{amsthm}         % Paquete para definir ambientes de teoremas
\usepackage{amssymb}        % Paquete con simbolos matematicos
\usepackage{xfrac}          % Paquete para usar fracciones inclinadas
%\usepackage{graphicx}       % Paquete para incluir imagenes
\usepackage{bbm}            % Paquete con simbolos de campos
\usepackage{enumerate}      % Paquete para crear ambientes de listas enumeradas
%\usepackage{tikz}           % Paquete para crear diagramas
\usepackage{todonotes}      % Paquete para crear anotaciones
\usepackage{mathrsfs}       % Paquete para mas simbolos de campos
%\usepackage{steinmetz}      % Paquete para escribir numeros complejos
\usepackage{tcolorbox}
% % Subpaquete para diagramas de bloques
% \usetikzlibrary{shapes, arrows, decorations.markings}
% % Estilos comunes de bloques para diagramas
% \tikzstyle{input} = [coordinate]
% \tikzstyle{output} = [coordinate]
% \tikzstyle{init} = [pin edge={to-, thin, black}]
% \tikzstyle{sum} = [draw, circle]
% \tikzstyle{empty} = [coordinate]
% \tikzstyle{vecArrow} = [thick, decoration={markings, mark=at position
%     1 with {\arrow[semithick]{open triangle 60}}},
%     double distance=1.4pt, shorten >= 5.5pt,
%     preaction = {decorate},
%     postaction = {draw,line width=1.4pt, white, shorten >= 4.5pt}]
% \tikzstyle{innerWhite} = [semithick, white, line width=1.4pt, shorten >= 4.5pt]
% Hace que la numeración dependa del numero de capitulo
\setcounter{secnumdepth}{0}
\numberwithin{equation}{chapter}
% Definición de matriz adjunta
\newenvironment{amatrix}[1]
    {\left(\begin{array}{@{}*{#1}{c}|c@{}}}
    {\end{array}\right)}
% Definiciones de teoremas
\theoremstyle{plain}
\newtheorem{lema}{Lema}[chapter]
\newtheorem{teorema}{Teorema}[chapter]
\newtheorem{corolario}{Corolario}[chapter]
\theoremstyle{remark}
\newtheorem{nota}{Nota}[chapter]
\theoremstyle{definition}
\newtheorem{definicion}{Definición}[chapter]
\newtheorem{problema}{Problema}[chapter]
% % Opciones para colores en teoremas
% \tcbuselibrary{skins}
% % Ambiente con colores para teoremas de estilo plano
% \tcolorboxenvironment{lema}
%     {enhanced jigsaw,
%     colframe=SkyBlue,
%     interior hidden,
%     before skip=10pt,
%     after skip=10pt}
% \tcolorboxenvironment{teorema}
%     {enhanced jigsaw,
%     colframe=SkyBlue,
%     interior hidden,
%     before skip=10pt,
%     after skip=10pt}
% \tcolorboxenvironment{corolario}
%     {enhanced jigsaw,
%     colframe=SkyBlue,
%     interior hidden,
%     before skip=10pt,
%     after skip=10pt}
% % Ambiente con colores para teoremas de estilo anotacion
% \tcolorboxenvironment{nota}
%     {blanker,
%     left=5mm,
%     before skip=10pt,
%     after skip=10pt,
%     borderline west={1mm}{0pt}{Orchid}}
% % Ambiente con colores para teoremas de estilo definicion
% \tcolorboxenvironment{definicion}
%     {frame empty,
%     colback=OliveGreen!10!White,
%     grow to left by=6pt,
%     grow to right by=6pt,
%     left=1.6pt,
%     right=1.6pt,
%     arc=5pt}
% \tcolorboxenvironment{problema}
%     {frame empty,
%     colback=OliveGreen!10!White,
%     grow to left by=6pt,
%     grow to right by=6pt,
%     left=1.6pt,
%     right=1.6pt,
%     arc=5pt}
% Definiciones de operadores
\DeclareMathOperator{\rango}{rango}
\DeclareMathOperator{\imagen}{imagen}
\DeclareMathOperator{\grado}{grado}
\DeclareMathOperator{\traza}{traza}
\DeclareMathOperator{\supremo}{supremo}
\DeclareMathOperator{\cardinalidad}{cardinalidad}
% Definiciones para recordatorios por hacer
\newcommand{\tarea}[2][]
    {\todo[
        linecolor=black!50!white,       % Color de linea: Gris
        backgroundcolor=black!30!white, % Color de fondo: Gris claro
        bordercolor=black!30!white,     % Color de borde: Gris claro
        #1]{#2}}
\newcommand{\faltante}[2][]
    {\todo[
        linecolor=red!50!white,        % Color de linea: Rojo
        backgroundcolor=red!30!white,  % Color de fondo: Rojo claro
        bordercolor=red!30!white,      % Color de borde: Rojo claro
        #1]{#2}}
