%-------------------------------------------------------------------------------
%	EMPIEZA SECCION
%-------------------------------------------------------------------------------

\section{Grupos}

    \subsection{Definiciones}

        \begin{definicion}
            Un grupo es un conjunto no vacio $G$ en el que esta definida la operacion $\op$, tal que:

            \begin{eqnarray}
                \op \colon G, G & \to & G \nonumber \\
                (a, b) & \to & (a \op b)
            \end{eqnarray}

            Existen definiciones parciales de grupo dependiendo de las propiedades que cumple su operación:

            \begin{description}
                \item [Cerradura] $a \op b \in G \quad \forall \+ a, b \in G$
                \item [Asociatividad] $a \op (b \op c) = (a \op b) \op c \quad \forall \+ a, b, c, \in G$
                \item [Identidad] $\exists\+ e \in G \ni a \op e = e \op a = a \quad \forall \+ a \in G$
                \item [Inverso] $\exists\+ b \in G \ni a \op b = b \op a = e \quad \forall \+ a \in G$
            \end{description}

            Cuando se cumplen las propiedades de \emph{cerradura} y \emph{asociatividad} se le llama \emph{semigrupo}; si adicionalmente se cumple la propiedad de \emph{existencia de identidad} se le llama \emph{monoide}; si adicionalmente se cumple la propiedad de \emph{existencia de inverso} se le llama \emph{grupo}.
        \end{definicion}

        \begin{ejercicio}
            Demostrar que el grupo cimpuesto por las matrices de la forma:

            \begin{equation*}
                \begin{pmatrix}
                    \cos{\theta} & \sin{\theta} \\
                    -\sin{\theta} & \cos{\theta}
                \end{pmatrix}, \quad \forall\+ \theta \in \mathbb{R}
            \end{equation*}

            es un grupo.
        \end{ejercicio}

        \begin{definicion}
            Se dice que un grupo $G$ es abeliano si:

            \begin{equation}
                a \op b = b \op a
            \end{equation}
        \end{definicion}

        \begin{ejemplo}
            El conjunto $\sfrac{\mathbb{Z}}{n\mathbb{Z}}$
        \end{ejemplo} \faltante{Falta escribir ejemplo}

        \begin{ejercicio}
            Consideremos a $\mathbb{Z}$ con el producto usual ¿Es este un grupo?
        \end{ejercicio}

        \begin{ejercicio}
            Consideremos a $\mathbb{Z}^+$ con el producto usual ¿Es este un grupo?
        \end{ejercicio}

        \begin{ejercicio}
            Sea $G = \mathbb{R} \setminus \{0\}$. Si definimos $a \op b = a^2 b$ ¿G es un grupo?
        \end{ejercicio}

        \begin{definicion}
            Orden de un grupo es el numero de elementos que tiene dicho grupo y se denota por $|G|$.

            Un grupo $G$ será finito si tiene orden finito, de lo contrario será infinito.
        \end{definicion}

        \begin{ejemplo}
            Si $G = \{e\}$, su orden será $\left| G = \{e\} \right| = 1$
        \end{ejemplo}

        \begin{ejemplo}
            El orden del conjunto de numeros reales es infinito $\left| \mathbb{R} \right| = \infty$.
        \end{ejemplo}

        \begin{proposicion}
            Si $G$ es un grupo, entonces:

            \begin{enumerate}
                \item El elemento identidad es único.
                \item El elemento inverso $a^{-1} \quad \forall \+ a \in G$ es único.
                \item El elemento inverso del inverso del un elemento del grupo es el mismo elemento $\left( a^{-1} \right)^{-1} = a \quad \forall \+ a \in G$.
                \item El elemento inverso de la operación de dos elementos del grupo es la operacion de los inversos de los elementos en orden inverso $\left( a \op b \right)^{-1} = b^{-1} \op a^{-1}$
                \item En general lo anterior se cumple para cualquier numero de elementos $(a_1 \op a_2 \op \dots \op a_n)^{-1} = a_n^{-1} \op \dots \op a_2^{-1} \op a_1^{-1}$.
            \end{enumerate}
        \end{proposicion}

        \begin{proof}\mbox{}\\
            \begin{enumerate}
                \item Dados $e_1$ y $e_2$ identidades del grupo, son identicos.
                Si aplicamos la identidad $e_2$ a $e_1$, tenemos como resultado $e_1$, y si aplicamos la identidad $e_1$ a $e_2$ obtenemos como resultado $e_2$:

                \begin{equation*}
                    e_1 = e_2 \op e_1 = e_1 \op e_2 = e_2
                \end{equation*}

                por lo que podemos ver que ambas identidades son la misma.

                \item Sean $b, c$ inversos de $a$, entonces:

                \begin{eqnarray*}
                    b \op a & = & e \\
                    a \op c & = & e
                \end{eqnarray*}

                por lo que podemos ver que:

                \begin{equation*}
                    b = b \op e = b \op (a \op c) = (b \op a) \op c = e \op c = c
                \end{equation*}

                \item Sabemos que existe un inverso $a^{-1}$ tal que:

                \begin{equation*}
                    a \op a^{-1} = a^{-1} \op a = e \quad \forall \+ a \in G
                \end{equation*}

                asi pues, se sigue que:

                \begin{equation*}
                    \left( a^{-1} \right)^{-1} \op a^{-1} = e
                \end{equation*}

                y como sabemos que el elemento que operado con el inverso sea la identidad es el elemento mismo tenemos que:

                \begin{equation*}
                    \left( a^{-1} \right)^{-1} = a
                \end{equation*}

                \item Si operamos por la izquierda el termino $b^{-1} \op a^{-1}$ con $a \op b$:

                \begin{equation*}
                    \left( b^{-1} \op a^{-1} \right) \op \left( a \op b \right) = b^{-1} \op \left( a^{-1} \op a \right) b = b^{-1} \op e \op b = b^{-1} \op b = e
                \end{equation*}

                de la misma manera si operamos por la derecha:

                \begin{equation*}
                    \left( a \op b \right) \op \left( b^{-1} \op a^{-1} \right) = a^{-1} \op \left( b^{-1} \op b \right) a = a^{-1} \op e \op a = a^{-1} \op a = e
                \end{equation*}

                por lo tanto:

                \begin{equation*}
                    b^{-1} \op a^{-1} = \left( a \op b \right)^{-1}
                \end{equation*}

            \end{enumerate}
        \end{proof}

%-------------------------------------------------------------------------------

    \subsection{Reglas de cancelación}

        \begin{proposicion}
            Sea $G$ un grupo y $a, b, c \in G$, tendremos que:

            \begin{eqnarray}
                a \op b = a \op c & \implies & b = c \nonumber \\
                b \op a = c \op a & \implies & b = c
            \end{eqnarray}
        \end{proposicion}

        \begin{proof}
            Si tomamos en cuenta que $a \op b = a \op c$:

            \begin{equation*}
                b = e \op b = \left( a^{-1} \op a \right) \op b = a^{-1} \op \left( a \op b \right) = a^{-1} \op \left( a \op c \right) = \left( a^{-1} \op a \right) \op c = e \op c = c
            \end{equation*}

            de la misma manera para $b \op a = c \op a$:

            \begin{equation*}
                b = b \op e = b \op \left( a \op a^{-1} \right) = \left( b \op a \right) \op a^{-1} = \left( c \op a \right) \op a^{-1} = c \op \left( a \op a^{-1} \right) = c \op e = c
            \end{equation*}
        \end{proof}

%-------------------------------------------------------------------------------

    \subsection{Subgrupos}

        \begin{definicion}
            Un subconjunto no vacio $H$ de un grupo $G$ se llama subgrupo si $H$ mismo forma un grupo respecto a la operación de $G$.

            Cuando $H$ es subgrupo de $G$ se denota $H < G$ o $G > H$.
        \end{definicion}

        \begin{observacion}
            Todo grupo $G$ tiene automaticamente dos subconjuntos triviales, el mismo $G$ y la identidad $\{e\}$.
        \end{observacion}

        \begin{proposicion}
            Un subconjunto no vacio $H \subset G$ es un subgrupo de $G$ si y solo si $H$ es cerrado respecto a la operación de $G$ y $a \in H \implies a^{-1} \in H$.
        \end{proposicion}

        \begin{proof}
            Teniendo que $H$ es un subgrupo de $G$ tenemos que $H$ es un grupo, por lo que automaticamente se cumple la cerradura y la existencia del inverso dentro del subgrupo.

            Teniendo que $H$ es cerrado, no vacio y $a^{-1} \in H \quad \forall \+ a \in H$.
            Sabemos que $a^{-1} \op a  = e \in H$ debido a que $H$ es cerrado.
            Ademas para $a, b, c \in H$ sabemos que $a \op (b \op c) = (a \op b) \op c$ debido a que se cumple en $G$ y $H$ hereda esta propiedad.

            Por lo que $H$ es un grupo, y por lo tanto subgrupo de $G$.
        \end{proof}

        \begin{ejemplo}
            Sea $G = \mathbb{Z}$ con la suma usual y sea $H$ el conjunto de los enteros pares, es decir:

            \begin{equation*}
                H = \left\{ 2n | n \in \mathbb{Z} \right\}
            \end{equation*}

            ¿Es $H$ un subgrupo de $G$?

            Empecemos con dos elementos $a, b \in H$, por lo que tenemos que:

            \begin{eqnarray*}
                a & = & 2 q \quad q \in \mathbb{Z} \\
                b & = & 2 q' \quad q' \in \mathbb{Z}
            \end{eqnarray*}

            y al sumarlos tenemos que:

            \begin{equation*}
                a + b = 2q + 2q' = 2(q + q') = 2q'' \quad q'' \in \mathbb{Z}
            \end{equation*}

            por lo que $a + b \in H$.

            Por otro lado, para $a \in H$ existe un $q \in \mathbb{Z}$ tal que $a = 2q$.
            Su inverso será:

            \begin{equation*}
                -a = -2 q = 2(-q)
            \end{equation*}

            por lo que existe $q' = -q \in \mathbb{Z}$ tal que:

            \begin{equation*}
                2 q' = -a \in H
            \end{equation*}

            y por lo tanto $H < \mathbb{Z}$.
        \end{ejemplo}

        \begin{ejemplo}
            Consideremos $G = \mathbb{C}^* = \mathbb{C} \setminus \{ 0 \}$ con el producto usual, y un subconjunto $\mathcal{U}$

            \begin{equation*}
                \mathcal{U} = \left\{ z \in \mathbb{C}^* \mid |z| = 1 \right\}
            \end{equation*}

            ¿Es $\mathcal{U}$ un subgrupo de $G$?

            Dados dos elementos $z_1, z_2 \in \mathcal{U}$ sabemos que $|z_1| = |z_2| = 1$, por lo tanto:

            \begin{equation*}
                |z_1 z_2| = |z_1| |z_2| = 1
            \end{equation*}

            por lo que $z_1 z_2 \in \mathcal{U}$.

            Por otro lado, para $z \in \mathcal{U}$ tenemos que $|z| = 1$, y por lo tanto:

            \begin{equation*}
                |z^{-1}| = |z|^{-1} = \frac{a}{|z|} = 1
            \end{equation*}

            por lo que $z^{-1} \in \mathcal{U}$ y $\mathcal{U} < \mathbb{C}^*$
        \end{ejemplo}

        \begin{ejemplo}
            Sea $G$ un grupo, $a$ un elemento del grupo y $C(a) = \left\{ g \in G \mid g \op a = a \op g \right\}$ ¿Es $C(a)$ subgrupo de $G$?

            Primero notamos que $C(a)$ es no vacio debido a que al menos tiene a la identidad.

            \begin{equation*}
                e \op a = a \op e \implies e \in C(a)
            \end{equation*}

            Ahora tomemos dos elementos $g_1, g_2 \in C(a)$, para los cuales:

            \begin{eqnarray*}
                g_1 \op a & = & a \op g_1 \\
                g_2 \op a & = & a \op g_2
            \end{eqnarray*}

            Ahora, si operamos estos dos elementos tendremos:

            \begin{equation*}
                (g_1 \op g_2) \op a = g_1 \op (g_2 \op a) = g_1 \op (a \op g_2) = (g_1 \op a) \op g_2 = (a \op g_1) \op g_2 = a \op (g_1 \op g_2)
            \end{equation*}

            por lo que $g_1 \op g_2 \in C(a)$.

            Por ultimo, podemos ver que:

            \begin{equation*}
                a = a \op e = a \op \left( g \op g^{-1} \right) = (g \op a) \op g^{-1}
            \end{equation*}

            En donde para que el elemento inverso exista en $C(a)$, se debe de cumplir que $g^{-1} \op a = a \op g^{-1}$:

            \begin{equation*}
                g^{-1} \op a = g^{-1} \op \left( (g \op a) \op g^{-1} \right) = g^{-1} \op (g \op a) \op g^{-1} = g^{-1} \op g \op a \op g^{-1} = e \op a \op g^{-1} = a \op g^{-1}
            \end{equation*}

            Por lo que $C(a) < G$.
        \end{ejemplo}

        \begin{ejercicio}
            Sea $X$ un conjunto no vacio. Consideremos $G = \delta X$. Sea $a \in X$, $H(a) = \left\{ f \in \delta X \mid f(a) = a \right\}$. Verificar que $H \subset G$ es un subgrupo bajo la composición de funciones. Note que $H(a)$ es no vacio, debido a que $\identidad_X \in H(a)$.
        \end{ejercicio}

        \begin{definicion}
            Sea $G$ un grupo y $a \in G$. El conjunto

            \begin{equation}
                A = \langle a \rangle = \left\{ a^i \mid i \in \mathbb{Z} \right\}
            \end{equation}

            es un subgrupo de $G$.

            A es no vacio, puesto que $a^0 = e \in A$.

            Por otro lado, para dos elementos $a^i, a^j \in A$ tenemos que:

            \begin{equation*}
                a^i a^j = a^{i+j} \in A
            \end{equation*}

            y para un elemento $a^i \in A$, tenemos que:

            \begin{equation*}
                a^{-i} = \left( a^i \right)^{-1} = \left( a^{-1} \right)^i \in A
            \end{equation*}

            por lo que $\langle a \rangle$ es un subgrupo. A este se le llama subgrupo cíclico de $G$ generado por $a$.
        \end{definicion}

        \begin{definicion}
            Sea $G$ un grupo, decimos que $G$ es cíclico si $G = \langle a \rangle$ para algun $a \in G$.
        \end{definicion}

        \begin{ejemplo}
            Dado el grupo $G = \{e\}$, tenemos que el subgrupo cíclico generador de $G$ es:

            \begin{equation*}
                \langle e \rangle = \left\{ e^i \in G \mid i \in \mathbb{Z} \right\}
            \end{equation*}

            al operar este subgrupo tenemos:

            \begin{eqnarray*}
                e^1 & = & e \\
                e^2 & = & e \op e = e \\
                e^3 & = & e \op e \op e = e
            \end{eqnarray*}

            por lo que obtenemos todos los elementos del grupo.
        \end{ejemplo}

        \begin{ejemplo}
            Dado el grupo $G = \{a, e\}$, y la siguiente tabla para la operación del grupo:

            \begin{center}
                \begin{tabular}{l | c r}
                    $\op$ & e & a \\
                    \hline
                    e & e & a \\
                    a & a & e
                \end{tabular}
            \end{center}

            con esto, tenemos que el subgrupo ciclico generador de $G$ es:

            \begin{equation*}
                \langle a \rangle = \left\{ a^i \in G \mid i \in \mathbb{Z} \right\}
            \end{equation*}

            y al operar este subgrupo tenemos:

            \begin{eqnarray*}
                a^1 & = & a  \\
                a^2 & = & a \op a = e
            \end{eqnarray*}

            y obtenemos todos los elementos del grupo.
        \end{ejemplo}

        \begin{ejercicio}
            Dado el grupo $G = \{e, a, b\}$ y la operación:

            \begin{center}
                \begin{tabular}{l | c c r}
                    $\op$ & e & a & b \\
                    \hline
                    e & e & a & b \\
                    a & a & b & e \\
                    b & b & e & a
                \end{tabular}
            \end{center}

            Encontrar el subgrupo cíclico generador.
        \end{ejercicio}

        \begin{ejercicio}
            Dado el grupo $\sfrac{\mathbb{Z}}{2\mathbb{Z}} = \mathbb{Z}_2 = \left\{ [0], [1] \right\}$ con la operación $[a] + [b]$; encontrar el subgrupo cíclico generador.
        \end{ejercicio}

        \begin{ejercicio}
            Sea $G$ un grupo en el que $x^2 = e$ para todo $x \in G$. Verificar que $G$ es abeliano, es decir $a \op b = b \op a$.
        \end{ejercicio}

        \begin{definicion}
            Sea $G$ un grupo, $H$ un subgrupo de $G$ ($H < G$), para $a, b \in G$, decimos que $a$ es congruente con $b \mod H$, denotado por:

            \begin{equation}
                a \cong b \mod H
            \end{equation}

            si

            \begin{equation}
                a \op b^{-1} \in H
            \end{equation}
        \end{definicion}

        \marginnote{De aqui en adelante se deja la notación de la operación $\op$ para la operación genérica del grupo, sin que por esto se entienda que la operación es siempre el producto usual.}

        \begin{ejercicio}
            Demostrar que $\cong$ es una relación de equivalencia.
        \end{ejercicio}

        \begin{definicion}
            Si $H$ es un subgrupo de $G$ y $a \in G$, entonces

            \begin{equation}
                Ha = \left\{ ha \mid h \in H \right\}
            \end{equation}

            se llama clase lateral derecha de $H$ en $G$.
        \end{definicion}

        \begin{lema}
            Para todo $a \in G$ se tiene que:

            \begin{equation}
                Ha = \left\{ x \in G \mid a \cong x \mod H \right\}
            \end{equation}
        \end{lema}

        \begin{proof}
            Sea un conjunto definido como $[a] = \left\{ x \in G \mid a \cong x \mod H \right\}$, por verificar que $Ha = [a]$. Para verficar esto, tenemos que verificar que $Ha \subseteq [a]$ y despues que $[a] \subseteq Ha$.

            Para verificar que $Ha \subseteq [a]$ definimos un elemento $h \in H$ y $ha \in Ha$, si ahora operamos $a$ con $(ha)^{-1}$ y verificamos que esta en $H$, podemos decir que $a \cong ha \mod H$:

            \begin{equation*}
                a(ha)^{-1} = a(a^{-1}h^{-1}) = (a a^{-1})h^{-1} = h^{-1} \in H
            \end{equation*}

            por lo que podemos concluir que $a \cong ha \mod H$, lo que implica que $ha \in [a]$; pero como $ha$ es un elemento arbitrario de $Ha$, tenemos que:

            \begin{equation*}
                Ha \subseteq [a]
            \end{equation*}

            Para verificar que $[a] \subseteq Ha$ empezamos con un elemento $x \in [a]$, es decir $a \cong x \mod H$, lo cual implica $a x^{-1} \in H$, en particular nos interesa:

            \begin{equation*}
                (a x^{-1})^{-1} = x a^{-1} \in H
            \end{equation*}

            Por otro lado, sea $h = x a^{-1} \in H$, entonces tenemos que:

            \begin{equation*}
                ha = (x a^{-1})a = x (a^{-1} a) = x \in Ha
            \end{equation*}

            por lo que podemos decir que:

            \begin{equation*}
                [a] \subseteq Ha
            \end{equation*}

            y por lo tanto

            \begin{equation*}
                [a] = Ha
            \end{equation*}
        \end{proof}

        \begin{teorema}
            Sea $G$ un grupo finito y $H \subset G$, entonces el orden de $H$ divide al orden de $G$

            \begin{equation}
                \sfrac{|H|}{|G|}
            \end{equation}

            y esto implica que existe una $k \in \mathbb{Z}$ tal que:

            \begin{equation}
                |G| = k |H|
            \end{equation}

            A esto se le conoce como Teorema de Lagrange.
        \end{teorema}

        \begin{proof}
            Dado $[a] = Ha$, las clases de equivalencia forman una partición de $G$:

            \begin{eqnarray*}
                [a_1] \cup [a_2] \cup \dots \cup [a_k] & = & G \\[0cm]
                [a_i] \cap [a_j] & = & \varnothing \quad i \ne j
            \end{eqnarray*}

            Por otro lado, las clases laterales derechas forman una partición:

            \begin{eqnarray*}
                H a_1 \cup H a_2 \cup \dots \cup H a_k & = & G \\
                H a_i \cap H a_j & = & \varnothing \quad i \ne j
            \end{eqnarray*}

            Establezcamos una biyección:

            \begin{eqnarray*}
                Ha_i & \to & H \\
                ha_i & \to & h
            \end{eqnarray*}

            es decir, el orden de $Ha_i$ es el orden de $H$

            \begin{equation*}
                |Ha_i| = |H| \quad \forall \+ 1 \le i \le k
            \end{equation*}

            entonces:

            \begin{eqnarray*}
                |G| & = & |Ha_1| + |Ha_2| + \dots + |Ha_k| \\
                & = & |H| + |H| + \dots + |H| \\
                |G| & = & k |H|
            \end{eqnarray*}

            pero $k \in \mathbb{Z}$, entonces:

            \begin{equation*}
                \sfrac{|H|}{|G|}
            \end{equation*}
        \end{proof}

        \begin{definicion}
            Si $G$ es finito y $H$ es un subgrupo de $G$ lllamamos $\frac{|G|}{|H|}$ el indice de $H$ en $G$ y lo denotamos por $i_G(H)$.
        \end{definicion}

        \begin{definicion}
            Si $G$ es finito y $a \in G$, llamamos orden de $a$ al mínimo entero positivo $n$ tal que $a^n = e$ y lo denotamos por $O(a)$, por lo que se sigue que:

            \begin{equation}
                a^{O(a)} = e
            \end{equation}
        \end{definicion}

        \begin{proposicion}
            Si $G$ es finito y $a \in G$, entonces el orden de $a$ divide al orden de $G$:

            \begin{equation}
                \sfrac{O(a)}{|G|}
            \end{equation}
        \end{proposicion}

        \begin{proof}
            Supongamos $H = \langle a \rangle$, entonces $O(a) = |H|$. Podemos ver ahora, por el teorema de Lagrange:

            \begin{equation*}
                \sfrac{|H|}{|G|} \implies \sfrac{O(a)}{|G|}
            \end{equation*}
        \end{proof}

        \begin{corolario}
            Si $G$ es un grupo finito de orden $n$, entonces:

            \begin{equation}
                a^n = e \quad \forall \+ a \in G
            \end{equation}
        \end{corolario}

        \begin{proof}
            Por la proposición anterior tenemos que:

            \begin{equation*}
                \sfrac{O(a)}{|G|} = \sfrac{O(a)}{n}
            \end{equation*}

            esto equivale a decir que existe un $k \in \mathbb{Z}$, tal que $n = k O(a)$, entonces podemos decir:

            \begin{equation*}
                a^n = a^{k O(a)} = \left( a^{O(a)} \right)^k = e^k = e \quad \forall \+ a \in G
            \end{equation*}
        \end{proof}

%-------------------------------------------------------------------------------

    \subsection{Subgrupo Normal}

        \begin{definicion}
            Un grupo $N$ de $G$ se dice que es un subgrupo normal de $G$ denotado por $N \bigtriangleup G$, si para todo $g \in G$ y para todo $n \in N$ se tiene que:

            \begin{equation}
                g n g^{-1} \in N
            \end{equation}
        \end{definicion}

        \begin{lema}
            $N$ es un subgrupo normal de $G$ si y solo si:

            \begin{equation}
                gNg^{-1} = N \quad \forall \+ g \in G
            \end{equation}
        \end{lema}

        \begin{proof}
            Si $gNg^{-1} = N \quad \forall \+ g \in G$, entonces en particular tenemos que:

            \begin{equation*}
                gNg^{-1} \subseteq N
            \end{equation*}

            por lo que se tiene que $gng^{-1} \in N \quad \forall \+ n \in N$, por lo tanto  $N \bigtriangleup G$.

            Por otro lado, si $N$ es un subgrupo normal de $G$, entonces tenemos que:

            \begin{equation*}
                gng^{-1} \in N
            \end{equation*}

            para todo $g \in G$ y para todo $n \in N$, esto implica que:

            \begin{equation*}
                gNg^{-1} \subseteq N
            \end{equation*}

            Por ultimo, podemos ver que $g^{-1} N g = g^{-1} N \left( g^{-1} \right)^{-1} \subseteq N$, ademas:

            \begin{equation*}
                N = e N e = g \left( g^{-1} N g \right) g^{-1} \subseteq g N g^{-1}
            \end{equation*}

            por lo tanto, podemos concluir que $gNg^{-1} = N$
        \end{proof}

        \begin{lema}
            El subgrupo $N$ de $G$, es un subgrupo normal de $G$ ($N \bigtriangleup G$), si y solo si toda clase lateral izquierda de $N$ en $G$ es una clase lateral derecha de $N$ en $G$.
        \end{lema}

        \begin{proof}
            Sea $aH = \left\{ ah \mid h \in H \right\}$ la clase lateral izquierda de $H$.

            Si $N$ es un subgrupo normal de $G$, para todo $g \in G$ y para todo $n \in N$, tenemos que:

            \begin{equation*}
                gNg^{-1} = N
            \end{equation*}

            entonces tenemos que:

            \begin{equation*}
                gN = gNe = gN \left( g^{1} g \right) = \left( gNg^{-1} \right) g = N g
            \end{equation*}

            por lo que toda clase lateral izquierda coincide con la clase lateral derecha.

            Por otro lado, si ahora suponemos que las clases laterales coinciden, entonces:

            \begin{equation*}
                gNg^{-1} = \left( gN \right) g^{-1} = Ng g^{-1} = N
            \end{equation*}

            por lo que podemos concluir que se trata de un subgrupo normal.
        \end{proof}

        \begin{definicion}
            Denotamos $\sfrac{G}{N}$ a la colección de las clases laterales derechas de $N$ en $G$.

            \begin{equation}
                \sfrac{G}{N} = \left\{ Na \mid a \in G \right\}
            \end{equation}
        \end{definicion}

        \begin{teorema}
            Si $G$ es un grupo y $N$ es un subgrupo normal de $G$, entonces $\sfrac{G}{N}$ es tambien un grupo y se denomina grupo cociente.
        \end{teorema}

        \begin{proof}
            Para demostrar la existencia de identidad primero verificamos que para un elemento $x \in \sfrac{G}{N}$, el elemento tiene la forma $x = Na \quad a \in G$, por lo que podemos ver que:

            \begin{eqnarray*}
                xNe = xN = Na N = NNa = Na & = & x \\
                Nex = Nx = NNa = Na & = & x
            \end{eqnarray*}
        \end{proof}

        Para demostrar la existencia de un inverso definimos un elemento $x \in \sfrac{G}{N}$ y $Na^{-1} \in \sfrac{G}{N}$, y queremos demostrar que $x^{-1} = Na^{-1}$ es el inverso de $x = Na$. Al operar estos elementos por la derecha y la izquierda tenemos:

        \begin{eqnarray*}
            x x^{-1} = Na Na^{-1} = NNaa^{-1} = Ne & = & N \\
            x^{-1} x = Na^{-1} Na = NNa^{-1}a = Ne & = & N
        \end{eqnarray*}

        por lo tanto $Na^{-1} = x^{-1}$ es el inverso de $x$.
        Por lo tanto, $\sfrac{G}{N}$ es un grupo.

%-------------------------------------------------------------------------------

    \subsection{Homomorfismos de grupo}

        \begin{definicion}
            Sea una aplicación $\varphi \colon G \to \bar{G}$, $G$ un grupo con operacion $\opone$ y $\bar{G}$ un grupo con operación $\optwo$. Se dice que $\varphi$ es un homomorfismo si para $a, b \in G$ cualesquiera se tiene que:

            \begin{equation}
                \varphi(a \opone b) = \varphi(a) \optwo \varphi(b)
            \end{equation}
        \end{definicion}

        \begin{ejemplo}
            Sea $G = \mathbb{R}^+ \setminus \{0\}$, bajo el producto usual y sea $\bar{G} = \mathbb{R}$ bajo la adición, definimos $\varphi \colon G \to \bar{G}$ como:

            \begin{eqnarray*}
                \varphi \colon \mathbb{R}^+ \setminus \{0\} & \to & \mathbb{R} \\
                r & \to & \ln{r}
            \end{eqnarray*}

            Sean $r_1, r_2 \in \mathbb{R}^+ \setminus \{0\}$ tal que:

            \begin{equation*}
                \varphi(r_1 \cdot r_2) = \ln{(r_1 \cdot r_2)} = \ln{r_1} + \ln{r_2} = \varphi(r_1) + \varphi(r_2)
            \end{equation*}

            por lo que podemos asegurar que $\varphi$ es un homomorfismo.
        \end{ejemplo}

        \begin{lema}
            Supongamos que $G$ es un grupo y que $N$ es un subgrupo normal de $G$. Definamos la aplicación:

            \begin{eqnarray*}
                \varphi \colon G & \to & \sfrac{G}{N} \\
                x & \to & Nx
            \end{eqnarray*}

            entonces $\varphi$ es es un homomorfismo.
        \end{lema}

        \begin{proof}
            Sean $x, y \in G$, entonces $\varphi(x) = Nx$ y $\varphi(y) = Ny$, por lo que podemos ver que:

            \begin{equation*}
                \varphi(x \opone y) = Nxy = NNxy = NxNy = \varphi(x) \optwo \varphi(y)
            \end{equation*}

            por lo que $\varphi$ es un homomorfismo.
        \end{proof}

        \begin{definicion}
            Si $\varphi$ es un homomorfismo de $G$ en $\bar{G}$, el nucleo de $\varphi$, denominado $\ker{\varphi}$ se define:

            \begin{equation}
                \ker{\varphi} = \left\{ x \in G \mid \varphi(x) = \bar{e} \right\}
            \end{equation}

            donde $\bar{e}$ es la identidad de $\bar{G}$.
        \end{definicion}

        \begin{lema}
            Si $\varphi$ es un homomorfismo de $G$ es $\bar{G}$, entonces:

            \begin{enumerate}
                \item $\varphi(e) = \bar{e}$
                \item $\varphi(x^{-1}) = \varphi(x)^{-1} \quad \forall \+ x \in G$
            \end{enumerate}
        \end{lema}

        \begin{proof}
            Para demostrar la primera parte tenemos un elemento $x \in G$, por lo que:

            \begin{equation*}
                \varphi(x) \optwo \bar{e} = \varphi(x) = \varphi(x \opone e) = \varphi(x) \optwo \varphi(e)
            \end{equation*}

            por lo que $\bar{e} = \varphi(e)$

            Para demostrar la segunda parte, notamos que:

            \begin{eqnarray*}
                \bar{e} = \varphi(e) & = & \varphi(x \opone x^{-1}) = \varphi(x) \optwo \varphi(x^{-1}) = \bar{e} \\
                & = & \varphi(x^{-1} \opone x) = \varphi(x^{-1}) \optwo \varphi(x) = \bar{e}
            \end{eqnarray*}

            por lo que $\varphi(x)^{-1} = \varphi(x^{-1})$.
        \end{proof}

        \marginnote{De nuevo, se dejará la notación de $\opone$ y $\optwo$ para los operadores de grupos, sin que por eso se entienda que la operación es la misma en ambos grupos, es decir, se deberá entender por el contexto, la operación indicada.}

        \begin{ejercicio}
            Sea $G$ un grupo abeliano, tenemos que:

            \begin{eqnarray*}
                \varphi \colon G & \to & G \\
                a & \to & a^2
            \end{eqnarray*}

            Verificar que $\varphi$ es un homomorfismo.
        \end{ejercicio}

        \begin{ejercicio}
            Sea $G$ y $G'$ dos grupos y sea $e'$ la identidad en $G'$, entonces:

            \begin{eqnarray*}
                \varphi \colon G & \to & G' \\
                g & \to & e'
            \end{eqnarray*}

            verificar que $\varphi$ rd un homomorfismo.
        \end{ejercicio}

        \begin{ejercicio}
            La identidad dada por:

            \begin{eqnarray*}
                \identidad_G \colon G & \to & G \\
                g & \to & g
            \end{eqnarray*}

            verificar que $\identidad_G$ es un homomorfismo.
        \end{ejercicio}

        \begin{ejercicio}
            Sea $G = \mathbb{Z}$ con la suma usual y $G' = \left\{ 1, -1 \right\}$ con el producto usual. Si definimos:

            \begin{eqnarray*}
                \varphi \colon \mathbb{Z} & \to & \left\{ 1, -1 \right\} \\
                n & \to &
                \begin{cases}
                    1 & \text{si } n \text{ es par} \\
                    -1 & \text{si } n \text{ es impar}
                \end{cases} \quad \forall \+ n \in G
            \end{eqnarray*}

            ¿Será $\varphi$ un homomorfismo?
        \end{ejercicio}

        \begin{ejercicio}
            Sean $G = \mathbb{C}^* = \mathbb{C} \setminus \{0\}$ con el producto correspondiente y $G' = \mathbb{R}^+$ con el producto correspondiente. Entonces definimos:

            \begin{eqnarray*}
                \varphi \colon \mathbb{C}^* & \to & \mathbb{R}^+ \\
                z & \to & |z|
            \end{eqnarray*}

            ¿Será $\varphi$ un homomorfismo?
        \end{ejercicio}

        \begin{ejercicio}
            Definimos:

            \begin{eqnarray*}
                \varphi \colon \mathbb{Z} & \to & \mathbb{Z}_n \\
                a & \to & [a]
            \end{eqnarray*}

            y sabemos que:

            \begin{equation*}
                [a + b] = [a] + [b]
            \end{equation*}

            ¿Será $\varphi$ un homomorfismo?
        \end{ejercicio}

        \begin{definicion}
            Un homomorfismo $\varphi \colon G \to G'$ se dice que es:

            \begin{description}
                \item[Monomorfismo] si es inyectivo (1 - 1).
                \item[Epimorfismo] si es suprayectivo (sobre).
                \item[Isomorfismo] si es biyectivo (1 - 1 y sobre).
            \end{description}
        \end{definicion}

        \begin{definicion}
            Si $\varphi \colon G \to G'$ es un isomorfismo, decimos que $G$ y $G'$ son isomorfos y escribimos:

            \begin{equation}
                G \cong G'
            \end{equation}
        \end{definicion}

        \begin{proposicion}
            Si $\varphi \colon G \to G'$ es un homomorfismo, entonces:

            \begin{equation}
                \imagen{\varphi} < G'
            \end{equation}

            donde:

            \begin{equation}
                \imagen{\varphi} = \left\{ x \in G, y \in G' \mid \varphi(x) = y \right\} \subset G'
            \end{equation}
        \end{proposicion}

        \begin{proof}
            Para demostrar que $\imagen{\varphi}$ es un subgrupo de $G'$, tenemos que demostrar que esta contenido en $G'$ y que es grupo, pero por definición sabemos que $\imagen{\varphi} \subset G'$, por lo que solo nos queda demostrar que es un grupo.
            Para demostrar que es un grupo, pasamos a verificar la cerradura y la existencia de un inverso.

            Para demostrar la propiedad de cerradura, tomamos dos elementos $y_1, y_2 \in \imagen{\varphi}$ tales que tienen la forma:

            \begin{eqnarray*}
                y_1 = \varphi(x_1) \in G' & \quad & x_1 \in G \\
                y_2 = \varphi(x_2) \in G' & \quad & x_2 \in G
            \end{eqnarray*}

            por lo que al operarlos entre si tenemos:

            \begin{equation*}
                y_1 y_2 = \varphi(x_1) \varphi(x_2) = \varphi(x_1 x_2) = y_1 y_2 \in \imagen{\varphi}
            \end{equation*}

            Por otro lado, para demostrar la existencia de un inverso, operamos un elemento $y = \varphi(x) \in \imagen{\varphi}$ con el elemento $\varphi(x^{-1})$, el cual queremos demostrar es el inverso.

            \begin{eqnarray*}
                \varphi(x) \varphi(x^{-1}) = \varphi(x x^{-1}) = \varphi(e) & = & e' \in G' \\
                \varphi(x^{-1}) \varphi(x) = \varphi(x^{-1} x) = \varphi(e) & = & e' \in G'
            \end{eqnarray*}

            por lo que se concluye que el inverso es:

            \begin{equation*}
                \varphi(x)^{-1} = \varphi(x^{-1})
            \end{equation*}

            y por lo tanto $\imagen{\varphi}$ es subgrupo de $G'$.
        \end{proof}

        \begin{definicion}
            Sea $\varphi \colon G \to G'$ un homomorfismo, el nucleo de $\varphi$ es:

            \begin{equation}
                \ker{\varphi} = \left\{ a \in G \mid \varphi(a) = e' \right\} \subset G
            \end{equation}
        \end{definicion}

        \begin{observacion}
            \begin{equation}
                \ker{\varphi} = \left\{ \varphi^{-1}(e') \mid a = \varphi^{-1}(e'), a \in G \right\}
            \end{equation}

            son las preimagenes de $e'$.
        \end{observacion}

        \begin{proposicion}
            Sea $\varphi \colon G \to G'$ un homomorfismo. Entonces $\varphi$ es un monomorfismo, si y solo si:

            \begin{equation}
                \ker{\varphi} = \{0\}
            \end{equation}

            es decir $e = 0 \in G$.
        \end{proposicion}

        \begin{proof}
            Si suponemos que $\ker{\varphi} = \{0\}$, tenemos que para dos elementos $\varphi(x_1)$ y $\varphi(x_2)$ iguales:

            \begin{eqnarray*}
                \varphi(x_1) & = & \varphi(x_2) \\
                \varphi(x_1) - \varphi(x_2) & = & 0 \\
                \varphi(x_1 - x_2) & = & 0
            \end{eqnarray*}

            por lo que $x_1 - x_2 \in \ker{\varphi}$, lo cual implica que:

            \begin{eqnarray*}
                x_1 - x_2 & = & 0 \\
                x_1 & =  & x_2
            \end{eqnarray*}

            por lo que podemos concluir que $\varphi$ es un monomorfismo.
        \end{proof}

        \begin{teorema}
            Si $\varphi$ es un homomorfismo, entonces se satisface que:

            \begin{enumerate}
                \item $\ker{\varphi} < G$
                \item $a^{-1} \ker{\varphi} a \subseteq \ker{\varphi} \quad \forall \+ a \in G$
            \end{enumerate}
        \end{teorema}

        \begin{proof}
            Sabemos que $\ker{\varphi} \ne \varnothing$, ya que existe un $e \in G$ tal que $\varphi(e) = e'$.
            Por lo que pasamos a comprobar que $\ker{\varphi}$ es un grupo, ya que por definición $\ker{\varphi} \subset G$.
            Para comprobar que $\ker{\varphi}$ definimos dos elementos $x, y \in \ker{\varphi}$, por lo que al operarlos entre si tenemos:

            \begin{equation*}
                \varphi(x y) = \varphi(x) \varphi(y) = e' e' = e'
            \end{equation*}

            por lo que $xy \in \ker{\varphi}$.

            Por otro lado, para un elemento $x \in \ker{\varphi}$, para el cual $\varphi(x) = e'$, tenemos que:

            \begin{equation*}
                \varphi(x^{-1}) = \varphi(x^{-1}) e' = \varphi(x^{-1}) \varphi(x) = \varphi(x^{-1} x) = \varphi(e) = e'
            \end{equation*}

            por lo que podemos ver que existe un inverso $x^{-1} \in \ker{\varphi}$ y por lo tanto conlcuir que $\ker{\varphi} < G$.

            Para comprobar la segunda proposición tomamos un elemento $a \in G$ y un elemento $g \in \ker{\varphi}$, para el cual $\varphi(g) = e'$, por lo que queremos verificar que:

            \begin{equation*}
                a^{-1} g a \in \ker{\varphi} \quad \forall \+ g \in \ker{\varphi}
            \end{equation*}

            Sabemos que cualquier elemento en $\ker{\varphi}$, al evaluarlo en $\varphi$, obtendremos la identidad, por lo que procederemos a tratar de obtenerla:

            \begin{eqnarray*}
                \varphi(a^{-1} g a) & = & \varphi(a^{-1}) \varphi(g) \varphi(a) = \varphi(a^{-1}) e' \varphi(a) \\
                & = & \varphi(a^{-1}) \varphi(a) = \varphi(a^{-1} a) = \varphi(e) = e'
            \end{eqnarray*}

            por lo que podemos concluir que:

            \begin{equation*}
                \varphi(a^{-1} g a) \in \ker{\varphi}
            \end{equation*}
        \end{proof}

        \begin{observacion}
            $\ker{\varphi}$ es un subgrupo normal de $G$.
        \end{observacion}

        \begin{ejercicio}
            Sea $\varphi$ definida como:

            \begin{eqnarray*}
                \varphi \colon \mathbb{Z} & \to & \left\{ -1, 1 \right\} \\
                n & \to &
                \begin{cases}
                    -1 & \text{si } n \text{ es par} \\
                    1 & \text{si } n \text{ es impar}
                    \end{cases} \quad \forall \+ n \in G
            \end{eqnarray*}

            Verificar si $\varphi$ es monomorfismo.
        \end{ejercicio}

        \begin{ejercicio}
            Sea $\varphi$ definida como:

            \begin{eqnarray*}
                \varphi \colon \mathbb{C}^* & \to & \mathbb{C}^* \\
                z & \to & |z|
            \end{eqnarray*}

            ¿Será $\varphi$ un monomorfismo?
        \end{ejercicio}

        \begin{proposicion}
            Sea $G$ un grupo y $N$ un subgrupo normal de $G$.
            Existe un epimorfismo $\varphi \colon G \to \sfrac{G}{N}$ cuyo nucleo es $N$.
        \end{proposicion}

        \begin{proof}
            Sea $\varphi$ definida como:

            \begin{eqnarray*}
                \varphi \colon G & \to & \sfrac{G}{N} \\
                a & \to & [a]
            \end{eqnarray*}

            sean $a, b \in G$ tales que al operarlos tenemos que:

            \begin{equation*}
                \varphi(ab) = [ab] = [a][b] = \varphi(a)\varphi(b)
            \end{equation*}

            por lo que podemos concluir que $\varphi$ es un homomorfismo.
            Ahora, como $\varphi$ es sobre por construcción, sabemos que $\varphi$ es un epimorfismo, es decir, si $[a] \in \sfrac{G}{N}$ existe un $a \in G$ tal que $\varphi(a) = [a]$.
            Si ahora tenemos un elemento $a \in \ker{\varphi}$, esto implica que $\varphi(a) = [e]$, es decir:

            \begin{equation*}
                a \cong e \mod N
            \end{equation*}

            lo cual implica:

            \begin{eqnarray*}
                a e^{-1} & \in & N \\
                a e & \in & N \\
                a & \in & N
            \end{eqnarray*}

            por lo que podemos concluir que $\ker{\varphi} \subseteq N$.
        \end{proof}

%-------------------------------------------------------------------------------

    \subsection{Teoremas de isomorfismos}

        \begin{teorema}
            Sea $\varphi$ definido por:

            \begin{eqnarray*}
                \varphi \colon G & \to & G' \\
                g & \to & \varphi(g)
            \end{eqnarray*}

            un epimorfismo con nucleo $K$, entonces:

            \begin{equation}
                \sfrac{G}{K} \cong G'
            \end{equation}
        \end{teorema}

        \begin{proof}
            Para demostrar que $\sfrac{G}{K}$ y $G'$ son isomorfos, debemos demostrar que existe un isomorfismo entre los dos grupos. Empezamos definiendo un mapeo $\bar{\varphi}$ definido por:

            \begin{eqnarray*}
                \bar{\varphi} \colon \sfrac{G}{K} & \to & G' \\
                Kg & \to & \varphi(g)
            \end{eqnarray*}

            es decir $\bar{\varphi}(Kg) = \varphi(g)$.

            Para demostrar que es un isomorfismo, debemos demostrar que es 1 - 1 y sobre.
            Para demostrar su inyectividad tomamos dos elementos $g_1, g_2 \in G$ y al evaluarlos e igualarlos, tenemos que:

            \begin{equation*}
                \bar{\varphi}(Kg_1) = \bar{\varphi}(Kg_2) \implies \varphi(g_1) = \varphi(g_2)
            \end{equation*}

            Por otro lado, si operamos $g_1 g_2^{-1}$ tenemos que:

            \begin{equation*}
                \varphi(g_1 g_2^{-1}) = \varphi(g_1) \varphi(g_2^{-1}) = \varphi(g_2) \varphi(g_2^{-1}) = \varphi(g_2 g_2^{-1}) = \varphi(e) = e'
            \end{equation*}

            Esto es equivalente a decir que $g_1 g_2^{-1} \in K$, lo cual implica que:

            \begin{eqnarray*}
                g_1 & \cong & g_2 \mod K \\
                g_2 & \cong & g_1 \mod K
            \end{eqnarray*}

            es decir:

            \begin{eqnarray*}
                [g_1] & = & [g_2] \\
                Kg_1 & = & Kg_2
            \end{eqnarray*}

            por lo que podemos concluir que $\bar{\varphi}$ es inyectiva.

            Por otro lado, $\bar{\varphi}$ es sobre por construcción, por lo que podemos afirmar que es una biyección.
        \end{proof}

        \begin{ejercicio}
            Verificar que $\varphi$ es un homomorfismo, es decir:

            \begin{equation*}
                \bar{\varphi}(Kg_1 Kg_2) = \bar{\varphi}(Kg_1) \bar{\varphi}(Kg_2)
            \end{equation*}
        \end{ejercicio}

        \begin{teorema}
            Sea $G$ un grupo y $H < G$ y $N \bigtriangleup G$, entonces:

            \begin{eqnarray}
                HN & < & G \\
                H \cap N & \bigtriangleup & H \\
                N & \bigtriangleup & HN
            \end{eqnarray}

            y ademas:

            \begin{equation}
                \sfrac{HN}{N} \cong \sfrac{H}{H \cap N}
            \end{equation}
        \end{teorema}

        \begin{teorema}
            Sea $G$ un grupo, $N \bigtriangleup G$ y $K < N$ con $K \bigtriangleup G$, entonces:

            \begin{equation}
                \sfrac{\sfrac{G}{K}}{\sfrac{N}{K}} \cong \sfrac{G}{N}
            \end{equation}

            Sea $G_1, G_2, \dots, G_n$ grupos. Su producto directo o externo denotado por:

            \begin{equation}
                G_1 \times G_2 \times \dots \times G_n
            \end{equation}

            es el conjunto de n-adas $(a_1, a_2, \dots, a_n)$, donde cada $a_i \in G_i$ para todo $i \in \mathbb{N}$ y la operación en $G_1 \times G_2 \times \dots \times G_n$ se define componente a componente:

            \begin{equation}
                \left( a_1, a_2, \dots, a_n \right) \left( b_1, b_2, \dots, b_n \right) = \left( a_1 b_1, a_2 b_2, \dots, a_n b_n \right)
            \end{equation}

            Tenemos que $G = G_1 \times G_2 \times \dots \times G_n$ es un grupo, cuyo elemento identidad es $(e_1, e_2, \dots, e_n)$ y el inverso de $(a_1, a_2, \dots, a_n)$ es $(a_1^{-1}, a_2^{-1}, \dots, a_n^{-1})$.
        \end{teorema}

%-------------------------------------------------------------------------------
%	EMPIEZA SECCION
%-------------------------------------------------------------------------------

\newpage
\section{Anillos}

%-------------------------------------------------------------------------------

    \subsection{Definiciones}

        \begin{definicion}
            Un conjunto no vacio $R$ es un anillo si tiene definidas dos operaciones $(+, \cdot)$, tales que se cumplen las siguientes propiedades:

            \begin{description}
                \item[Cerradura] $a + b \in R \quad \forall \+ a, b \in R$
                \item[Asociatividad] $a + (b + c) = (a + b) + c \quad \forall \+ a, b, c \in R$
                \item[Conmutatividad] $a + b = b + a \quad \forall \+ a, b \in R$
                \item[Identidad] $\exists \+ 0 \in R \ni a + 0 = a \quad \forall \+ a \in R$
                \item[Inverso] $\exists \+ b \in R \ni a + b = 0 \quad \forall \+ a \in R$
            \end{description}

            De estas propiedades podemos concluir que $R$ es un grupo abeliano con respecto a $(+)$, pero aun tenemos lo siguiente:

            \begin{description}
                \item[Cerradura] $a \cdot b \in R \quad \forall \+ a, b \in R$
                \item[Asociatividad] $a \cdot (b \cdot c) = (a \cdot b) \cdot c$ 
            \end{description}

            De estas propiedades podemos concluir que $R$ es un semigrupo con respecto a $(\cdot)$ y ademas:

            \begin{description}
                \item[Distributividad] $a \cdot (b + c) = a \cdot b + a \cdot c \quad \forall \+ a, b, c \in R$
            \end{description}
        \end{definicion}

        \begin{definicion}
            Diremos que un anillo $R$ es un anillo con identidad si existe un $1 \in R$ diferente de $0$ tal que:

            \begin{equation}
                a \cdot 1 = 1 \cdot a \quad \forall \+ a \in R
            \end{equation}
        \end{definicion}

        \begin{definicion}
            Un anillo $R$ es un anillo conmutativo si:

            \begin{equation}
                a \cdot b = b \cdot a \forall \+ a, b \in R
            \end{equation}
        \end{definicion}

        \begin{definicion}
            Sea $R$ un anillo y $a \in R$ con $a \ne 0$; diremos que $a$ es divisor de cero si existe un $b \in R$ con $b \ne 0$ tal que:

            \begin{equation}
                a \cdot b = 0
            \end{equation}

            a este se le llama divisor por la derecha, o bien si existe un $c \in R$ con $c \ne 0$ tal que:

            \begin{equation}
                c \cdot a = 0
            \end{equation}

            al que se le llama divisor por la izquierda.
        \end{definicion}

        \begin{definicion}
            Sea $R$ un anillo con identidad. Diremos que $R$ es un anillo con división si para todo $a \in R$ existe un $b \in R$ tal que:

            \begin{equation}
                a \cdot b = b \cdot a = 1
            \end{equation}
        \end{definicion}

        \begin{definicion}
            Un campo es un anillo con división, que además es conmutativo.
            Un campo es un grupo abeliano con respecto a la suma y a la multiplicación.
        \end{definicion}

        \begin{definicion}
            Un anillo conmutativo con identidad es un dominio entero si:

            \begin{equation}
                a \cdot b = 0 \implies a = 0 \text{ o } b = 0
            \end{equation}

            es decir, no existen divisores de cero en el anillo.
        \end{definicion}

        \begin{observacion}
            Si $p$ es primo, entonces $\mathbb{Z}_p$ es campo.
        \end{observacion}

        \begin{ejercicio}
            Sea $\mathcal{M}_{2 \times 2}(\mathbb{R})$ definido como:

            \begin{equation}
                \mathcal{M}_{2 \times 2}(\mathbb{R}) = \left\{ \begin{pmatrix} a & b \\ c & d \end{pmatrix} \mid a, b, c, d \in \mathbb{R} \right\}
            \end{equation}

            Verificar que $\mathcal{M}_{2 \times 2}(\mathbb{R})$ es un anillo con identidad no conmutativo y ademas no es dominio entero.
        \end{ejercicio}

        \begin{proposicion}
            Sea $R$ un anillo y sean $a, b \in R$, entonces se cumplen las siguientes propiedades:

            \begin{enumerate}
                \item $a \cdot 0 = 0 \cdot a = 0$
                \item $a \cdot (-b) = (-a) \cdot b = -(a \cdot b)$
                \item $(-a) \cdot (-b) = (-(-a)) \cdot b = ab$
                \item Si $1 \in R \implies (-1) \cdot a = -a$
            \end{enumerate}
        \end{proposicion}

        \begin{proof}
            Para la verificación de la primer proposición tenemos que:

            \begin{equation*}
                a \cdot 0 + 0 = a \cdot 0 = a \cdot (0 + 0) = a \cdot 0  + a \cdot 0
            \end{equation*}

            y cancelando $a \cdot 0$ de ambos lados:

            \begin{equation*}
                0 = a \cdot 0
            \end{equation*}

            Para la verificación de la segunda proposición tenemos que:

            \begin{equation*}
                a \cdot (-b) + a \cdot b = a \cdot (-b + b) = a \cdot (0) = 0
            \end{equation*}

            si conservamos los extremos de este procedimiento y despejamos el segundo termino del lado izquierdo, tenemos:

            \begin{equation*}
                a \cdot (-b) = - (a \cdot b)
            \end{equation*}

            Para la verificación de la tercer proposición tenemos que:

            \begin{equation*}
                (-a) \cdot (-b) + (-a) \cdot (b) = (-a) \cdot (-b + b) = (-a) \cdot (0) = 0
            \end{equation*}

            de nuevo, despejando el segundo termino del lado izquierdo, tenemos:

            \begin{equation*}
                (-a) \cdot (-b) = -(-a) \cdot (b) = (-(-a)) \cdot (b)
            \end{equation*}

            al operar de nuevo con el mismo termino tenemos:

            \begin{equation*}
                (-(-a)) \cdot (b) - (a) \cdot (b) = (b) \cdot (-a - (-a)) = (b) \cdot (0) = 0
            \end{equation*}

            lo cual nos da que:

            \begin{equation*}
                (-(-a)) \cdot (b) = (a) \cdot (b) = (a \cdot b)
            \end{equation*}

            Para la ultima proposición tenemos que:

            \begin{equation*}
                (-1) \cdot a + (1) \cdot a = a \cdot (1 - 1) = a \cdot (0) = 0 
            \end{equation*}

            por lo que tenemos que:

            \begin{equation*}
                (-1) \cdot a = -(1) \cdot a = -a
            \end{equation*}
        \end{proof}

        \marginnote{Aqui se empezará a omitir la notación para el producto usual, sin que por eso se entienda que nos referimos a otra operación.}

%-------------------------------------------------------------------------------

    \subsection{Homomorfismos de anillo}

        \begin{definicion}
            Una función $\varphi \colon R \to R'$ es un homomorfismo si:

            \begin{eqnarray}
                \varphi(a) + \varphi(b) & = & \varphi(a + b) \\
                \varphi(a) \varphi(b) & = & \varphi(ab)
            \end{eqnarray}
        \end{definicion}

%-------------------------------------------------------------------------------

    \subsection{Ideales}

        \faltante{Falta escribir apunte}

%-------------------------------------------------------------------------------
%	EMPIEZA SECCION
%-------------------------------------------------------------------------------

\newpage
\section{Dominios Enteros}

%-------------------------------------------------------------------------------

    \subsection{Definiciones}

        \faltante{Falta escribir apunte}

%-------------------------------------------------------------------------------

    \subsection{Máximo Común Divisor}

        \faltante{Falta escribir apunte}

%-------------------------------------------------------------------------------

    \subsection{mínimo común multiplo}

        \faltante{Falta escribir apunte}

%-------------------------------------------------------------------------------

    \subsection{Algoritmo de la división de Euclides}

        \faltante{Falta escribir apunte}
