%--------------------------------------
%       CONFIGURACIONES
%--------------------------------------
\documentclass{tufte-book}
\usepackage[spanish,es-noquoting]{babel}
\usepackage[utf8]{inputenc}
\usepackage{amsmath}
\usepackage{amsthm}
\usepackage{amssymb}
\usepackage{graphicx}
\usepackage{bbm}
\usepackage{enumerate}
\usepackage{tikz}
\usepackage{todonotes}

\usetikzlibrary{shapes, arrows, decorations.markings}

\setcounter{secnumdepth}{0}
\numberwithin{equation}{chapter}

%--------------------------------------
%       TITULO
%--------------------------------------

\title{Control\\Laboratorio}
\author{Generación 2014}


%--------------------------------------
%       INICIO DEL DOCUMENTO
%--------------------------------------
\begin{document}

 \maketitle

\section{Ejercicios adicionales 1}
		\begin{equation}
	G_{(s)} = \frac{b}{s+a}
	\end{equation}

1. Calcular el error en estado estacionario si (1) es exponecialmente estable y si se regula en lazo cerrado con un cotrol proporciaonal y la referencia es un escal\`on unitario. Como se podria reducir el error? 

		\begin{equation} 
		E_{(s)} = \frac{1}{ 1 + G_{(s)}} \nonumber  
		\end{equation}	
		
		\begin{equation}
		\lim_{s \rightarrow 0 } \ s \ E_{(s)} \ R_{(s)}	\nonumber
 \end{equation}
		\begin{equation}
		E_{(s)} = \frac{s+a}{s+a+b} \nonumber
			\end{equation}

	
			\begin{equation}
			\lim_{s\rightarrow 0} s \ \frac{s+a}{(s+a+b)s} = \frac{s+a}{s+a+b} = \frac{a}{a+b} \nonumber
\end{equation}
\\
Como se podria reducir el error ? \\ Aumentado \emph{b} asi aumenta en denominador  esto reducira en error.
\\

\begin{figure}[htp]
\centering
\includegraphics[scale=0.2]{/home/satex/Documents/Matlab/p1P.jpg}
\includegraphics[scale=0.2]{/home/satex/Documents/Matlab/piG.jpg}
\end{figure}

En la imagen de lado izquierdo \emph{d=0.01} y del lado derecho \emph{d=100}

2. Consid\`erese le sistema (1) en lazo cerrado con un controlador  Proporcional. El polo tiene un valor \emph{s=5} y la ganacia \emph{b=10}. Calcular el rango de los valores de ganancia proporcional para los cuales en lazo cerrado es exponecialmente estable. 

\begin{center}Por Criterio de estabilidad de Routh-Horwitz\end{center}

		\begin{table}[htbp]
	\centering
	\begin{tabular}{c|c c}
	$s^1$ & $1$ & $0$\\
	$s^0$ & $a+b$ &  \\
	\end{tabular}
	\end{table} 
Simpre va a ser estable para valores \emph{b$>$5} \\


3. Sea el sistema (1) con \emph{a=1} y \emph{b=10}. Utilizando un control Proporcional Integral (PI) calcular sus ganancias para que el sistema en lazo cerrado tenga los polos en \emph{s=$-$20} 

	\begin{equation}
	 	(k_p  + \frac{k_i}{s}) * \frac{10}{s+1}= \frac{s k_p + K_i}{s(s+1)+k_p s + k_i} \nonumber
	\end{equation}

 Entonces en lazo cerrado:
 \begin{equation}
    \frac{ k_p + k_i }{ s^2 + ( 1 + 10 k_p )s + 10 k_i } \nonumber	
\end{equation} 
Queremos polos repetidos en $-20$

		\begin{equation}
			(s + 20)(s + 20)= s^2 + 40 + 400 \nonumber
		\end{equation}

		Igualado con el polinomio de la planta en lazo cerrado

\begin{equation}
 			1 + 10 K_p = 40 \nonumber
\end{equation}
		\begin{equation}
			10 k_p = 39 \nonumber
		\end{equation}
				
		\begin{equation}
			k_p = 3.9 \nonumber
\end{equation}
 \\ por otro lado
\begin{equation}
	10 k_i = 400 \nonumber
\end{equation} 

\begin{equation}
		K_i = 40 \nonumber
\end{equation}




%--------------------------------------
%       FIN DEL DOCUMENTO
%--------------------------------------
\end{document}