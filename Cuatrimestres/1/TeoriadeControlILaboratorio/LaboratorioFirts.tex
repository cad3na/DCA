\documentclass[11pt]{article}

\title{\textbf{Control Autom\`atico}}
\author{Christian A. Elizalde \\ Laboratorio}
\date{}
\usepackage{graphicx}
\begin{document}

\maketitle

\section{Ejercicios adicionales 1}
		\begin{equation}
	G_{(s)} = \frac{b}{s+a}
	\end{equation}

1. Calcular el error en estado estacionario si (1) es exponecialmente estable y si se regula en lazo cerrado con un cotrol proporciaonal y la referencia es un escal\`on unitario. Como se podria reducir el error? 

		\begin{equation} 
		E_{(s)} = \frac{1}{ 1 + G_{(s)}} \nonumber  
		\end{equation}	
		
		\begin{equation}
		\lim_{s \rightarrow 0 } \ s \ E_{(s)} \ R_{(s)}	\nonumber
 \end{equation}
		\begin{equation}
		E_{(s)} = \frac{s+a}{s+a+b} \nonumber
			\end{equation}

	
			\begin{equation}
			\lim_{s\rightarrow 0} s \ \frac{s+a}{(s+a+b)s} = \frac{s+a}{s+a+b} = \frac{1}{b} \nonumber
\end{equation}
\\
Como se podria reducir el error ? \\ Aumentado \emph{b} asi aumenta en denominador  esto reducira en error.
\\

\begin{figure}[htp]
\centering
\includegraphics[scale=0.2]{/home/satex/Documents/Matlab/p1P.jpg}
\includegraphics[scale=0.2]{/home/satex/Documents/Matlab/piG.jpg}
\end{figure}

En la imagen de lado izquierdo \emph{d=0.01} y del lado derecho \emph{d=100}
\newpage
2. Consid\`erese le sistema (1) en lazo cerrado con un controlador  Proporcional. El polo tiene un valor \emph{s=5} y la ganacia \emph{b=10}. Calcular el rango de los valores de ganancia proporcional para los cuales en lazo cerrado es exponecialmente estable. 

\begin{center}Por Criterio de estabilidad de Routh-Horwitz\end{center}

		\begin{table}[htbp]
	\centering
	\begin{tabular}{c|c c}
	$s^1$ & $1$ & $0$\\
	$s^0$ & $a+b$ &  \\
	\end{tabular}
	\end{table} 
Simpre va a ser estable para valores \emph{b$>$0}
\\
\\
3. Sea el sistema (1) con \emph{a=1} y \emph{b=10}. Utilizando un control Proporcional Integral (PI) calcular sus ganancias para que el sistema en lazo cerrado tenga los polos en \emph{s=$-$20} 
\end{document}
	
