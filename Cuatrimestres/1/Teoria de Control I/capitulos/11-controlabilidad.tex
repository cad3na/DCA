
\chapter{Controlabilidad y asignación de polos}

    Sea un sistema para la Ecuación Diferencial Ordinaria:

    \begin{equation}
        M \left(\frac{d}{dt} \right) y(t) = N \left(\frac{d}{dt} \right) u(t)
    \end{equation}

    Sea la siguiente representación de estado de esta Ecuación Diferencial Ordinaria:

    \begin{eqnarray}
    \frac{d}{dt} x & = & A x + b u \nonumber \\
    y & = & c x + d u \nonumber
    \end{eqnarray}

    donde $x \in \mathbbm{R}^n$ y $u,y \in \mathbbm{R}$.

    Problema. Se desea encontrar una ley de control $u = f(x)$, que nos permita asignar los polos a voluntad.

    Sabemos que:

    \begin{description}
        \item [Polos.] $\left\{ s \in \mathbbm{C} \mid M(s) = 0 \right\} = \left\{ s \in \mathbbm{C} \mid \det{(sI - A)} \right\}$
    \end{description}

    Para resolver este problema hay que investigar el concepto estructural de la alcanzabilidad.

    \section{Alcanzabilidad y Controlabilidad}

        \begin{itemize}
            \item Una representación de estado se dice controlable, si para cualquier condición inicial, $x(0) = x_0 \in \mathbbm{R}^n$, existe una trayectoria, $x(\cdot)$, solución de la ecuación de estado, tal que en tiempo finito $t_f \in \mathbbm{R}$ se llega al origen $\left( x(t_f) = 0 \right)$.
            \item Una representación de estado se dice alcanzable, si para cualquier punto $x \in \mathbbm{R}$, existe una trayectoria, $x(\cdot)$, solución de la ecuación de estado, tal que en tiempo finito $t_f \in \mathbbm{R}$ se llega a un punto cualquiera $\left( x(t_f) = x_f \right)$ desde el origen.
        \end{itemize}

        En los sistemas lineales estas dos propiedades están mutuamente implicadas, por lo que se les trata indistinguiblemente. Pero en general:

        \begin{equation}
            \text{Alcanzabilidad} \implies \text{Controlabilidad}
        \end{equation}

        La solución temporal de $\frac{dx}{dt} = Ax + bu$ con $x(0) = 0$ es:

        \begin{equation}
            x(t) = \int_0^t \exp{(A(t - \tau))} b u(\tau) d \tau
        \end{equation}

        del teorema de Cayley-Hamilton se tiene:

        \begin{equation}
            \exp{(A t)} = \sum\limits_{i=0}^{\infty} \frac{1}{i!} A^i t^i = \sum\limits_{i=0}^{n-1} \varphi_i(t) A^i
        \end{equation}

        donde $\varphi_i(t) = \sum\limits_{j=0}^{\infty} \varphi_{ij} t^j$, $\varphi_{ij} \in \mathbbm{R}$, $n \in \mathbbm{N}$, $j \in \mathbbm{Z}^+$. Por lo anterior, tenemos:

        \begin{eqnarray}
        x(t) & = & \sum\limits_{i=0}^{n-1} \psi_i(t) A^i b \nonumber \\
        x(t) & = &
        \begin{pmatrix}
        b & A b & \dots & A^{n-1} b
        \end{pmatrix}
        \begin{pmatrix}
        \psi_0(t) \\
        \psi_1(t) \\
        \vdots \\
        \psi_{n-1}(t)
        \end{pmatrix}
        \end{eqnarray}

        donde $\psi_i(t) = \int_0^t \varphi_i (t - \tau) u(\tau) d \tau$.

        Entonces una condición necesaria para que $x(t_f) = x_f \quad \forall x \in \mathbbm{R}, \forall t_f \in \mathbbm{R}, t_f > 0$, es que la matriz de controlabilidad

        \begin{equation}
            C_{(A,b)} =
            \begin{pmatrix}
            b & Ab & \dots & A^{n-1}b
            \end{pmatrix}
        \end{equation}

        sea de rango pleno por filas, de lo contrario existen componentes de $x(t)$ que siempre seran nulos. En nuestro caso particular $\left( y, u \in \mathbbm{R}^n \right)$:

        \begin{equation}
            \det{C_{(A,b)}} \ne 0
        \end{equation}

        Si la matriz de controlabilidad $C_{(A,b)}$ es de rango pleno por filas, entonces el gramiano de controlabilidad es invertible. \footnote{Aquí se esta abusando de la notación, ya que el gramiano de controlabilidad corresponde al caso en que $t \to \infty$}

        \begin{equation}
            W = \int_0^t \exp{(A \sigma)} b b^t \exp{(A^t \sigma)} d\sigma \quad t > 0, \sigma = t - \tau
        \end{equation}

        Entonces, con la siguiente ley de control se tiene:

        \begin{equation}
            u(t) = b^t \exp{(A^t(t_f - t))} W_{t_f}^{-1} x_f
        \end{equation}

        Por lo que si sustituimos $t_f$ en la solución para $x(t)$:

        \begin{multline}
            x(t_f) = \int_0^{t_f} \exp{(A(t - \tau))} b u(\tau) d\tau \\
                   = \int_0^{t_f} \exp{(A(t - \tau))} b b^t \exp{(A^t(t_f - \tau))} W_{t_f}^{-1} x_f d\tau \\
                   = \int_{t_f}^0 \exp{(A \sigma)} b b^t \exp{(A^t \sigma)} d\sigma W_{t_f}^{-1} x_f \\
                   = W_{t_f} W_{t_f}^{-1} x_f = x_f
        \end{multline}

        \begin{itemize}
            \item Por lo que una condición suficiente y necesaria para que la ecuación de estado sea alcanzable (y por lo tanto controlable), es que su matriz de controlabilidad, $C_{(A,b)}$, sea de rango pleno por filas.
            \item Cuando la matriz de controlabilidad es de rango pleno por filas, se dice que el par $(A, b)$ es controlable.
        \end{itemize}

    \section{Asignación de polos}

        Sea la ecuación de estado controlable, es decir $\frac{dx}{dt} = Ax + bu$ con $b \ne 0$, $\det{(C_{(A,b)})} \ne 0$, $\Pi(s)$ y $\alpha(s)$ los polinomios característico y mínimo de $A$ respectivamente.

        \begin{equation}
            \Pi(s) = \det{(sI - A)} \quad \text{grado } \Pi(s) = n \nonumber
        \end{equation}

        \begin{equation}
            \alpha(s) \text{ es el polinomio de menor grado tal que } \alpha(A) = 0 \nonumber
        \end{equation}

        Sea $\kappa = \text{grado } \alpha(s)$, donde obviamente $1 \le \kappa \le n$.

        \begin{equation}
            \alpha(s) = s^{\kappa} + (a_{\kappa} + a_{\kappa - 1} s + \dots + a_1 s^{\kappa - 1}) \nonumber
        \end{equation}

        \begin{equation}
            \alpha(A) = A^{\kappa} + (a_{\kappa} + a_{\kappa - 1} A + \dots + a_1 A^{\kappa - 1}) \nonumber
        \end{equation}

        Sean $\alpha_i$ con $i \in \{ 0, 1, \dots, \kappa \}$, los polinomios mónicos auxiliares tales que:

        \begin{eqnarray}
        \alpha_0(s) & = & \alpha(s) \nonumber \\
        \alpha_1(s) & = & s^{\kappa - 1} + (a_{\kappa - 1} + a_{\kappa - 2} s + \dots + a_1 s^{\kappa - 2}) \nonumber \\
        \vdots \nonumber \\
        \alpha_{\kappa - 1}(s) & = & s + a_1 \nonumber \\
        \alpha_{\kappa}(s) & = & 1 \nonumber
        \end{eqnarray}

        en donde, por definición, $\alpha_1(A) \ne 0$ y $\alpha_0(A) = 0$.

        Sea $b \ne 0$, un vector en $\mathbbm{R}^n$ tal que su polinomio mínimo coincide con $\alpha(s)$.

        \begin{eqnarray}
        \alpha_i(A) b & \ne & 0 \quad i \in \{ 1, 2, \dots, \kappa \} \nonumber \\
        \alpha_0(A) b & = & 0 \nonumber
        \end{eqnarray}

        \begin{equation}
            \vdots \nonumber
        \end{equation}

        \begin{eqnarray}
        (A^{\kappa - 1} + (a_{\kappa - 1} + a_{\kappa - 2} A + \dots + a_1 A^{\kappa - 2})) b & \ne & 0 \nonumber \\
        \alpha_0(A) b & = & 0 \nonumber
        \end{eqnarray}

        \begin{equation}
            \vdots \nonumber
        \end{equation}

        \begin{eqnarray}
        \begin{pmatrix}
        b & Ab & \dots & A^{\kappa - 1} b
        \end{pmatrix}
        \begin{pmatrix}
        a_{\kappa - 1} \\
        a_{\kappa - 2} \\
        \vdots \\
        a_{1}
        \end{pmatrix} & \ne & 0 \nonumber \\
        \alpha_0(A) b & = & 0 \nonumber
        \end{eqnarray}

        Suponga que el par $(A,b)$ es controlable, por lo tanto $\det{C_{(A,b)}} \ne 0$, entonces:

        \begin{equation}
            \begin{pmatrix}
            b & A b & \dots & A^{n-1} b
            \end{pmatrix} v \ne 0 \quad \forall v \ne 0 \quad \therefore \kappa = n
        \end{equation}

        Por lo que el polinomio mínimo y el polinomio característico coinciden cuando el par $(A,b)$ es controlable. Definimos la base:

        \begin{eqnarray}
        e_n & = & \alpha_n(A) b = b \nonumber \\
        e_{n-1} & = & \alpha_{n-1}(A) b = (A + a_1I) b = A e_n + a_1 e_n \nonumber \\
        e_{n-2} & = & \alpha_{n-2}(A) b = (A^2 + (a_2I + a_1A)) b = A(A+a_1I) b + a_2 b = A e_{n-1} + a_2 e_n \nonumber \\
        \vdots & = & \vdots \nonumber \\
        e_1 & = & \alpha_1(A) b = A e_2 + a_{n-1} e_n
        \end{eqnarray}

        Note que sustituyendo $A$, tenemos:

        \begin{equation}
            \alpha(A) b = (A^n + (a_nI + a_{n-1}A + \dots + a_1 A^{n-1})) b = A e_1 + a_n e_n = 0
        \end{equation}

        Por lo que:

        \begin{eqnarray}
        e_n & = & b \nonumber \\
        A e_{n} & = & e_{n-1} + a_1 e_n \nonumber \\
        A e_{n-1} & = & e_{n-2} + a_2 e_n \nonumber \\
        \vdots & = & \vdots \nonumber \\
        A e_2 & = & e_1 + a_{n-1} e_n \nonumber \\
        A e_1 & = & - a_{n} e_n
        \end{eqnarray}

        Entonces, bajo la base definida, las transformaciones lineales tienen la siguiente forma:

        \begin{equation}
            A_c = \left[ A \right]_{\{ e_1, e_2, \dots, e_n \}} =
            \begin{pmatrix}
            0 & 1 & 0 & \dots & 0 & 0 & 0 \\
            0 & 0 & 1 & \dots & 0 & 0 & 0 \\
            0 & 0 & 0 & \dots & 0 & 0 & 0 \\
            \vdots & \vdots & \vdots & & \vdots & \vdots & \vdots \\
            0 & 0 & 0 & \dots & 0 & 1 & 0 \\
            0 & 0 & 0 & \dots & 0 & 0 & 1 \\
            -a_{n} & -a_{n-1} & -a_{n-2} & \dots & -a_{3} & -a_{2} & -a_{1}
            \end{pmatrix}
        \end{equation}

        \begin{equation}
            b_c = \left[ b \right]_{\{ e_1, e_2, \dots, e_n \}} =
            \begin{pmatrix}
            0 \\
            0 \\
            0 \\
            \vdots \\
            0 \\
            0 \\
            1
            \end{pmatrix}
        \end{equation}

        Por lo tanto:

        \begin{equation}
            \frac{d}{dt} x_c =
            \begin{pmatrix}
            0 & 1 & 0 & \dots & 0 & 0 & 0 \\
            0 & 0 & 1 & \dots & 0 & 0 & 0 \\
            0 & 0 & 0 & \dots & 0 & 0 & 0 \\
            \vdots & \vdots & \vdots & & \vdots & \vdots & \vdots \\
            0 & 0 & 0 & \dots & 0 & 1 & 0 \\
            0 & 0 & 0 & \dots & 0 & 0 & 1 \\
            -a_{n} & -a_{n-1} & -a_{n-2} & \dots & -a_{3} & -a_{2} & -a_{1}
            \end{pmatrix} x_c +
            \begin{pmatrix}
            0 \\
            0 \\
            0 \\
            \vdots \\
            0 \\
            0 \\
            1
            \end{pmatrix} u
        \end{equation}

        \begin{description}
            \item [Observaciones.] \mbox{}\\
            \begin{enumerate}
                \item Polinomio Característico

                \begin{equation}
                    \det{(sI - A_c)} = s^n + a_1 s^{n-1} + \dots + a_{n-1} s + a_n = \Pi(s) \nonumber
                \end{equation}

                \item Retroalimentación de Estado

                Sea $u = f_c x_c + v$, donde $f_c = (a_n - \bar{a}_n)(a_{n-1} - \bar{a}_{n-1})\dots(a_1 - \bar{a}_1)$. Entonces el sistema de lazo cerrado es:

                \begin{equation}
                    \frac{d}{dt} x_c = A_{f_c} x_c + v
                \end{equation}

                donde $A_{f_c} = A_c + b_c f_c$, es decir:

                \begin{equation}
                    A_{f_c} =
                    \begin{pmatrix}
                    0 & 1 & 0 & \dots & 0 & 0 & 0 \\
                    0 & 0 & 1 & \dots & 0 & 0 & 0 \\
                    0 & 0 & 0 & \dots & 0 & 0 & 0 \\
                    \vdots & \vdots & \vdots & & \vdots & \vdots & \vdots \\
                    0 & 0 & 0 & \dots & 0 & 1 & 0 \\
                    0 & 0 & 0 & \dots & 0 & 0 & 1 \\
                    -\bar{a}_{n} & -\bar{a}_{n-1} & -\bar{a}_{n-2} & \dots & -\bar{a}_{3} & -\bar{a}_{2} & -\bar{a}_{1}
                    \end{pmatrix}
                \end{equation}

                y su polinomio característico es $\det{(sI - A_{f_c})} = s^n + \bar{a}_1 s^{n-1} + \dots + \bar{a}_{n-1} s + \bar{a}_n$.

            \end{enumerate}
        \end{description}

    \section{Propiedades de la matriz de controlabilidad}

        \begin{enumerate}
            \item Matriz de controlabilidad del par $(A_c, b_c)$.

                \begin{equation}
                    C_{(A_c,b_c)} =
                    \begin{pmatrix}
                    0 & 0 & 0 & \dots & 1 \\
                    \vdots & \vdots & \vdots & & \vdots \\
                    0 & 0 & 1 & \dots & * \\
                    0 & 1 & * & \dots & * \\
                    1 & * & * & \dots & *
                    \end{pmatrix}
                \end{equation}

                \begin{equation}
                    \det{C_{(A_c, b_c)}} = \pm 1
                \end{equation}

            \item Invarianza de la matriz de controlabilidad bajo cambio de base

                Sea el par $(A, b)$ controlable, sea $T$ una matriz de cambio de base y sean $A_1 = T^{-1} A T$ y $b_1 = T^{-1} b$ las matrices de nuestra nueva base.

                \begin{multline}
                C_{(A_1, b_1)} =
                \begin{pmatrix}
                b_1 & A_1 b_1 & \dots & A^{n-1} b_1
                \end{pmatrix} = \\
                \begin{pmatrix}
                T^{-1} b & T^{-1} A T T^{-1} b & \dots & (T^{-1} A T \dots T^{-1} A T) T^{-1} b
                \end{pmatrix} = \\
                \begin{pmatrix}
                T^{-1} b & T^{-1} A b & \dots & T^{-1} A^{n-1} b
                \end{pmatrix} = \\
                T^{-1}
                \begin{pmatrix}
                b & A b & \dots & A^{n-1} b
                \end{pmatrix} =
                T^{-1} C_{(A, b)} \nonumber
                \end{multline}

                \begin{equation}
                    T = C_{(A, b)} C_{(A_1, b_1)}^{-1}
                \end{equation}

            \item Invarianza de la matriz de controlabilidad bajo retroalimentación de estado $u = f x + v$

            \item
        \end{enumerate}

    \section{Formas canónicas}
        \subsection{Forma canónica controlador}
        \subsection{Forma canónica controlabilidad}
