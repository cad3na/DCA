%-------------------------------------------------------------------------------
%	EMPIEZA CAPITULO
%-------------------------------------------------------------------------------

\chapter{Nociones de control robusto $H_{\infty}$}

%-------------------------------------------------------------------------------
%	EMPIEZA SECCION
%-------------------------------------------------------------------------------

    \section{Anillo de las fracciones propias Hurwitz estable $RH_{\infty}$}

        Consideremos el siguiente conjunto

        \begin{equation}
            \left\{ F(s) \in \mathbb{R}(s) \mid F(s) = \frac{N(s)}{M(s)} : N(s), M(s) \in \mathbb{R}[s], \lim_{s \to \infty} \frac{N(s)}{M(s)} < \infty, F^{-1}(\bar{s}) = 0 \implies \bar{s} \in \mathbb{C} \setminus \mathbb{C}_+ \right\}
        \end{equation}

        en donde $\mathbb{C}_+ = \left\{ s \in \mathbb{C} \mid \Re{s} \ge 0 \right\}$.

        Es decir, $RH_{\infty}$ es el conjunto de funciones de transferencia propias Hurwitz estable.

        \begin{definicion}
            El conjunto $RH_{\infty}$ forma un anillo conmutativo bajo las operaciones usuales $+$ y $\cdot$.
        \end{definicion}

        \begin{definicion}
            Existe una función grado $\delta: RH_{\infty} \to \mathbb{Z}^+$ tal que si $F(s) \in RH_{\infty} - {0}$, entonces

            \begin{equation}
                \delta(F) = \cardinalidad{\left\{ s \in \mathbb{C}_+ : F(s) = 0 \right\}} + \grado{F(s)}
            \end{equation}

            En donde a $\left\{ s \in \mathbb{C}_+ : F(s) = 0 \right\}$ les llamamos ceros finitos en $\mathbb{C}_+$ y a $F(s)$ les llamamos ceros al infinito.
        \end{definicion}

        Dado que $RH_{\infty}$ es un anillo conmutativo euclidiano podemos aplicar

        \begin{enumerate}
            \item Algoritmo de la división de Euclides
            \item Identidad de Bezout
            \item Ecuación Diofantina
        \end{enumerate}

        Note que, las unidades en este anillo son las fracciones propias de grado relativo cero, Hurwitz estebles, y de fase mínima.

        \begin{definicion}
            La norma de $RH_{\infty}$ esta definida de la siguiente manera

            \begin{equation}
                ||F||_{\infty} = \supremo{|F(j\omega)|} \quad \omega \in \mathbb{R}, F \in RH_{\infty}
            \end{equation}

            $RH_{\infty}$ bajo esta norma forma un espacio de Banach.
        \end{definicion}

        \missingfigure{Diagrama de Bode de funcion de transferencia propia con valor supremo $||F||_{\infty}$.}

        \missingfigure{Diagrama de Nyquist de funcion de transferencia propia con valor supremo $||F||_{\infty}$}

%-------------------------------------------------------------------------------
%	EMPIEZA SECCION
%-------------------------------------------------------------------------------

    \section{Factorización coprima}

        Sea $F(s)$ una función de transferencia propia, es decir

        \begin{equation*}
            F(s) \in \mathbb{R}(s) \quad \lim_{s \to \infty} F(s) < \infty
        \end{equation*}

        Una factorización coprima de $F$ en $RH_{\infty}$, es la descomposición de $F$ en fracciones coprimas en $RH_{\infty}$, es decir

        \begin{equation*}
            F(s) = \frac{N(s)}{M(s)}
        \end{equation*}

        donde $M, N \in RH_{\infty}$, $M \ne 0$.
        Por lo tanto existirán $X, Y \in RH_{\infty}$ tal que

        \begin{equation}
            X(s) M(s) + Y(s) N(s) = 1
        \end{equation}

        Note que los polos no Hurwitz de $F(s)$ pasan a ser ceros no Hurwitz de $M(s)$ y los ceros no Hurwitz de $F(s)$ pasan a ser ceros no Hurwitz de $N(s)$.

        \begin{equation*}
            F(s) = \frac{(s-1)(s+2)}{(s-3)(s+4)} = \frac{\frac{(s-1)(s+2)}{(s+a)^2}}{\frac{(s-3)(s+4)}{(s+a)^2}} \text{ con } a > 0
        \end{equation*}

%-------------------------------------------------------------------------------
%	EMPIEZA SECCION
%-------------------------------------------------------------------------------

    \section{Estabilidad interna}

        Sea el diagrama de bloques de la figura, la representación de un sistema con función de transferencia estrictamente propia, es decir

        \begin{equation*}
            G(s) \in \mathbb{R}[s] \quad \lim_{s \to \infty} G(s) = 0
        \end{equation*}

        y controlado por un controlador representado por una función de transferencia propia, es decir

        \begin{equation*}
            K(s) \in \mathbb{R}[s] \quad \lim_{s \to \infty} K(s) < \infty
        \end{equation*}

        \missingfigure{Diagrama de bloques en donde $u$ es la señal de control, $y$ es la salida y $v_1$ y $v_2$ con señales exógenas (consignas, perturbaciones, ruidos, etc.).}

        Este sistema puede ser representado matricialmente de la siguiente manera

        \begin{equation}
            \begin{pmatrix}
                1 & -K(s) \\
                -G(s) & 1
            \end{pmatrix}
            \begin{pmatrix}
                u \\
                y
            \end{pmatrix} =
            \begin{pmatrix}
                v_1 \\
                v_2
            \end{pmatrix}
        \end{equation}

        es decir

        \begin{equation*}
            m(s)
            \begin{pmatrix}
                u \\
                y
            \end{pmatrix} =
            \begin{pmatrix}
                v_1 \\
                v_2
            \end{pmatrix}
        \end{equation*}

        de donde podemos ver que

        \begin{equation*}
            \det{m(s)} = 1 - G(s) K(s)
        \end{equation*}

        por lo que dadas las condiciones asumidas para $K(s)$ y $G(s)$, tenemos que

        \begin{equation*}
            \lim_{s \to \infty} \det{m(s)} = 1 - 0 \lim_{s \to \infty} K(s) = 1
        \end{equation*}

        por lo que nuestro sistema es invertible, es decir tiene solución dependiente de

        \begin{equation*}
            m^{-1}(s) = \frac{1}{\det{m(s)}}
            \begin{pmatrix}
                1 & K(s) \\
                G(s) & 1
            \end{pmatrix}
        \end{equation*}

        y esta solución es propia para todo $K(s) \in \mathbb{R}(s)$ con $\lim_{s \to \infty} K(s) < \infty$.
        Por lo que las cuatro funciones de transferencia del sistema en lazo cerrado del sistema son propias.

        Diremos que $K(s)$ estabiliza a $G(s)$ si sus cuatro funciones de transferencia son estables, esto es, si

        \begin{equation*}
            m^{-1}(s) \in RH_{\infty}
        \end{equation*}

        A esto le llamamos estabilidad interna.

        Sean $M, N$ y $V, U$ factorizaciones coprimas de $RH_\infty$ de $G(s)$ y $K(s)$ respectivamente, es decir

        \begin{eqnarray} \label{eq:robus0}
            G(s) & = & \frac{N(s)}{M(s)} \quad N, M \in RH_{\infty} \quad (M, N) = 1 \\
            K(s) & = & \frac{U(s)}{V(s)} \quad U, V \in RH_{\infty} \quad (U, V) = 1
        \end{eqnarray}

        Utilizando estas factorizaciones coprimas, el diagrama de bloques de la figura se puede ver como el de la figura.

        \missingfigure{Diagrama de bloques en donde $u$ es la señal de control, $y$ es la salida y $v_1$ y $v_2$ con señales exógenas (consignas, perturbaciones, ruidos, etc.), $\eta$ y $\xi$ son señales intermedias entre los terminos de la factorizaciónes coprimas de $G$ y $K$}

        De la figura tenemos que

        \begin{eqnarray} \label{eq:robus1}
            y & = & N \xi + v_2 \nonumber \\
            M \xi & = & u \nonumber \\
            u & = & U \eta + v_1 \nonumber \\
            V \eta & = & y
        \end{eqnarray}

        De estas ecuaciones se deduce que la estabilidad del par $\xi, \eta$ implica la estabilidad $y, u$.
        Por otro lado, dado que $(M, N) = 1$ sabemos que existen $X, Y \in RH_{\infty}$ tal que

        \begin{equation} \label{eq:robus2}
            X M + Y N = 1
        \end{equation}

        lo cual implica que

        \begin{equation} \label{eq:robus3}
            X M \xi + Y N \xi = \xi
        \end{equation}

        por lo que de la ecuación~\ref{eq:robus1} y la ecuación~\ref{eq:robus3} podemos obtener que

        \begin{eqnarray}
            X u + Y (y - v_2) & = & \xi \nonumber \\
            X u + Y y - Y v_2 & = & \xi
        \end{eqnarray}

        y tenemos que la estabilidad del par $u, y$ implica la estabilidad de $\xi$.

        Similarmente para $\eta$, tenemos que $(U, V) = 1$, por lo que existen $\hat{X}, \hat{Y} \in RH_{\infty}$ tales que

        \begin{eqnarray} \label{eq:robus4}
            \hat{X} V + \hat{Y} U & = & 1 \nonumber \\
            \hat{X} V \eta + \hat{Y} U \eta & = & \eta \nonumber \\
            \hat{X} y + \hat{Y} (u - v_1) & = & 1 \nonumber \\
            \hat{X} y + \hat{Y} u - \hat{Y} v_1 & = & 1
        \end{eqnarray}

        por lo que la estabilidad del par $u, y$ implica la estabilidad de $\eta$.

        Esto es, las cuatro funciones de transferencia de $u, y$ contra $v_1, v_2$ pertenecen a $RH_{\infty}$ si y solamente si las cuatro funciones de transferencia de $\xi, \eta$ contra $v_1, v_2$ pertenecen a $RH_{\infty}$.
        Por lo que podemos ver el siguiente resultado:

        \begin{resultado}
            $K$ estabiliza a $G$ si

            \begin{equation}
                \begin{pmatrix}
                    M & U \\
                    N & V
                \end{pmatrix}^{-1} \in RH_{\infty}
            \end{equation}
        \end{resultado}

        En efecto, notamos primero que

        \begin{equation*}
            \begin{pmatrix}
                M & U \\
                N & V
            \end{pmatrix} =
            \begin{pmatrix}
                1 & K \\
                G & 1
            \end{pmatrix}
            \begin{pmatrix}
                M & 0 \\
                0 & V
            \end{pmatrix}
        \end{equation*}

        puesto que $M \ne 0$, $V \ne 0$ y $\lim_{s \to \infty} G(s) = 0$, tenemos que

        \begin{eqnarray*}
            \lim_{s \to \infty} \det{
            \begin{pmatrix}
                M & U \\
                N & V
            \end{pmatrix}
            } & = & \lim_{s \to \infty} \left( \left( 1 - K(s) G(s) \right) \left( M(s) V(s) \right) \right) \\
            & = & \lim_{s \to \infty} M(s) V(s) \ne 0
        \end{eqnarray*}

        por lo que

        \begin{equation*}
            \begin{pmatrix}
                M & U \\
                N & V
            \end{pmatrix}^{-1} \in \mathbb{R}(s)
        \end{equation*}

        es decir, la inversa de esta matriz existe.
        Notemos ahora de la figura que:

        \begin{equation*}
            \begin{pmatrix}
                M & U \\
                N & V
            \end{pmatrix}
            \begin{pmatrix}
                \xi \\
                - \eta
            \end{pmatrix} =
            \begin{pmatrix}
                v_1 \\
                v_2
            \end{pmatrix}
        \end{equation*}

        Por lo que $K$ estabiliza a $G$ si y solamente si:

        \begin{equation*}
            \begin{pmatrix}
                M & U \\
                N & V
            \end{pmatrix}^{-1} \in RH_{\infty}
        \end{equation*}

    \section{Parametrización de Youla}

        \begin{teorema}
            Sea el sistema de la figura, sean $M, N$ y $V, U$ factorizaciones coprimas en $RH_{\infty}$ de $G(s)$ y $K(s)$ respectivamente.

            Sean $X, Y$ y $\hat{X}, \hat{Y}$, soluciones de las identidades de Bezout en las ecuaciones~\ref{eq:robus2} y~\ref{eq:robus4} respectivamente, entonces el conjunto de todos los controladores propios $K(s) \in \mathbb{R}(s)$ que estabilizan a la planta estan parametrizados de la siguiente manera:

            \begin{equation} \label{eq:robus5}
                K(s) = - \frac{Y(s) - M(s) Q(s)}{X(s) - N(s)Q(s)}
            \end{equation}

            en donde $Q(s) \in RH_{\infty}$.
        \end{teorema}

        \begin{proof}
            De la ecuación~\ref{eq:robus2} se tiene la siguiente igualdad no trivial:

            \begin{equation} \label{eq:robus6}
                \begin{pmatrix}
                    X(s) & Y(s) \\
                    N(s) & -M(s)
                \end{pmatrix}
                \begin{pmatrix}
                    M(s) & Y(s) \\
                    N(s) & -X(s)
                \end{pmatrix} =
                \begin{pmatrix}
                    1 & 0 \\
                    0 & 1
                \end{pmatrix}
            \end{equation}

            lo cual implica que:

            \begin{equation*}
                \begin{pmatrix}
                    1 & Q(s) \\
                    0 & 1
                \end{pmatrix}
                \begin{pmatrix}
                    X(s) & Y(s) \\
                    N(s) & -M(s)
                \end{pmatrix}
                \begin{pmatrix}
                    M(s) & Y(s) \\
                    N(s) & -X(s)
                \end{pmatrix}
                \begin{pmatrix}
                    1 & -Q(s) \\
                    0 & 1
                \end{pmatrix} =
                \begin{pmatrix}
                    1 & Q(s) \\
                    0 & 1
                \end{pmatrix}
                \begin{pmatrix}
                    1 & -Q(s) \\
                    0 & 1
                \end{pmatrix}
            \end{equation*}

            es decir:

            \begin{equation} \label{eq:robus7}
                \begin{pmatrix}
                    X + NQ & Y- MQ \\
                    N & -M
                \end{pmatrix}
                \begin{pmatrix}
                    M & Y - MQ \\
                    N & -(x + NQ)
                \end{pmatrix} = I
            \end{equation}

            definiendo ahora:

            \begin{eqnarray} \label{eq:robus8}
                U & = & Y - MQ \in RH_{\infty} \nonumber \\
                V & = & -(X + NQ) \in RH_{\infty}
            \end{eqnarray}

            la ecuación~\ref{eq:robus7} queda:

            \begin{equation} \label{eq:robus9}
                \begin{pmatrix}
                    -V & U \\
                    N & -M
                \end{pmatrix}
                \begin{pmatrix}
                    M & U \\
                    N & V
                \end{pmatrix} = I
            \end{equation}

            y de la ecuación~\ref{eq:robus9} se concluye que:

            \begin{equation*}
                V(s)(-M(s)) + U(s)N(s) = 1
            \end{equation*}

            por lo que $(U, V) = 1$, tambien se concluye que la inversa de la matriz en la ecuación~\ref{eq:robus9} existe y es:

            \begin{equation}
                \begin{pmatrix}
                    M & U \\
                    N & V
                \end{pmatrix}^{-1} =
                \begin{pmatrix}
                    -V & U \\
                    N & -M
                \end{pmatrix} \in RH_{\infty}
            \end{equation}

            Por lo que del resultado 18.1 se sigue que el controlador $K(s) \in \mathbb{R}(s)$ estabiliza a $G(s)$.

            Note tambien que las ecuaciones ~\ref{eq:robus0},~\ref{eq:robus2} y~\ref{eq:robus5} implican:

            \begin{eqnarray*}
                K(s) & = & \frac{Y - MQ}{X + NQ} =\frac{(Y - MQ)M}{XM + NQM} = - \frac{(Y - MQ)M}{1 - YN + MNQ} \\
                 & = & - \frac{(Y - MQ)M}{1 + (MQ - Y)N} = - \frac{Y - MQ}{\frac{1}{M} + (MQ - Y)G}
            \end{eqnarray*}

            y reocordando la condición de propiedad del controlador, tenemos:

            \begin{equation*}
                \lim_{s \to \infty} K(s) = - \lim_{s \to \infty} (Y - MQ) M < \infty
            \end{equation*}

            es decir $K$ es un controlador propio, que estabiliza a la planta.

            Por otro lado, si definimos $D = NU - VM \in RH_{\infty}$, de la ecuación~\ref{eq:robus2} se obtiene:

            \begin{equation} \label{eq:robus10}
                \begin{pmatrix}
                    X & Y \\
                    N & -M
                \end{pmatrix}
                \begin{pmatrix}
                    M & U \\
                    N & V
                \end{pmatrix} =
                \begin{pmatrix}
                    1 & XU + YV \\
                    0 & D
                \end{pmatrix}
            \end{equation}

            Notemos que la ecuación~\ref{eq:robus6} implica la invertibilidad de la primer matriz y dado que $K$ estabiliza a $G$, sabemos que el producto tendrá inversa en $RH_{\infty}$.

            \begin{equation*}
                \begin{pmatrix}
                    1 & XU + YV \\
                    0 & D
                \end{pmatrix}^{-1} \in RH_{\infty}
            \end{equation*}

            por lo que $D^{-1} \in RH_{\infty}$ y se puede definir:

            \begin{equation}
                Q = - (XU + YV) D^{-1} \in RH_{\infty}
            \end{equation}

            y podemos reescribir la ecuación~\ref{eq:robus10} como:

            \begin{equation} \label{eq:robus11}
                \begin{pmatrix}
                    X & Y \\
                    N & -M
                \end{pmatrix}
                \begin{pmatrix}
                    M & U \\
                    N & V
                \end{pmatrix} =
                \begin{pmatrix}
                    1 & -QD \\
                    0 & D
                \end{pmatrix}
            \end{equation}

            Premultiplicando $\begin{pmatrix} M & Y \\ N & -X \end{pmatrix}$ a la ecuación\label{eq:robus11} y aplicando la identidad de Bezout en los terminos adecuados, obtenemos:

            \begin{equation*}
                \begin{pmatrix}
                    M & Y \\
                    N & -X
                \end{pmatrix}
                \begin{pmatrix}
                    X & Y \\
                    N & -M
                \end{pmatrix}
                \begin{pmatrix}
                    M & U \\
                    N & V
                \end{pmatrix} =
                \begin{pmatrix}
                    M & Y \\
                    N & -X
                \end{pmatrix}
                \begin{pmatrix}
                    1 & -QD \\
                    0 & D
                \end{pmatrix}
            \end{equation*}

            \begin{equation*}
                \begin{pmatrix}
                    M & U \\
                    N & V
                \end{pmatrix} =
                \begin{pmatrix}
                    M & (Y - MQ)D \\
                    N & -(X + NQ)D
                \end{pmatrix}
            \end{equation*}

            lo cual implica que:

            \begin{eqnarray*}
                U & = & (Y - MQ)D \\
                V & = & -(X + NQ)D
            \end{eqnarray*}

            y recordando que, tanto $D$ como $D^{-1}$ estan en $RH_{\infty}$, tenemos que el controlador queda:

            \begin{eqnarray*}
                K(s) = \frac{U(s)}{V(s)} = - \frac{(Y - MQ)D}{(X + NQ)D} = - \frac{Y(s) - M(s) Q(s)}{X(s) + N(s) Q(s)}
            \end{eqnarray*}
        \end{proof}

        Note que la función de transferencia en lazo cerrado es:

        \missingfigure{Diagrama de bloques de planta $G(s) \in RH_{\infty}$ con controlador $K(s) \in RH_{\infty}$.}

        \faltante{Falta desarrollo algebraico}

        La parametrización de Youla nos proporciona el conjunto de controladores estabilizadores de $G$ con un parametro libre $Q \in RH_{\infty}$, que es utilizado para minimizar el indice de desempeño predeterminado.
