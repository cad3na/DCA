
\chapter{Introducción a la optimización de funcionales}

    El problema que tratamos de resolver es el siguiente; determinar $v(t)$, tal que la siguiente ecuación:

    \begin{equation} \label{eq:opfun1}
        J(v(t), v'(t)) = \int_{t_1}^{t_2} f_1(v(t), v'(t), t) dt
    \end{equation}

    sea un extremo.

    Primero definamos una función de vecindad $v(t) \to v(\alpha, t)$ tal que si $\alpha = 0 \implies v^* = v(0, t) = v(t)$, es decir, $v^*$ es la función que extremiza a $J(v(t), v'(t))$.

    Proponemos una solución en forma lineal:

    \begin{equation}
        v(\alpha, t) = v(0, t) + \alpha \eta(t)
    \end{equation}

    pero $v(t)$ y $u(t)$ deben ser idénticos en los puntos extremos.

    \missingfigure{Trayectorias diferentes con inicio y fin comunes.}

    es decir $\eta(t_1) = \eta(t_2) = 0$, pero además $\eta(t) \in e^1$.\footnote{Si $f \in e^1$, $f$ es una función diferenciable al menos una vez.}

    Con esta parametrización en $\alpha$:

    \begin{equation*}
        v(t) \to v(\alpha, t) = v(0, t) + \alpha \eta(t)
    \end{equation*}

    tenemos que la ecuación~\ref{eq:opfun1} nos queda:

    \begin{equation*}
        J(\alpha) = \int_{t_1}^{t_2} f_1(v(\alpha, t), v'(\alpha, t), t) dt
    \end{equation*}

    donde tenemos que $\alpha = 0$ implica que $J$ es un extremo y $\alpha \ne 0$ implica que $J$ no es un extremo.

    Debido a esto, podemos concluir que $J$ tambien esta parametrizada de esta manera:

    \begin{equation*}
        J \to J(\alpha)
    \end{equation*}

    La condición necesaria para que $J$ tenga un valor estacionario (extremo), es que $J$ sea independiente de $\alpha$ en primer orden (que este relacionado linealmente), a lo largo de la trayectoria que otorga el extremo ($\alpha = 0$), es decir:

    \begin{equation}
        \left. \frac{\partial J}{\partial \alpha} \right|_{\alpha=0} = 0 \quad \forall \eta \in e^1
    \end{equation}

    \begin{nota}
        Observe que solo es una condición necesaria, es decir:

        \begin{equation*}
            J \text{ es extremo } \implies \left. \frac{\partial J}{\partial \alpha} \right|_{\alpha=0} = 0
        \end{equation*}
    \end{nota}
    \newpage
    \section{Ecuación de Euler}

    La condición necesaria es:

    \begin{equation*}
        \left. \frac{\partial J}{\partial \alpha} \right|_{\alpha=0} = 0
    \end{equation*}

    entonces hay que seguir los siguientes pasos:

    \begin{enumerate}
        \item Calcular $\frac{\partial J}{\partial \alpha}$.
        \item Hacer $\alpha = 0$.
    \end{enumerate}

    Empecemos calculando la derivada parcial de $J$:

    \begin{multline*}
        \frac{\partial J}{\partial \alpha} = \frac{\partial}{\partial \alpha} \int_{t_1}^{t_2} f_1(v(\alpha, t), v'(\alpha, t), t)dt = \\
        \int_{t_1}^{t_2}\left( \frac{\partial f_1(\dots)}{\partial v(\alpha, t)} \frac{\partial v(\alpha, t)}{d \alpha} + \frac{\partial f_1(\dots)}{\partial v'(\alpha, t)} \frac{\partial v'(\alpha, t)}{d \alpha} \right) dt
    \end{multline*}

    en este punto aparecen términos reducibles:

    \begin{equation*}
        \frac{\partial v(\alpha, t)}{\partial \alpha} = \frac{\partial v(t)}{\partial \alpha} + \frac{\partial \left(\alpha \eta(t) \right)}{\partial \alpha} = \eta(t)
    \end{equation*}

    \begin{equation*}
        \frac{d v'(\alpha, t)}{d\alpha} = \frac{\partial}{\partial \alpha} \left( \frac{d v(\alpha, t)}{dt} \right) = \frac{\partial}{\partial \alpha} \left( v'(t) + \alpha \frac{d \eta(t)}{dt} \right) = \frac{d \eta(t)}{dt}
    \end{equation*}

    lo que nos deja:

    \begin{equation*}
        \frac{\partial J}{\partial \alpha} = \int_{t_1}^{t_2}\left( \frac{\partial f_1(\dots)}{\partial v(\alpha, t)} \eta(t) + \frac{\partial f_1(\dots)}{\partial v'(\alpha, t)} \frac{d \eta(t)}{dt} \right) dt
    \end{equation*}

    la segunda parte de esta integral es integrable por partes, si hacemos $u = \frac{\partial f_1(\dots)}{\partial v'(\alpha, t)}$, $dv = \frac{d\eta(t)}{dt}dt$, $du = \frac{d}{dt}\left( \frac{\partial f_1(\dots)}{dv'(\alpha, t)} \right)$ y $v = \eta(t)$:

    \begin{equation*}
        \frac{\partial J}{\partial \alpha} = \int_{t_1}^{t_2}\left( \frac{\partial f_1(\dots)}{\partial v(\alpha, t)} \eta(t) - \frac{d}{dt} \left( \frac{\partial f_1(\dots)}{\partial v'(\alpha, t)} \right) \eta(t) \right) dt + \left. \frac{\partial f_1(\dots)}{\partial v'(\alpha, t)} \eta(t) \right|_{t_1}^{t_2}
    \end{equation*}

    pero recordemos que $\eta(t_1) = \eta(t_2) = 0$, por lo que el ultimo termino se elimina y nos queda:

    \begin{multline} \label{eq:opfun2}
        \frac{\partial J}{\partial \alpha} = \int_{t_1}^{t_2}\left( \frac{\partial f_1(\dots)}{\partial v(\alpha, t)} \eta(t) - \frac{d}{dt} \left( \frac{\partial f_1(\dots)}{\partial v'(\alpha, t)} \right) \eta(t) \right) dt = \\
        \int_{t_1}^{t_2}\left( \frac{\partial f_1(v(\alpha, t), v'(\alpha, t), t)}{\partial v(\alpha, t)} - \frac{d}{dt} \left( \frac{\partial f_1(v(\alpha, t), v'(\alpha, t), t)}{\partial v'(\alpha, t)} \right) \right) \eta(t) dt
    \end{multline}

    Si ahora, en la ecuación~\ref{eq:opfun2} sustituimos $\alpha = 0$, obtendremos:

    \begin{equation*}
        \left. \frac{\partial J}{\partial \alpha} \right|_{\alpha=0} = \int_{t_1}^{t_2}\left( \frac{\partial f_1(v(t), v'(t), t)}{\partial v(t)} - \frac{d}{dt} \left( \frac{\partial f_1(v(t), v'(t), t)}{\partial v'(t)} \right) \right) \eta(t) dt
    \end{equation*}

    Por lo que la condición necesaria es:

    \begin{equation*}
        \int_{t_1}^{t_2}\left( \frac{\partial f_1(v(t), v'(t), t)}{\partial v(t)} - \frac{d}{dt} \left( \frac{\partial f_1(v(t), v'(t), t)}{\partial v'(t)} \right) \right) \eta(t) dt = 0 \quad \forall \eta(t) \in e^1
    \end{equation*}

    lo cual implica que:

    \begin{equation}
        \frac{\partial f_1(v(t), v'(t), t)}{\partial v(t)} - \frac{d}{dt} \left( \frac{\partial f_1(v(t), v'(t), t)}{\partial v'(t)} \right)  = 0
    \end{equation}

    esta es la que conocemos como ecuación de Euler.
