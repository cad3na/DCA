
\chapter{Lugar de las Raíces}
Si tenemos un sistema con retroalimentación, su polinomio característico es el siguiente:
\begin{equation}
F(s) = 1 + H(s) G(s) = 0
\end{equation}

Donde $G(s)$ es la planta y $H(s)$ es el elemento de retroalimentación. Las condiciones de angulo y magnitud son las siguientes:

\begin{equation}
\angle H(s) G(s) = \pm 180^{\circ} (2R + 1) \mid R \in \mathbbm{Z}^+
\end{equation}

\begin{equation}
\lvert H(s) G(s) \rvert = 1
\end{equation}

De aquí notamos que la condición de angulo, nos da la forma del lugar de las raíces, y la condición de magnitud nos da su posición.

Pues bien, para trazar el lugar geométrico de las raíces seguimos una serie de pasos enumerados a continuación:

\begin{enumerate}
\item
Determinar el lugar de las raíces en el eje real.

Ejemplo: $H(s) = 1$, $G(s) = \frac{k}{s(s+1)(s+2)}$


\item
Determinar las asintotas del lugar de las raíces.
\item
Determinar el punto de ruptura o partida de las asintotas en el eje real.
\item
Determinar los puntos donde el lugar de las raíces atraviesa el eje imaginario.
\end{enumerate}
