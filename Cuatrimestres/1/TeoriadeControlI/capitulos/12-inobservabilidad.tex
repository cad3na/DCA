%-------------------------------------------------------------------------------
%	EMPIEZA CAPITULO
%-------------------------------------------------------------------------------

\chapter{Inobservabilidad y observador de estado}

	Sea la siguiente representación de estado:

	\begin{eqnarray} \label{eq:inob1}
		\frac{d}{dt} x & = & A x + b u \nonumber \\
		y & = & c^T x + d u
	\end{eqnarray}

	para el siguiente sistema:

	\begin{figure}
    \centering
    \resizebox{\textwidth}{!}{
	    \tikzstyle{input} = [coordinate]
        \tikzstyle{output} = [coordinate]
        \tikzstyle{block} = [draw, rectangle, minimum height=3em, minimum width=4em]
        \tikzstyle{sum} = [draw, circle]
        \tikzstyle{init} = [pin edge={to-, thin, black}]
        \tikzstyle{vecArrow} = [thick, decoration={markings,mark=at position
             1 with {\arrow[semithick]{open triangle 60}}},
             double distance=1.4pt, shorten >= 5.5pt,
             preaction = {decorate},
             postaction = {draw,line width=1.4pt, white,shorten >= 4.5pt}]
        \tikzstyle{innerWhite} = [semithick, white,line width=1.4pt, shorten >= 4.5pt]

        \begin{tikzpicture}[auto, node distance=2cm, >=latex']
            \node [input, name=entrada] {};
            \node [block, right of=entrada] (b) {$b$};
            \node [sum, right of=b] (s1) {$+$};
            \node [block, right of=s1] (int) {$\int$};
      		\node [inner sep=0,minimum size=0,right of=int] (v) {};
            \node [block, right of=v] (c) {$c^T$};
            \node [sum, right of=c] (s2) {$+$};
            \node [output, right of=s2] (salida) {};
            \node [block, below of=int] (a) {$A$};
            \node [block, above of=int] (d) {$d$};

            \draw [->] (entrada) -- node[name=u] {$u$} (b);
            \draw [vecArrow] (b) -- (s1);
            \draw [vecArrow] (s1) -- (int);
            \draw [vecArrow] (int) -- (c);
            \draw [->] (c) -- (s2);
            \draw [->] (s2) -- node[name=y] {$y$} (salida);
            \draw [vecArrow] (v) |- (a);
            \draw [vecArrow] (a) -| (s1);
            \draw [->] (u) |- (d);
            \draw [->] (d) -| (s2);

            \draw [innerWhite] (b) -- (s1);
            \draw [innerWhite] (s1) -- (int);
            \draw [innerWhite] (int) -- (c);
            \draw [innerWhite] (v) |- (a);
            \draw [innerWhite] (a) -| (s1);
        \end{tikzpicture}}
	\end{figure}

	donde $x \in \mathbbm{R}^n$ es el estado, $u(t) \in \mathbbm{R}^n$ y $y(t) \in \mathbbm{R}^n$ sean la entrada y salida respectivamente; siendo la condición inicial del estado $x(0) = x_0 \in \mathbbm{R}^n$. La solución esta descrita por:

	\begin{equation*}
		x(t) = \exp{(At)} x_0 + \int_0^t \exp{\left(A(t - \tau)\right)} b u(\tau) d\tau
	\end{equation*}

	\paragraph{Problema.}

	Sea la representación de estado de la ecuación~\ref{eq:inob1}, donde el estado $x$ no esta disponible. Se desea reconstruir el estado $x$, para poder aplicar una retroalimentación de estado.

	\begin{equation}
		u = f^T x + v
	\end{equation}

	Para resolver este problema, hay que investigar el concepto estructural de la inobservabilidad.

%-------------------------------------------------------------------------------
%	EMPIEZA SECCION
%-------------------------------------------------------------------------------

	\newpage
    \section{Observabilidad e inobservabilidad}

	    Una representación de estado se dice observable si dadas las trayectorias de salida, $y(t)$, y entrada, $u(t)$, en un horizonte de tiempo finito, $t_1 \in \mathbbm{R}$, $t_1 > 0$, existe una función $\mathbbm{F}(t, u, y)$, tal que:

	    \begin{equation}
	    	\mathbbm{F}(t_1, u(t), y(t)) = x(0) \quad t \in [0, t_1]
	    \end{equation}

	    Una representación de estado se dice inobservable, si no es observable.

	    \begin{eqnarray} \label{eq:inob2}
			x(t) = \exp{(At)} x_0 + \int_0^t \exp{\left(A(t - \tau)\right)} b u(\tau) d\tau \nonumber \\
			y(t) = c^T \exp{(At)} x_0 + \int_0^t c^T \exp{\left(A(t - \tau)\right)} b u(\tau) d\tau +d u(t)
		\end{eqnarray}

		Sabemos del teorema de Caley-Hamilton que:

		\begin{equation} \label{eq:inob3}
			\exp{(At)} = \sum_{i=0}^{n-1} \varphi(t) A^i
		\end{equation}

		donde:

		\begin{equation*}
			\varphi_i(t) = \sum_{j=0}^{\infty} \varphi_{ij} t^j, \quad \varphi_{ij} \in \mathbbm{R}, \quad i \in \{ 0, 1, \dots , n-1 \}, \quad j \in \mathbbm{Z}^+
		\end{equation*}

		Si juntamos las ecuaciones ~\ref{eq:inob2} y ~\ref{eq:inob3} obtendremos:

		\begin{multline*}
			y(t_1) = \sum_{i=0}^{n-1} \varphi_i (t_1) c^T A^i x_0 + \\
			\int_0^{t_1} c^T \exp{\left(A(t - \tau)\right)} b u(\tau) d\tau +d u(t_1) = \\
			\begin{pmatrix}
				\varphi_0(t_1) & \varphi_1(t_1) & \dots & \varphi_{n-1}(t_1)
			\end{pmatrix}
			\begin{pmatrix}
				c^T \\
				c^T A \\
				\vdots \\
				c^T A^{n-1}
			\end{pmatrix} x_0 + \\
			\int_0^{t_1} c^T \exp{\left(A(t - \tau)\right)} b u(\tau) d\tau +d u(t_1)
		\end{multline*}

		\begin{equation} \label{eq:inob4}
			\varphi^T(t_1) \mathcal{O}_{(c^T, A)} x(0) = y(t_1) - \int_0^{t_1} c^T \exp{\left(A(t - \tau)\right)} b u(\tau) d\tau +d u(t_1)
		\end{equation}

		siendo el lado derecho, la función $\mathbbm{F}$. Entonces, una condición necesaia para que se pueda inferir cualquier condición inicial del estado $x(0) = x_0$, a partir de las trayectorias de salida, $y(t)$, y de entrada ,$u(t)$, en el horizonte de tiempo, es que la matriz de observabilidad:

		\begin{equation}
			\mathcal{O}_{(c^T, A)} =
			\begin{pmatrix}
				c^T \\
				c^T A \\
				\vdots \\
				c^T A^{n-1}
			\end{pmatrix}
		\end{equation}

		sea de rango pleno por columnas.

		En nuestro caso particular, como la entrada y la salida estan en $\mathbbm{R}$, esta condición es:

		\begin{equation*}
			\det{\mathcal{O}_{(c^T, A)}} \ne 0
		\end{equation*}

		En efecto, si $\mathcal{O}_{(c^T, A)}$, no es de rango pleno por columna, existe una transformación $T$, invertible, tal que:

		\begin{equation*}
			\mathcal{O}_{(c^T, A)} T^{-1} =
			\begin{pmatrix}
				\mathbbm{X} & 0
			\end{pmatrix}
		\end{equation*}

		siendo $\mathbbm{X}$ una matriz de rango plano por columnas.

		Haciendo el cambio de base, $\bar{x} = T x$, se obtiene de la ecuación ~\ref{eq:inob4}:

		\begin{eqnarray*}
			\varphi^T(t_1) \mathcal{O}_{(c^T, A)} T^{-1} \bar{x}(0) & = & \bar{\mathbbm{F}}(t_1, u, y) \\
			\varphi^T(t_1)
			\begin{pmatrix}
			 	\mathbbm{X} & 0
			\end{pmatrix}
			\begin{pmatrix}
			 	\bar{x}_1(0) \\
			 	\bar{x}_2(0)
			\end{pmatrix} & = & \bar{\mathbbm{F}}(t_1, u, y) \\
			\varphi^T(t_1) \mathbbm{X} \bar{x}_1(0) & = & \bar{\mathbbm{F}}(t_1, u, y)
		\end{eqnarray*}

		por lo que no es posible determinar la segunda parte de componentes, $\bar{x}_2(0)$, a partir de $\bar{\mathbbm{F}}(t_1, u, y)$.

		Si la matriz de observabilidad es de rango pleno por columnas, entonces\footnote{Este es un resultado del teorema de espacios vectoriales que indica que $\dim{V} = \dim{\ker{T}} + \dim{\imagen{T}}$, y que lo que queremos es que $\dim{V} = \dim{\imagen{T}}$}:

		\begin{equation*}
			\ker{\mathcal{O}_{(c^T, A)}} = {0}
		\end{equation*}

		y para nuestro caso particular, $u, y \in \mathbbm{R}$:

		\begin{equation*}
			\det{\mathcal{O}_{(c^T, A)}} \ne 0
		\end{equation*}

		es decir, $\mathcal{O}_{(c^T, A)}$ es invertible.

		De la representación de estado en la ecuación ~\ref{eq:inob1} se tiene:

		\begin{equation*}
			y = c^T x + d u
		\end{equation*}

		\begin{eqnarray*}
			\frac{dy}{dt} & = & c^T \frac{dx}{dt} + d \frac{du}{dt} \\
			& = & c^T A x + c^T b u + d \frac{du}{dt}
		\end{eqnarray*}

		\begin{eqnarray*}
			\frac{d^2y}{dt^2} & = & c^T A \frac{dx}{dt} + c^T b \frac{du}{dt} + d \frac{d^2u}{dt^2} \\
			& = & c^T A^2 x + c^T A b u + c^T b \frac{du}{dt} + d \frac{d^2u}{dt^2}
		\end{eqnarray*}

		\begin{equation*}
			\vdots
		\end{equation*}

		\begin{equation*}
			\frac{d^{n-1}y}{dt^{n-1}} = c^T A^{n-1} x + \sum_{i=0}^{n-2} c^T A^i b \frac{d^{n-2-i}u}{dt^{n-2-i}}
		\end{equation*}

		por lo que tendremos que:

		\begin{equation*}
			\begin{pmatrix}
				1 \\
				\frac{d}{dt} \\
				\vdots \\
				\frac{d^{n-1}}{dt^{n-1}}
			\end{pmatrix} y =
			\begin{pmatrix}
				c^T \\
				c^T A \\
				\vdots \\
				c^T A^{n-1}
			\end{pmatrix} x +
			\begin{pmatrix}
				0 + d \\
				b + d \frac{d}{dt} \\
				\vdots \\
				\sum_{i=0}^{n-2} c^T A^i b \frac{d^{n-2-i}}{dt^{n-2-i}} + d \frac{d^{n-1}}{dt^{n-1}}
			\end{pmatrix} u
		\end{equation*}

		o escrito de otra manera:

		\begin{equation*}
			\Delta \left( \frac{d}{dt} \right) y = \mathcal{O}_{(c^T, A)} x + \Gamma \left( \frac{d}{dt} \right) u
		\end{equation*}

		lo cual implica:

		\begin{equation}
			x = \mathcal{O}_{(c^T, A)}^{-1} \left[ \Delta \left( \frac{d}{dt} \right) y - \Gamma \left( \frac{d}{dt} \right) u \right]
		\end{equation}

		Por lo que es una condición necesaria y suficiente, para que la representación de estado sea observable que su matriz de observabilidad, $\mathcal{O}_{(c^T, A)}$, sea de rango pleno por columnas.

		Cuando la matriz de observabilidad es de rango pleno por columnas, se dice que el par $(c^T, A)$ es observable.

%-------------------------------------------------------------------------------
%	EMPIEZA SECCION
%-------------------------------------------------------------------------------

    \newpage
    \section{Dualidad}

		La operación matricial "transpuesta", establece una dualidad entre la observabilidad y la controlabilidad. En efecto,

		\begin{equation}
			\mathcal{O}_{(c^T, A)} =
			 \begin{pmatrix}
			 	c^T \\
				c^T A \\
				\vdots \\
				c^T A^{n-1}
			 \end{pmatrix}
		\end{equation}

		\begin{equation}
			\mathcal{O}_{(c^T, A)}^T =
			\begin{pmatrix}
				c^T \\
				c^T A \\
				\vdots \\
				c^T A^{n-1}
			\end{pmatrix}^T =
			\begin{pmatrix}
				c & A^T c & \dots & \left( A^T \right)^{n-1} c
			\end{pmatrix} = C_{(A^T, c)}
		\end{equation}

		\begin{equation}
			C^T_{(A, b)} =
			\begin{pmatrix}
				b & A b & \dots & A^{n-1} b
			\end{pmatrix}^T =
			\begin{pmatrix}
				b^T \\
				b^T A^T \\
				\vdots \\
				b^T (A^T)^{n-1}
			\end{pmatrix} = \mathcal{O}_{(b^T, A^T)}
		\end{equation}

		note tambien que la función de transferencia es una funcion continua en $\mathbbm{R}^1$, por lo que:

		\begin{equation*}
			FT = FT^T
		\end{equation*}

		\begin{equation}
			b^T \left( sI - A^T \right)^{-1} c + d = c^T \left( sI - A \right)^{-1} b + d
		\end{equation}

		ademas, el polinomio caracteristico es el mismo:

		\begin{equation}
			\det{(sI - A)} = \det{(sI - A^T)} \implies \sigma(A) = \sigma (A^T)
		\end{equation}

		Por lo que obtenemos la siguiente dualidad:

		\begin{equation*}
			A \leftrightarrow A^T
		\end{equation*}

		\begin{equation*}
			b \leftrightarrow c
		\end{equation*}

		\begin{equation*}
			c^T \leftrightarrow b^T
		\end{equation*}

		\begin{equation*}
			d \leftrightarrow b
		\end{equation*}

		\begin{equation*}
			f^T \leftrightarrow k^T
		\end{equation*}

%-------------------------------------------------------------------------------
%	EMPIEZA SECCION
%-------------------------------------------------------------------------------

    \newpage
    \section{Propiedades de la matriz de observabilidad}

		Dada la dualidad entre observabilidad y controlabilidad, todos los resultados de controlabilidad del par $(A, b)$ son extrapolables a la observabilidad del par $(c^T, A)$.

		\begin{enumerate}
			\item Asignación de los valores propios (polos), mediante la inyección de salida.

			\begin{enumerate}
				\item Si $\det{C_{(A, b)}} \ne 0$, entonces dado un conjunto simetrico con respecto al eje real de $n$ numeros complejos, $\Lambda$, existe un vector $f \in \mathbbm{R}^n$, tal que:

				\begin{equation*}
					\sigma(A + b f^T) = \Lambda
				\end{equation*}

				\item Si $\det{\mathcal{O}_{(c^T, A)}} \ne 0$, entonces dado un conjunto simetrico con respecto al eje real de $n$ numeros complejos, $\Lambda$, existe un vector $k \in \mathbbm{R}^n$, tal que:

				\begin{equation*}
					\sigma(A + k c^T) = \Lambda
				\end{equation*}

				\begin{equation*}
					\sigma(A^T + c k^T) = \Lambda
				\end{equation*}
			\end{enumerate}

			\item Invarianza de la matriz de observabilidad bajo cambio de base

			Sea el siguiente cambio de base:

			\begin{eqnarray}
				A_1 & = & T^{-1} A T \\
				c_1^T & = & c^T T
			\end{eqnarray}

			dado un cambio de base $T$ invertible; entonces tendremos lo siguiente:

			\begin{equation}
				\mathcal{O}_{(c_1^T, A_1)} = \mathcal{O}_{(c^T, A)} T^{-1}
			\end{equation}

			de donde podemos notar que:

			\begin{equation}
				T = \mathcal{O}_{(c_1^T, A_1)}^{-1} \mathcal{O}_{(c^T, A)}
			\end{equation}

			siempre que el par $(c^T, A)$ sea observable.

			\item Invarianza de la matriz de observabilidad bajo inyección de salida

			\item Invarianza de los ceros del sistema bajo iyección de salida
		\end{enumerate}

%-------------------------------------------------------------------------------
%	EMPIEZA SECCION
%-------------------------------------------------------------------------------

    \newpage
    \section{Formas canónicas}
	\faltante{Falta escribir apunte}

%-------------------------------------------------------------------------------

        \subsection{Forma canónica observador}
		\faltante{Falta escribir apunte}

%-------------------------------------------------------------------------------

        \subsection{Forma canónica observabilidad}
		\faltante{Falta escribir apunte}

%-------------------------------------------------------------------------------
%	EMPIEZA SECCION
%-------------------------------------------------------------------------------

    \newpage
    \section{Observador de estado}
	\faltante{Falta escribir apunte}
