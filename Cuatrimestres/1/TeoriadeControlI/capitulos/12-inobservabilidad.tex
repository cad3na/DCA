
\chapter{Inobservabilidad y observador de estado}

	Sea la siguiente representación de estado:

	\begin{eqnarray} \label{eq:inob1}
		\frac{d}{dt} x & = & A x + b u \nonumber \\
		y & = & c^T x + d u
	\end{eqnarray}

	para el siguiente sistema:

	\begin{figure}
    \centering
    \resizebox{\textwidth}{!}{
	    \tikzstyle{input} = [coordinate]
        \tikzstyle{output} = [coordinate]
        \tikzstyle{block} = [draw, rectangle, minimum height=3em, minimum width=4em]
        \tikzstyle{sum} = [draw, circle]
        \tikzstyle{init} = [pin edge={to-, thin, black}]
        \tikzstyle{vecArrow} = [thick, decoration={markings,mark=at position
             1 with {\arrow[semithick]{open triangle 60}}},
             double distance=1.4pt, shorten >= 5.5pt,
             preaction = {decorate},
             postaction = {draw,line width=1.4pt, white,shorten >= 4.5pt}]
        \tikzstyle{innerWhite} = [semithick, white,line width=1.4pt, shorten >= 4.5pt]

        \begin{tikzpicture}[auto, node distance=2cm, >=latex']
            \node [input, name=entrada] {};
            \node [block, right of=entrada] (b) {$b$};
            \node [sum, right of=b] (s1) {$+$};
            \node [block, right of=s1] (int) {$\int$};
      		\node [inner sep=0,minimum size=0,right of=int] (v) {};
            \node [block, right of=v] (c) {$c^T$};
            \node [sum, right of=c] (s2) {$+$};
            \node [output, right of=s2] (salida) {};
            \node [block, below of=int] (a) {$A$};
            \node [block, above of=int] (d) {$d$};

            \draw [->] (entrada) -- node[name=u] {$u$} (b);
            \draw [vecArrow] (b) -- (s1);
            \draw [vecArrow] (s1) -- (int);
            \draw [vecArrow] (int) -- (c);
            \draw [->] (c) -- (s2);
            \draw [->] (s2) -- node[name=y] {$y$} (salida);
            \draw [vecArrow] (v) |- (a);
            \draw [vecArrow] (a) -| (s1);
            \draw [->] (u) |- (d);
            \draw [->] (d) -| (s2);

            \draw [innerWhite] (b) -- (s1);
            \draw [innerWhite] (s1) -- (int);
            \draw [innerWhite] (int) -- (c);
            \draw [innerWhite] (v) |- (a);
            \draw [innerWhite] (a) -| (s1);
        \end{tikzpicture}}
	\end{figure}

	donde $x \in \mathbbm{R}^n$ es el estado, $u(t) \in \mathbbm{R}^n$ y $y(t) \in \mathbbm{R}^n$ sean la entrada y salida respectivamente; siendo la condición inicial del estado $x(0) = x_0 \in \mathbbm{R}^n$. La solución esta descrita por:

	\begin{equation*}
		x(t) = \exp{(At)} x_0 + \int_0^t \exp{\left(A(t - \tau)\right)} b u(\tau) d\tau
	\end{equation*}

	\paragraph{Problema.}

	Sea la representación de estado ~\ref{eq:inob1}, donde el estado $x$ no esta disponible. Se desea reconstruir el estado $x$, para poder aplicar una retroalimentación de estado.

	\begin{equation}
		u = f^T x + v
	\end{equation}

	Para resolver este problema, hay que investigar el concepto estructural de la inobservabilidad.

	\newpage
    \section{Observabilidad e inobservabilidad}

    \newpage
    \section{Dualidad}

    \newpage
    \section{Propiedades de la matriz de observabilidad}

    \newpage
    \section{Formas canónicas}
        \subsection{Forma canónica observador}
        \subsection{Forma canónica observabilidad}

    \newpage
    \section{Observador de estado}
