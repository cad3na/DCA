%-------------------------------------------------------------------------------
%	EMPIEZA CAPITULO
%-------------------------------------------------------------------------------

\chapter{Nociones de control adaptable}

    El control adaptable es la combinación de una ley de control lineal con un algoritmo de identificación en linea.

    \begin{figure}
        \centering
        \resizebox{0.8\textwidth}{!}{
        \tikzstyle{input} = [coordinate]
        \tikzstyle{output} = [coordinate]
        \tikzstyle{block} = [draw, rectangle, minimum height=4.5em, minimum width=4em]
        \tikzstyle{sum} = [draw, circle]
        \tikzstyle{init} = [pin edge={to-, thin, black}]

        \begin{tikzpicture}[auto, scale=1, node distance=3.2cm, >=latex']
            \node [input, name=entrada] {};
            \node [sum, right of=entrada] (suma) {$+$};
            \node [block, right of=suma] (cont) {Controlador};
            \node [block, right of=cont] (planta) {Planta};
            \node [output, right of=planta] (salida) {};
            \node [block, align=center, below of=planta] (ident) {Algoritmo \\ de \\ Identificación};
            \node [block, below of=ident] (retro) {$-1$};

            \draw [->] (entrada) -- node[name=x] {$\hat{r}(s)$} (suma);
            \draw [->] (suma) -- node[name=e] {$e$} (cont);
            \draw [->] (cont) -- node[name=u] {$u$} (planta);
            \draw [->] (planta) -- node[name=y] {$\hat{y}(s)$} (salida);
            \draw [->] (y) |- (retro);
            \draw [->] (y) |- (ident);
            \draw [->] (u) |- (ident);
            \draw [->] (ident.south) -| (cont.south);
            \draw [->] (retro) -| (suma);
        \end{tikzpicture}}
        \caption{\label{dia:adap1}Sistema con un esquema de control adaptable.}
    \end{figure}

    Para abordar este tipo de controlador necesitamos estudiar primero los conceptos de regresor lineal, algoritmo de identificacion y prueba de convergencia con estabilidad.

%-------------------------------------------------------------------------------
%	EMPIEZA SECCION
%-------------------------------------------------------------------------------

    \section{Regresor lineal}

        Consideremos un sistema lineal, invariante en el tiempo, representado por la siguiente ecuación diferencial ordinaria:

        \begin{equation} \label{eq:adap1}
            M \left( \frac{d}{dt} \right) y(t) = N \left( \frac{d}{dt} \right) u(t)
        \end{equation}

        la cual es equivalente al diagrama de bloques de la figura~\ref{dia:adap2}.

        \begin{marginfigure}
            \centering
            \resizebox{\textwidth}{!}{
                \tikzstyle{input} = [coordinate]
                \tikzstyle{output} = [coordinate]
                \tikzstyle{block} = [draw, rectangle, minimum height=3em, minimum width=4em]
                \tikzstyle{sum} = [draw, circle]
                \tikzstyle{init} = [pin edge={to-, thin, black}]

                \begin{tikzpicture}[auto, node distance=2cm, >=latex']
                    \node [input, name=entrada] {};
                    \node [block, right of=entrada] (planta) {$\frac{N(s)}{M(s)}$};
                    \node [output, right of=planta] (salida) {};

                    \draw [->] (entrada) -- node[name=u] {$u(s)$} (planta);
                    \draw [->] (planta) -- node[name=y] {$y(s)$} (salida);
                \end{tikzpicture}}
            \caption{\label{dia:adap2}Sistema con una función de transferencia propia.}
        \end{marginfigure}

        Sabemos que $M(s), N(s) \in \mathbbm{R}[s]$ estarán dados por:

        \begin{eqnarray} \label{eq:adap2}
            M(s) & = & s^n + a_1 s^{n-1} + \dots + a_{n-1} s + a_n \nonumber \\
            N(s) & = & b_0 s^m + b_1 s^{m-1} + \dots + b_{m-1} s + b_m
        \end{eqnarray}

        con $m \le n$.

        \begin{marginfigure}
            \centering
            \resizebox{\textwidth}{!}{
                \tikzstyle{input} = [coordinate]
                \tikzstyle{output} = [coordinate]
                \tikzstyle{block} = [draw, rectangle, minimum height=3em, minimum width=4em]
                \tikzstyle{sum} = [draw, circle]
                \tikzstyle{init} = [pin edge={to-, thin, black}]

                \begin{tikzpicture}[auto, node distance=2cm, >=latex']
                    \node [input, name=entrada] {};
                    \node [block, right of=entrada] (planta) {$\frac{1}{F(s)}$};
                    \node [output, right of=planta] (salida) {};

                    \draw [->] (entrada) -- node[name=u] {$u(s)$} (planta);
                    \draw [->] (planta) -- node[name=y] {$u_f(s)$} (salida);
                \end{tikzpicture}}
            \caption{\label{dia:adap3}Filtro auxiliar de la entrada del sistema.}
        \end{marginfigure}

        \begin{marginfigure}
            \centering
            \resizebox{\textwidth}{!}{
                \tikzstyle{input} = [coordinate]
                \tikzstyle{output} = [coordinate]
                \tikzstyle{block} = [draw, rectangle, minimum height=3em, minimum width=4em]
                \tikzstyle{sum} = [draw, circle]
                \tikzstyle{init} = [pin edge={to-, thin, black}]

                \begin{tikzpicture}[auto, node distance=2cm, >=latex']
                    \node [input, name=entrada] {};
                    \node [block, right of=entrada] (planta) {$\frac{1}{F(s)}$};
                    \node [output, right of=planta] (salida) {};

                    \draw [->] (entrada) -- node[name=u] {$y(s)$} (planta);
                    \draw [->] (planta) -- node[name=y] {$y_f(s)$} (salida);
                \end{tikzpicture}}
            \caption{\label{dia:adap4}Filtro auxiliar de la salida del sitema.}
        \end{marginfigure}

        Consideremos los filtros auxiliares de las figuras~\ref{dia:adap3} y~\ref{dia:adap4}, por lo que las ecuaciones que representan su comportamiento serán:

        \begin{eqnarray} \label{eq:adap3}
            F \left( \frac{d}{dt} \right) u_f (t) & = & u(t) \nonumber \\
            F \left( \frac{d}{dt} \right) y_f (t) & = & y(t)
        \end{eqnarray}

        en donde $F(s) \in \mathbbm{R}[s]$ es Hurwitz estable y tiene la forma:

        \begin{equation*}
            F(s) = s^n + f_1 s^{n-1} + \dots + f_{n-1} s + f_n
        \end{equation*}

        Si sustituimos las ecuaciones~\ref{eq:adap3} en la ecuación~\ref{eq:adap1}, tendremos:

        \begin{equation*}
            M \left( \frac{d}{dt} \right) F \left( \frac{d}{dt} \right) y_f (t) = N \left( \frac{d}{dt} \right) F \left( \frac{d}{dt} \right) u_f (t)
        \end{equation*}

        es decir:

        \begin{equation*}
            F \left( \frac{d}{dt} \right) \left( M \left( \frac{d}{dt} \right)  y_f (t) - N \left( \frac{d}{dt} \right) u_f (t) \right) = 0
        \end{equation*}

        por lo que podemos introducir la función $\xi(t)$:

        \begin{equation*}
            F \left( \frac{d}{dt} \right) \xi(t) = 0
        \end{equation*}

        por lo que podemos escribir nuestra función original como:

        \begin{equation} \label{eq:adap4}
            M \left( \frac{d}{dt} \right) y(t) = N \left( \frac{d}{dt} \right) u(t) + \xi(t)
        \end{equation}

        Puesto que $F(s)$ es un polinomio Hurwitz estable, entonces existe $k, \alpha \in \mathbbm{R}^+$, con $k > 0$ y $\alpha > 0$, tal que:

        \begin{equation*}
            |\xi(t)| \le k e^{- \alpha t} \quad \forall t \ge 0
        \end{equation*}

        \missingfigure{Comportamiento general acotado por exponencial}

        por lo que para un $\alpha$ suficientemente grande, los comportamientos de las ecuaciones~\ref{eq:adap4} y~\ref{eq:adap1} será aproximadamente iguales.

        Desarrollando la ecuación~\ref{eq:adap4} se obtiene el siguiente regresor lineal:

        \begin{equation}
            y_f^{(n)}(t) = \theta^T \phi(t) + \xi(t)
        \end{equation}

        en donde:

        \begin{eqnarray*}
            \theta^T & = & \begin{pmatrix} a_1 & \dots & a_n & b_0 & \dots & b_m \end{pmatrix} \in \mathbbm{R}^{n+m+1} \\
            \phi^T & = & \begin{pmatrix} -y_f^{(n-1)} & \dots & -y_f & u_f^{(m)} & \dots & u_f \end{pmatrix} \in \mathbbm{R}^{n+m+1}
        \end{eqnarray*}

        a los que llamamos vector de parametros y vector de mediciones.

        Note que las derivadas sucesivas de $y_f(t)$ y $u_f(t)$, se obtienen directamente de los filtros de las ecuaciones~\ref{eq:adap3}.

        \missingfigure{Diagrama de bloques de un filtro para el regresor lineal}
