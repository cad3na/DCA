
\chapter{Controlabilidad y asignación de polos}

    Sea un sistema para la Ecuación Diferencial Ordinaria:

    \begin{equation}
        M \left(\frac{d}{dt} \right) y(t) = N \left(\frac{d}{dt} \right) u(t)
    \end{equation}

    Sea la siguiente representación de estado de esta Ecuación Diferencial Ordinaria:

    \begin{eqnarray}
    \frac{d}{dt} x & = & A x + b u \nonumber \\
    y & = & c^T x + d u \nonumber
    \end{eqnarray}

    donde $x \in \mathbbm{R}^n$ y $u,y \in \mathbbm{R}$.

    \begin{figure}
    \centering
    \resizebox{\textwidth}{!}{
	    \tikzstyle{input} = [coordinate]
        \tikzstyle{output} = [coordinate]
        \tikzstyle{block} = [draw, rectangle, minimum height=3em, minimum width=4em]
        \tikzstyle{sum} = [draw, circle]
        \tikzstyle{init} = [pin edge={to-, thin, black}]
        \tikzstyle{vecArrow} = [thick, decoration={markings,mark=at position
             1 with {\arrow[semithick]{open triangle 60}}},
             double distance=1.4pt, shorten >= 5.5pt,
             preaction = {decorate},
             postaction = {draw,line width=1.4pt, white,shorten >= 4.5pt}]
        \tikzstyle{innerWhite} = [semithick, white,line width=1.4pt, shorten >= 4.5pt]

        \begin{tikzpicture}[auto, node distance=2cm, >=latex']
            \node [input, name=entrada] {};
            \node [block, right of=entrada] (b) {$b$};
            \node [sum, right of=b] (s1) {$+$};
            \node [block, right of=s1] (int) {$\int$};
      		\node [inner sep=0,minimum size=0,right of=int] (v) {};
            \node [block, right of=v] (c) {$c^T$};
            \node [sum, right of=c] (s2) {$+$};
            \node [output, right of=s2] (salida) {};
            \node [block, below of=int] (a) {$A$};
            \node [block, above of=int] (d) {$d$};

            \draw [->] (entrada) -- node[name=u] {$u$} (b);
            \draw [vecArrow] (b) -- (s1);
            \draw [vecArrow] (s1) -- (int);
            \draw [vecArrow] (int) -- (c);
            \draw [->] (c) -- (s2);
            \draw [->] (s2) -- node[name=y] {$y$} (salida);
            \draw [vecArrow] (v) |- (a);
            \draw [vecArrow] (a) -| (s1);
            \draw [->] (u) |- (d);
            \draw [->] (d) -| (s2);

            \draw [innerWhite] (b) -- (s1);
            \draw [innerWhite] (s1) -- (int);
            \draw [innerWhite] (int) -- (c);
            \draw [innerWhite] (v) |- (a);
            \draw [innerWhite] (a) -| (s1);
        \end{tikzpicture}}
	\end{figure}

    Problema. Se desea encontrar una ley de control $u = f(x)$, que nos permita asignar los polos a voluntad.

    Sabemos que:

    \begin{description}
        \item [Polos.] $\left\{ s \in \mathbbm{C} \mid M(s) = 0 \right\} = \left\{ s \in \mathbbm{C} \mid \det{(sI - A)} \right\}$
    \end{description}

    Para resolver este problema hay que investigar el concepto estructural de la alcanzabilidad.

    \newpage
    \section{Alcanzabilidad y Controlabilidad}

        \begin{itemize}
            \item Una representación de estado se dice controlable, si para cualquier condición inicial, $x(0) = x_0 \in \mathbbm{R}^n$, existe una trayectoria, $x(\cdot)$, solución de la ecuación de estado, tal que en tiempo finito $t_f \in \mathbbm{R}$ se llega al origen $\left( x(t_f) = 0 \right)$.
            \item Una representación de estado se dice alcanzable, si para cualquier punto $x \in \mathbbm{R}$, existe una trayectoria, $x(\cdot)$, solución de la ecuación de estado, tal que en tiempo finito $t_f \in \mathbbm{R}$ se llega a un punto cualquiera $\left( x(t_f) = x_f \right)$ desde el origen.
        \end{itemize}

        En los sistemas lineales estas dos propiedades están mutuamente implicadas, por lo que se les trata indistinguiblemente. Pero en general:

        \begin{equation}
            \text{Alcanzabilidad} \implies \text{Controlabilidad}
        \end{equation}

        La solución temporal de $\frac{dx}{dt} = Ax + bu$ con $x(0) = 0$ es:

        \begin{equation}
            x(t) = \int_0^t \exp{(A(t - \tau))} b u(\tau) d \tau
        \end{equation}

        del teorema de Cayley-Hamilton se tiene:

        \begin{equation}
            \exp{(A t)} = \sum\limits_{i=0}^{\infty} \frac{1}{i!} A^i t^i = \sum\limits_{i=0}^{n-1} \varphi_i(t) A^i
        \end{equation}

        donde $\varphi_i(t) = \sum\limits_{j=0}^{\infty} \varphi_{ij} t^j$, $\varphi_{ij} \in \mathbbm{R}$, $n \in \mathbbm{N}$, $j \in \mathbbm{Z}^+$. Por lo anterior, tenemos:

        \begin{eqnarray}
        x(t) & = & \sum\limits_{i=0}^{n-1} \psi_i(t) A^i b \nonumber \\
        x(t) & = &
        \begin{pmatrix}
        b & A b & \dots & A^{n-1} b
        \end{pmatrix}
        \begin{pmatrix}
        \psi_0(t) \\
        \psi_1(t) \\
        \vdots \\
        \psi_{n-1}(t)
        \end{pmatrix}
        \end{eqnarray}

        donde $\psi_i(t) = \int_0^t \varphi_i (t - \tau) u(\tau) d \tau$.

        Entonces una condición necesaria para que $x(t_f) = x_f \quad \forall x \in \mathbbm{R}, \forall t_f \in \mathbbm{R}, t_f > 0$, es que la matriz de controlabilidad

        \begin{equation}
            C_{(A,b)} =
            \begin{pmatrix}
            b & Ab & \dots & A^{n-1}b
            \end{pmatrix}
        \end{equation}

        sea de rango pleno por filas, de lo contrario existen componentes de $x(t)$ que siempre seran nulos. En nuestro caso particular $\left( y, u \in \mathbbm{R}^n \right)$:

        \begin{equation}
            \det{C_{(A,b)}} \ne 0
        \end{equation}

        Si la matriz de controlabilidad $C_{(A,b)}$ es de rango pleno por filas, entonces el gramiano de controlabilidad es invertible. \footnote{Aquí se esta abusando de la notación, ya que el gramiano de controlabilidad corresponde al caso en que $t \to \infty$}

        \begin{equation}
            W = \int_0^t \exp{(A \sigma)} b b^t \exp{(A^t \sigma)} d\sigma \quad t > 0, \sigma = t - \tau
        \end{equation}

        Entonces, con la siguiente ley de control se tiene:

        \begin{equation}
            u(t) = b^t \exp{(A^t(t_f - t))} W_{t_f}^{-1} x_f
        \end{equation}

        Por lo que si sustituimos $t_f$ en la solución para $x(t)$:

        \begin{multline}
            x(t_f) = \int_0^{t_f} \exp{(A(t - \tau))} b u(\tau) d\tau \\
                   = \int_0^{t_f} \exp{(A(t - \tau))} b b^t \exp{(A^t(t_f - \tau))} W_{t_f}^{-1} x_f d\tau \\
                   = \int_{t_f}^0 \exp{(A \sigma)} b b^t \exp{(A^t \sigma)} d\sigma W_{t_f}^{-1} x_f \\
                   = W_{t_f} W_{t_f}^{-1} x_f = x_f
        \end{multline}

        \begin{itemize}
            \item Por lo que una condición suficiente y necesaria para que la ecuación de estado sea alcanzable (y por lo tanto controlable), es que su matriz de controlabilidad, $C_{(A,b)}$, sea de rango pleno por filas.
            \item Cuando la matriz de controlabilidad es de rango pleno por filas, se dice que el par $(A, b)$ es controlable.
        \end{itemize}

    \newpage
    \section{Asignación de polos}

        Sea la ecuación de estado controlable, es decir $\frac{dx}{dt} = Ax + bu$ con $b \ne 0$, $\det{(C_{(A,b)})} \ne 0$, $\Pi(s)$ y $\alpha(s)$ los polinomios característico y mínimo de $A$ respectivamente.

        \begin{equation}
            \Pi(s) = \det{(sI - A)} \quad \text{grado } \Pi(s) = n \nonumber
        \end{equation}

        \begin{equation}
            \alpha(s) \text{ es el polinomio de menor grado tal que } \alpha(A) = 0 \nonumber
        \end{equation}

        Sea $\kappa = \text{grado } \alpha(s)$, donde obviamente $1 \le \kappa \le n$.

        \begin{equation}
            \alpha(s) = s^{\kappa} + (a_{\kappa} + a_{\kappa - 1} s + \dots + a_1 s^{\kappa - 1}) \nonumber
        \end{equation}

        \begin{equation}
            \alpha(A) = A^{\kappa} + (a_{\kappa} + a_{\kappa - 1} A + \dots + a_1 A^{\kappa - 1}) \nonumber
        \end{equation}

        Sean $\alpha_i$ con $i \in \{ 0, 1, \dots, \kappa \}$, los polinomios mónicos auxiliares tales que:

        \begin{eqnarray}
        \alpha_0(s) & = & \alpha(s) \nonumber \\
        \alpha_1(s) & = & s^{\kappa - 1} + (a_{\kappa - 1} + a_{\kappa - 2} s + \dots + a_1 s^{\kappa - 2}) \nonumber \\
        \vdots \nonumber \\
        \alpha_{\kappa - 1}(s) & = & s + a_1 \nonumber \\
        \alpha_{\kappa}(s) & = & 1 \nonumber
        \end{eqnarray}

        en donde, por definición, $\alpha_1(A) \ne 0$ y $\alpha_0(A) = 0$.

        Sea $b \ne 0$, un vector en $\mathbbm{R}^n$ tal que su polinomio mínimo coincide con $\alpha(s)$.

        \begin{eqnarray}
        \alpha_i(A) b & \ne & 0 \quad i \in \{ 1, 2, \dots, \kappa \} \nonumber \\
        \alpha_0(A) b & = & 0 \nonumber
        \end{eqnarray}

        \begin{equation}
            \vdots \nonumber
        \end{equation}

        \begin{eqnarray}
        (A^{\kappa - 1} + (a_{\kappa - 1} + a_{\kappa - 2} A + \dots + a_1 A^{\kappa - 2})) b & \ne & 0 \nonumber \\
        \alpha_0(A) b & = & 0 \nonumber
        \end{eqnarray}

        \begin{equation}
            \vdots \nonumber
        \end{equation}

        \begin{eqnarray}
        \begin{pmatrix}
        b & Ab & \dots & A^{\kappa - 1} b
        \end{pmatrix}
        \begin{pmatrix}
        a_{\kappa - 1} \\
        a_{\kappa - 2} \\
        \vdots \\
        a_{1}
        \end{pmatrix} & \ne & 0 \nonumber \\
        \alpha_0(A) b & = & 0 \nonumber
        \end{eqnarray}

        Suponga que el par $(A,b)$ es controlable, por lo tanto $\det{C_{(A,b)}} \ne 0$, entonces:

        \begin{equation}
            \begin{pmatrix}
            b & A b & \dots & A^{n-1} b
            \end{pmatrix} v \ne 0 \quad \forall v \ne 0 \quad \therefore \kappa = n
        \end{equation}

        Por lo que el polinomio mínimo y el polinomio característico coinciden cuando el par $(A,b)$ es controlable. Definimos la base:

        \begin{eqnarray}
        e_n & = & \alpha_n(A) b = b \nonumber \\
        e_{n-1} & = & \alpha_{n-1}(A) b = (A + a_1I) b = A e_n + a_1 e_n \nonumber \\
        e_{n-2} & = & \alpha_{n-2}(A) b = (A^2 + (a_2I + a_1A)) b = A(A+a_1I) b + a_2 b = A e_{n-1} + a_2 e_n \nonumber \\
        \vdots & = & \vdots \nonumber \\
        e_1 & = & \alpha_1(A) b = A e_2 + a_{n-1} e_n
        \end{eqnarray}

        Note que sustituyendo $A$, tenemos:

        \begin{equation}
            \alpha(A) b = (A^n + (a_nI + a_{n-1}A + \dots + a_1 A^{n-1})) b = A e_1 + a_n e_n = 0
        \end{equation}

        Por lo que:

        \begin{eqnarray}
        e_n & = & b \nonumber \\
        A e_{n} & = & e_{n-1} - a_1 e_n \nonumber \\
        A e_{n-1} & = & e_{n-2} - a_2 e_n \nonumber \\
        \vdots & = & \vdots \nonumber \\
        A e_2 & = & e_1 - a_{n-1} e_n \nonumber \\
        A e_1 & = & - a_{n} e_n
        \end{eqnarray}

        Entonces, bajo la base definida, las transformaciones lineales tienen la siguiente forma:

        \begin{equation}
            A_c = \left[ A \right]_{\{ e_1, e_2, \dots, e_n \}} =
            \begin{pmatrix}
            0 & 1 & 0 & \dots & 0 & 0 & 0 \\
            0 & 0 & 1 & \dots & 0 & 0 & 0 \\
            0 & 0 & 0 & \dots & 0 & 0 & 0 \\
            \vdots & \vdots & \vdots & & \vdots & \vdots & \vdots \\
            0 & 0 & 0 & \dots & 0 & 1 & 0 \\
            0 & 0 & 0 & \dots & 0 & 0 & 1 \\
            -a_{n} & -a_{n-1} & -a_{n-2} & \dots & -a_{3} & -a_{2} & -a_{1}
            \end{pmatrix}
        \end{equation}

        \begin{equation}
            b_c = \left[ b \right]_{\{ e_1, e_2, \dots, e_n \}} =
            \begin{pmatrix}
            0 \\
            0 \\
            0 \\
            \vdots \\
            0 \\
            0 \\
            1
            \end{pmatrix}
        \end{equation}

        Por lo tanto:

        \begin{equation}
            \frac{d}{dt} x_c =
            \begin{pmatrix}
            0 & 1 & 0 & \dots & 0 & 0 & 0 \\
            0 & 0 & 1 & \dots & 0 & 0 & 0 \\
            0 & 0 & 0 & \dots & 0 & 0 & 0 \\
            \vdots & \vdots & \vdots & & \vdots & \vdots & \vdots \\
            0 & 0 & 0 & \dots & 0 & 1 & 0 \\
            0 & 0 & 0 & \dots & 0 & 0 & 1 \\
            -a_{n} & -a_{n-1} & -a_{n-2} & \dots & -a_{3} & -a_{2} & -a_{1}
            \end{pmatrix} x_c +
            \begin{pmatrix}
            0 \\
            0 \\
            0 \\
            \vdots \\
            0 \\
            0 \\
            1
            \end{pmatrix} u
        \end{equation}

        \begin{figure}
            \centering
            \resizebox{\textwidth}{!}{
                \tikzstyle{input} = [coordinate]
                \tikzstyle{output} = [coordinate]
                \tikzstyle{empty} = [coordinate]
                \tikzstyle{block} = [draw, rectangle, minimum height=2.25em, minimum width=2.25em]
                \tikzstyle{sum} = [draw, circle]
                \tikzstyle{init} = [pin edge={to-, thin, black}]

                \begin{tikzpicture}[auto, node distance=1.1cm, >=latex']
                    \node [input, name=entrada] {};
                    \node [block, right of=entrada] (uno) {$1$};
                    \node [sum, right of=uno] (s1) {$+$};
                    \node [block, right of=s1] (int1) {$\int$};
                    \node [inner sep=0,minimum size=0,right of=int1] (xn) {};
                    \node [block, right of=xn] (int2) {$\int$};
                    \node [inner sep=0,minimum size=0,right of=int2] (xn1) {};
                    \node [block, draw=none, right of=xn1] (vacio) {$\dots$};
                    \node [inner sep=0,minimum size=0,right of=vacio] (x3) {};
                    \node [block, right of=x3] (int3) {$\int$};
                    \node [inner sep=0,minimum size=0,right of=int3] (x2) {};
                    \node [block, right of=x2] (int4) {$\int$};
                    \node [inner sep=0,minimum size=0,right of=int4] (x1) {};
                    \node [output, right of=x1] (salida) {};

                    \node [block, below of=int1] (a1) {$-a_1$};
                    \node [block, below of=a1] (a2) {$-a_2$};
                    \node [block, draw=none, below of=a2] (avacio) {$\vdots$};
                    \node [block, below of=avacio] (an1) {$-a_{n-1}$};
                    \node [block, below of=an1] (an) {$-a_n$};

                    \draw [->] (entrada) -- node[name=u] {$u(t)$} (uno);
                    \draw [->] (uno)     -- (s1);
                    \draw [->] (s1)      -- (int1);
                    \draw [->] (int1)    -- node[name=xn] {$x_n$} (int2);
                    \draw [->] (int2)    -- node[name=xn1] {$x_{n-1}$} (vacio);
                    \draw [->] (vacio)   -- node[name=x3] {$x_3$} (int3);
                    \draw [->] (int3)    -- node[name=x2] {$x_2$} (int4);
                    \draw [->] (int4)    -- node[name=x1] {$x_1$} (salida);

                    \draw [->] (xn)  |- (a1);
                    \draw [->] (xn1) |- (a2);
                    \draw [->] (x2)  |- (an1);
                    \draw [->] (x1)  |- (an);

                    \draw [->] (a1)  -| (s1.324);
                    \draw [->] (a2)  -| (s1.288);
                    \draw [->] (an1) -| (s1.252);
                    \draw [->] (an)  -| (s1.216);
                \end{tikzpicture}}
            \end{figure}

        \begin{description}
            \item [Observaciones.] \mbox{}\\
            \begin{enumerate}
                \item Polinomio Característico

                \begin{equation}
                    \det{(sI - A_c)} = s^n + a_1 s^{n-1} + \dots + a_{n-1} s + a_n = \Pi(s) \nonumber
                \end{equation}

                \item Retroalimentación de Estado

                Sea $u = f_c x_c + v$, donde $f_c = (a_n - \bar{a}_n)(a_{n-1} - \bar{a}_{n-1})\dots(a_1 - \bar{a}_1)$. Entonces el sistema de lazo cerrado es:

                \begin{equation}
                    \frac{d}{dt} x_c = A_{f_c} x_c + b_c v
                \end{equation}

                donde $A_{f_c} = A_c + b_c f_c$, es decir:

                \begin{equation}
                    A_{f_c} =
                    \begin{pmatrix}
                    0 & 1 & 0 & \dots & 0 & 0 & 0 \\
                    0 & 0 & 1 & \dots & 0 & 0 & 0 \\
                    0 & 0 & 0 & \dots & 0 & 0 & 0 \\
                    \vdots & \vdots & \vdots & & \vdots & \vdots & \vdots \\
                    0 & 0 & 0 & \dots & 0 & 1 & 0 \\
                    0 & 0 & 0 & \dots & 0 & 0 & 1 \\
                    -\bar{a}_{n} & -\bar{a}_{n-1} & -\bar{a}_{n-2} & \dots & -\bar{a}_{3} & -\bar{a}_{2} & -\bar{a}_{1}
                    \end{pmatrix}
                \end{equation}

                y su polinomio característico es $\det{(sI - A_{f_c})} = s^n + \bar{a}_1 s^{n-1} + \dots + \bar{a}_{n-1} s + \bar{a}_n$.

            \end{enumerate}
        \end{description}

    \newpage
    \section{Propiedades de la matriz de controlabilidad}

        \begin{enumerate}
            \item Matriz de controlabilidad del par $(A_c, b_c)$.

                \begin{equation}
                    C_{(A_c,b_c)} =
                    \begin{pmatrix}
                    0 & 0 & 0 & \dots & 1 \\
                    \vdots & \vdots & \vdots & & \vdots \\
                    0 & 0 & 1 & \dots & * \\
                    0 & 1 & * & \dots & * \\
                    1 & * & * & \dots & *
                    \end{pmatrix}
                \end{equation}

                lo que implica:

                \begin{equation}
                    \det{C_{(A_c, b_c)}} = \pm 1
                \end{equation}

            \item Invarianza de la matriz de controlabilidad bajo cambio de base.

                Sea el par $(A, b)$ controlable, sea $T$ una matriz de cambio de base y sean $A_1 = T^{-1} A T$ y $b_1 = T^{-1} b$ las matrices de nuestra nueva base.

                \begin{multline}
                C_{(A_1, b_1)} =
                \begin{pmatrix}
                b_1 & A_1 b_1 & \dots & A^{n-1} b_1
                \end{pmatrix} = \\
                \begin{pmatrix}
                T^{-1} b & T^{-1} A T T^{-1} b & \dots & (T^{-1} A T \dots T^{-1} A T) T^{-1} b
                \end{pmatrix} = \\
                \begin{pmatrix}
                T^{-1} b & T^{-1} A b & \dots & T^{-1} A^{n-1} b
                \end{pmatrix} = \\
                T^{-1}
                \begin{pmatrix}
                b & A b & \dots & A^{n-1} b
                \end{pmatrix} =
                T^{-1} C_{(A, b)} \nonumber
                \end{multline}

                por lo que, podemos notar la siguiente correspondencia:

                \begin{equation*}
                    C_{(A_1, b_1)} = \frac{C_{(A, b)}}{T} 
                \end{equation*}

                Mas notablemente podemos notar una manera de calcular la transformación lineal a una forma controlable.

                \begin{equation}
                    T = C_{(A, b)} C_{(A_1, b_1)}^{-1}
                \end{equation}

            \item Invarianza de la matriz de controlabilidad bajo retroalimentación de estado, $u = f^T x + v$.

                Sea $A_f = A + b f^T$ la matriz A del sistema bajo la retroalimentación de estado $u = f^T x + v$. Tendremos que la matriz de controlabilidad de este sistema retroalimentado será:

                \begin{multline*}
                C_{(A_f, b)} =
                \begin{pmatrix}
                b & A_f b & \dots & A_f^{n-1} b
                \end{pmatrix} = \\
                \begin{pmatrix}
                b & \left( A + b f^T \right) b & \dots & \left( A + b f^T \right)^{n-1} b
                \end{pmatrix}
                \end{multline*}

                en donde podemos notar que los terminos van obteniendo la siguiente forma:

                \begin{eqnarray*}
                    \left( A + b f^T \right) b & = & A b + b (f^T b) = A b + k_1 b\\
                    \left( A + b f^T \right)^{2} b & = & A^2 b + k_1 A b + b (f^T A b + k_1 (f^T b)) \\
                    & = & A^2 b + k_1 A b + k_2 b \\
                    & \vdots & \\
                    \left( A + b f^T \right)^{n-1} b & = & A^{n-1} b + k_1 A^{n-2} b + \dots + k_{n-2} A b + k_{n-1} b
                \end{eqnarray*}

                en donde los terminos $k_i$ estan relacionados unicamente con $f_T$ y $b$, y dejan de fuera a un termino $b$, por lo que es inmediato ver que lo podemos reescribir de la siguiente manera:

                \begin{equation*}
                    C_{(A_f, b)} =
                    \begin{pmatrix}
                    b & A b & \dots & A^{n-1} b
                    \end{pmatrix} \mathbb{X}
                \end{equation*}

                \begin{equation}
                    C_{(A_f, b)} = C_{(A, b)} \mathbb{X}
                \end{equation}

                donde $\mathbb{X}$ toma la forma:

                \begin{equation*}
                    \mathbb{X} =
                    \begin{pmatrix}
                        1 & k_1 & k_2 & \dots & k_{n-1} \\
                        0 & 1 & k_1 & \dots & k_{n-2} \\
                        0 & 0 & 1 & \dots & k_{n-3} \\
                        \vdots & \vdots & \vdots & & \vdots \\
                        0 & 0 & 0 & \dots & 1
                    \end{pmatrix}
                \end{equation*}

                lo cual implica que $\det{\mathbb{X}} = \pm 1$, es decir:

                \begin{equation*}
                    \det{C_{(A, b)}} \ne 0 \implies \det{C_{(A_f, b)}} \ne 0
                \end{equation*}

                en particular nosotros tenemos que:

                \begin{equation}
                    \det{C_{(A, b)}} = \det{C_{(A_f, b)}}
                \end{equation}

                Dada la invarianza de la matriz de controlabilidad ($C_{(A, b)}$) bajo cambio de base y retroalimentación de estado, Brunovskii estudió la controlabilidad de los sistemas lineales con todos sus valores propios (polos) en el origen\footnote{El teorema de Brunovskii en realidad esta redactado para sistemas multientradas, y se expresa en matrices diagonales por bloques de tamaño $k_i \times (k_i + 1)$, con $\sum_{i=0}^n k_i = n$, donde $k_i$ son los indices de controlabilidad}:

                \begin{equation}
                    \begin{pmatrix}
                        A_{Br} & b_{Br}
                    \end{pmatrix} =
                    \begin{amatrix}{6}
                        0 & 1 & 0 & \dots & 0 & 0 & 0 \\
                        0 & 0 & 1 & \dots & 0 & 0 & 0 \\
                        0 & 0 & 0 & \dots & 0 & 0 & 0 \\
                        \vdots & \vdots & \vdots & & \vdots & \vdots & \vdots \\
                        0 & 0 & 0 & \dots & 1 & 0 & 0 \\
                        0 & 0 & 0 & \dots & 0 & 1 & 0 \\
                        0 & 0 & 0 & \dots & 0 & 0 & 1 \\
                    \end{amatrix}
                \end{equation}

                A los indeices $k_i$ de los polinomios mínimos se les denomina indices de controlabilidad. En nuestro caso particular, existe solamente un indice de controlabilidad; $k_i = n$.

            \item Invarianza de los ceros del sistema bajo retroalimentación de estado.

            Sea la matriz sistema del sistema en lazo abierto la siguiente:

            \begin{equation*}
                \Sigma(s) =
                \begin{pmatrix}
                    sI - A & b \\
                    -c^T & d
                \end{pmatrix}
            \end{equation*}

            entonces, la matriz sistema bajo la retroalimentación será:

            \begin{equation*}
                \Sigma_{lc}(s) =
                \begin{pmatrix}
                    sI - (A + b f^T) & b \\
                    -(c^T + d f^T) & d
                \end{pmatrix}
            \end{equation*}

            notando que:
            \begin{equation*}
                \begin{pmatrix}
                    sI - A & b \\
                    -c^T & d
                \end{pmatrix}
                \begin{pmatrix}
                    I & 0 \\
                    -f^T & 1
                \end{pmatrix} =
                \begin{pmatrix}
                    sI - (A + b f^T) & b \\
                    -(c^T + d f^T) & d
                \end{pmatrix}
            \end{equation*}

            por lo que:

            \begin{equation}
                \det{\Sigma(s)} = \det{\Sigma_{lc}(s)}
            \end{equation}

            Se concluye que la retroalimentaión de estado no afecta a los ceros del sistema; solo puede modificar a los polos controlables.
        \end{enumerate}

    \newpage
    \section{Formas canónicas}
        Sea un sistema lineal invariante en el tiempo, una entrada, una salida (SISO), descrito por la siguiente Ecuación Diferencial Ordinaria (EDO):

        \begin{multline}
            \left( \frac{d^n}{dt^n} + a_1 \frac{d^{n-1}}{dt^{n-1}} + \dots + a_{n-1} \frac{d}{dt} + a_n \right) y(t) = \\
            \left( b_n + b_{n-1} \frac{d}{dt} + \dots + b_1 \frac{d^{n-1}}{dt^{n-1}} \right) u(t)
        \end{multline}

        \subsection{Forma canónica controlador}

        \begin{equation}
            \frac{d}{dt} x_c =
            \begin{pmatrix}
            0 & 1 & 0 & \dots & 0 & 0 & 0 \\
            0 & 0 & 1 & \dots & 0 & 0 & 0 \\
            0 & 0 & 0 & \dots & 0 & 0 & 0 \\
            \vdots & \vdots & \vdots & & \vdots & \vdots & \vdots \\
            0 & 0 & 0 & \dots & 0 & 1 & 0 \\
            0 & 0 & 0 & \dots & 0 & 0 & 1 \\
            -a_{n} & -a_{n-1} & -a_{n-2} & \dots & -a_{3} & -a_{2} & -a_{1}
            \end{pmatrix} x_c +
            \begin{pmatrix}
            0 \\
            0 \\
            0 \\
            \vdots \\
            0 \\
            0 \\
            1
            \end{pmatrix} u
        \end{equation}

        \begin{equation}
            y =
            \begin{pmatrix}
                b_n & b_{n-1} & b_{n-2} & \dots & b_2 & b_1
            \end{pmatrix} x_c
        \end{equation}

        \begin{figure*}
            \centering
            \resizebox{\textwidth}{!}{
                \tikzstyle{input} = [coordinate]
                \tikzstyle{output} = [coordinate]
                \tikzstyle{empty} = [coordinate]
                \tikzstyle{block} = [draw, rectangle, minimum height=2.25em, minimum width=2.25em]
                \tikzstyle{sum} = [draw, circle]
                \tikzstyle{init} = [pin edge={to-, thin, black}]

                \begin{tikzpicture}[auto, node distance=1.1cm, >=latex']
                    \node [input, name=entrada] {};
                    \node [sum, right of=entrada] (s1) {$+$};
                    \node [block, right of=s1] (int1) {$\int$};
                    \node [inner sep=0,minimum size=0,right of=int1] (xn) {};
                    \node [block, right of=xn] (int2) {$\int$};
                    \node [inner sep=0,minimum size=0,right of=int2] (xn1) {};
                    \node [block, draw=none, right of=xn1] (vacio) {$\dots$};
                    \node [inner sep=0,minimum size=0,right of=vacio] (x3) {};
                    \node [block, right of=x3] (int3) {$\int$};
                    \node [inner sep=0,minimum size=0,right of=int3] (x2) {};
                    \node [block, right of=x2] (int4) {$\int$};
                    \node [inner sep=0,minimum size=0,right of=int4] (x1) {};
                    \node [block, right of=x1] (bm) {$b_n$};
                    \node [sum, right of=bm] (s2) {$+$};
                    \node [output, right of=s2] (salida) {};

                    \node [block, below of=int1] (a1) {$-a_1$};
                    \node [block, below of=a1] (a2) {$-a_2$};
                    \node [block, draw=none, below of=a2] (avacio) {$\vdots$};
                    \node [block, below of=avacio] (an1) {$-a_{n-1}$};
                    \node [block, below of=an1] (an) {$-a_n$};

                    \node [block, above of=bm] (bm1) {$b_{n-1}$};
                    \node [block, draw=none, above of=bm1] (bvacio) {$\vdots$};
                    \node [block, above of=bvacio] (b1) {$b_{2}$};
                    \node [block, above of=b1] (b0) {$b_{1}$};

                    \draw [->] (entrada) -- node[name=u] {$u(t)$} (s1);
                    \draw [->] (s1)      -- (int1);
                    \draw [->] (int1)    -- node[name=xn] {$x_n$} (int2);
                    \draw [->] (int2)    -- node[name=xn1] {$x_{n-1}$} (vacio);
                    \draw [->] (vacio)   -- node[name=x3] {$x_3$} (int3);
                    \draw [->] (int3)    -- node[name=x2] {$x_2$} (int4);
                    \draw [->] (int4)    -- node[name=x1] {$x_1$} (bm);
                    \draw [->] (bm)      -- (s2);
                    \draw [->] (s2)      -- node[name=y] {$y(t)$} (salida);

                    \draw [->] (xn)  |- (a1);
                    \draw [->] (xn1) |- (a2);
                    \draw [->] (x2)  |- (an1);
                    \draw [->] (x1)  |- (an);

                    \draw [->] (a1)  -| (s1.324);
                    \draw [->] (a2)  -| (s1.288);
                    \draw [->] (an1) -| (s1.252);
                    \draw [->] (an)  -| (s1.216);

                    \draw [->] (xn)  |- (b0);
                    \draw [->] (xn1) |- (b1);
                    \draw [->] (x2)  |- (bm1);

                    \draw [->] (b0)  -| (s2.45);
                    \draw [->] (b1)  -| (s2.90);
                    \draw [->] (bm1) -| (s2.135);
                \end{tikzpicture}}
        \end{figure*}

        \begin{description}
            \item [Polos.]

            \begin{equation}
                \det{(sI - A_c)} = s^n + a_1 s^{n-1} + a_2 s^{n-2} + \dots + a_{n-1} s + a_n
            \end{equation}

            \item [Ceros.]

            \begin{equation}
                \det{\Sigma(s)} = b_1 s^{n-1} + \dots + b_{n-1} s + b_n
            \end{equation}

            \item [Matriz de Controlabilidad.]

            \begin{equation}
                C_{(A_c, b_c)} =
                \begin{pmatrix}
                    b_c & A_c b_c & \dots & A_c^{n-1} b_c
                \end{pmatrix} \implies \det{C_{(A_c, b_c)}} = 1
            \end{equation}
        \end{description}

        \subsection{Forma canónica controlabilidad}

        \begin{equation}
            \frac{d}{dt} x_c =
            \begin{pmatrix}
            -a_{1} & 1 & 0 & \dots & 0 & 0 & 0 \\
            -a_{2} & 0 & 1 & \dots & 0 & 0 & 0 \\
            -a_{3} & 0 & 0 & \dots & 0 & 0 & 0 \\
            \vdots & \vdots & \vdots & & \vdots & \vdots & \vdots \\
            -a_{n-2} & 0 & 0 & \dots & 0 & 1 & 0 \\
            -a_{n-1} & 0 & 0 & \dots & 0 & 0 & 1 \\
            -a_{n} & 0 & 0 & \dots & 0 & 0 & 0
            \end{pmatrix} x_c +
            \begin{pmatrix}
            0 \\
            0 \\
            0 \\
            \vdots \\
            0 \\
            0 \\
            1
            \end{pmatrix} u
        \end{equation}

        \begin{equation}
            y =
            \begin{pmatrix}
                \beta_n & \beta_{n-1} & \beta_{n-2} & \dots & \beta_2 & \beta_1
            \end{pmatrix} x_c
        \end{equation}

        donde:

        \begin{equation}
            \begin{pmatrix}
                \beta_1 \\
                \beta_2 \\
                \beta_3 \\
                \vdots \\
                \beta_{n-1} \\
                \beta_{n}
            \end{pmatrix} =
            \begin{pmatrix}
            1 & 0 & 0 & \dots & 0 & 0 \\
            -a_{1} & 1 & 0 & \dots & 0 & 0 \\
            -a_{2} & -a_{1} & 1 & \dots & 0 & 0 \\
            \vdots & \vdots & \vdots & & \vdots & \vdots \\
            -a_{n-3} & -a_{n-4} & -a_{n-5} & \dots & 0 & 0 \\
            -a_{n-2} & -a_{n-3} & -a_{n-4} & \dots & 1 & 0 \\
            -a_{n-1} & -a_{n-2} & -a_{n-3} & \dots  & -a_{1} & 1
            \end{pmatrix} ^{-1}
            \begin{pmatrix}
                b_1 \\
                b_2 \\
                b_3 \\
                \vdots \\
                b_{n-2} \\
                b_{n-1} \\
                b_{n}
            \end{pmatrix}
        \end{equation}

        \begin{description}
            \item [Polos.]

            \begin{equation}
                \det{(sI - A_{co})} = s^n + a_1 s^{n-1} + a_2 s^{n-2} + \dots + a_{n-1} s + a_n
            \end{equation}

            \item [Ceros.]

            \begin{equation}
                \det{\Sigma(s)} = b_1 s^{n-1} + \dots + b_{n-1} s + b_n
            \end{equation}

            \item [Matriz de Controlabilidad.]

            \begin{equation}
                C_{(A_{co}, b_{co})} =
                \begin{pmatrix}
                    b_{co} & A_{co} b_{co} & \dots & A_{co}^{n-1} b_{co}
                \end{pmatrix} \implies \det{C_{(A_{co}, b_{co})}} = \pm 1
            \end{equation}
        \end{description}