\ifebook
    \documentclass[a5paper, 15pt]{book}
    \usepackage[margin=2mm]{geometry}
    \usepackage{mathpazo}
    \usepackage{eulervm}
\else
    \documentclass{tufte-book}
\fi

\usepackage[spanish,es-noquoting]{babel}
\usepackage[utf8]{inputenc}
\usepackage{amsmath}
\usepackage{amsthm}
\usepackage{amssymb}
\usepackage{xfrac}
\usepackage{graphicx}
\usepackage{bbm}
\usepackage{enumerate}
\usepackage{tikz}
\usepackage{todonotes}

\usetikzlibrary{shapes, arrows, decorations.markings}

\setcounter{secnumdepth}{0}
\numberwithin{equation}{chapter}

\newenvironment{amatrix}[1]{%
  \left(\begin{array}{@{}*{#1}{c}|c@{}}
}{%
  \end{array}\right)
}

\newtheorem{definicion}{Definición}
\numberwithin{definicion}{chapter}
\newtheorem{lema}{Lema}
\numberwithin{lema}{chapter}
\newtheorem{teorema}{Teorema}
\numberwithin{teorema}{chapter}
\newtheorem{nota}{Nota}
\numberwithin{nota}{chapter}
\newtheorem{corolario}{Corolario}
\numberwithin{corolario}{chapter}

\newcommand{\tarea}[2][]
    {\todo[linecolor=black!50!white, backgroundcolor=black!30!white, bordercolor=black!30!white, #1]{#2}}

\newcommand{\faltante}[2][]
    {\todo[linecolor=red!50!white, backgroundcolor=red!30!white, bordercolor=red!30!white, #1]{#2}}
