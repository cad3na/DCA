% Compila en uno de dos formatos diferentes dependiendo del valor de ebook, el
% cual puede ser pasado por medio de una cadena de caracteres al mandar compilar
% el archivo, un ejemplo es este:
%
% pdflatex -interaction=batchmode "\newif\ifebook\ebookfalse\input{archivo.tex}"
%
% pdflatex -interaction=batchmode "\newif\ifebook\ebooktrue\input{archivo.tex}"
%
% Estas dos entradas produciran archivos diferentes con un formato especifico
\ifebook
    \documentclass[a5paper, 15pt]{book} % Papel: A5, Letra: 15pt
    \usepackage[margin=2mm]{geometry} % Margenes estrechos
    \usepackage{mathpazo} % Letra: Palatino
    \usepackage{eulervm} % Letra para matematicas: Euler
    \usepackage{wrapfig}
    \newenvironment{marginfigure} % Definicion complementaria
        {\begin{wrapfigure}{r}{0.2\textwidth}}
        {\end{wrapfigure}}
\else
    \documentclass{tufte-book} % Papel: Letter, Estilo: Tufte
\fi
% Paquetes a utilizar
\usepackage[spanish, es-noquoting]{babel} % Español
\usepackage[utf8]{inputenc} % Texto de entrada en UTF-8
\usepackage{amsmath}        % Paquete con definiciones matematicas
\usepackage{amsthm}         % Paquete para definir ambientes de teoremas
\usepackage{amssymb}        % Paquete con simbolos matematicos
\usepackage{xfrac}          % Paquete para usar fracciones inclinadas
\usepackage{graphicx}       % Paquete para incluir imagenes
\usepackage{bbm}            % Paquete con simbolos de campos
\usepackage{enumerate}      % Paquete para crear ambientes de listas enumeradas
\usepackage{tikz}           % Paquete para crear diagramas
\usepackage{todonotes}      % Paquete para crear anotaciones
\usepackage{mathrsfs}       % Paquete para mas simbolos de campos
\usepackage{steinmetz}      % Paquete para escribir numeros complejos
% Subpaquete para diagramas de bloques
\usetikzlibrary{shapes, arrows, decorations.markings}
% Estilos comunes de bloques para diagramas
\tikzstyle{input} = [coordinate]
\tikzstyle{output} = [coordinate]
\tikzstyle{init} = [pin edge={to-, thin, black}]
% Hace que la numeración dependa del numero de capitulo
\setcounter{secnumdepth}{0}
\numberwithin{equation}{chapter}
% Definición de matriz adjunta
\newenvironment{amatrix}[1]
    {\left(\begin{array}{@{}*{#1}{c}|c@{}}}
    {\end{array}\right)}
% Definiciones de teoremas
\newtheorem{definicion}{Definición} % definicion de teorema: Definición
\numberwithin{definicion}{chapter}  % numeracion de teorema por capitulo
\newtheorem{lema}{Lema}
\numberwithin{lema}{chapter}
\newtheorem{teorema}{Teorema}
\numberwithin{teorema}{chapter}
\newtheorem{nota}{Nota}
\numberwithin{nota}{chapter}
\newtheorem{corolario}{Corolario}
\numberwithin{corolario}{chapter}
% Definiciones de operadores
\DeclareMathOperator{\rango}{rango}
\DeclareMathOperator{\imagen}{Im}
\DeclareMathOperator{\grado}{grado}
% Definiciones para recordatorios por hacer
\newcommand{\tarea}[2][]
    {\todo[
        linecolor=black!50!white,       % Color de linea: Gris
        backgroundcolor=black!30!white, % Color de fondo: Gris claro
        bordercolor=black!30!white,     % Color de borde: Gris claro
        #1]{#2}}
\newcommand{\faltante}[2][]
    {\todo[
        linecolor=red!50!white,        % Color de linea: Rojo
        backgroundcolor=red!30!white,  % Color de fondo: Rojo claro
        bordercolor=red!30!white,      % Color de borde: Rojo claro
        #1]{#2}}
